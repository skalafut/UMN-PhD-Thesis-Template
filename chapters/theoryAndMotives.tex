%%%%%%%%%%%%%%%%%%%%%%%%%%%%%%%%%%%%%%%%%%%%%%%%%%%%%%%%%%%%%%%%%%%%%%%%%%%%%%%%
%neutrino_physics.tex: Chapter on neutrino physics:
%%%%%%%%%%%%%%%%%%%%%%%%%%%%%%%%%%%%%%%%%%%%%%%%%%%%%%%%%%%%%%%%%%%%%%%%%%%%%%%%
\chapter{Theory of the Standard Model and Extensions}
\label{wrBosonAndHeavyNu}
%%%%%%%%%%%%%%%%%%%%%%%%%%%%%%%%%%%%%%%%%%%%%%%%%%%%%%%%%%%%%%%%%%%%%%%%%%%%%%%%
\section{Standard Model Particles and Interactions}
The Standard Model of particle physics (SM) postulates that the universe can be reduced down to a small 
number of particles and interactions between them.  During the past four decades the SM has successfully 
explained many phenomena observed in physics and astronomy, and predicted the existence of many particles 
which were later confirmed by experiments.

The theoretical framework of the SM was proposed by Glashow, Weinberg and Salam in 1967 \cite{weinbergSM,salamSM}.  
In their theory, the fundamental building blocks of matter are organized into three generations.  Each 
generation has two quarks, a charged lepton paired with a neutral lepton, and particle and anti-particle 
versions of all four.  The human body and every material or substance that humans are able to sense are 
built from the first generation of particles, which consists of the electron and electron neutrino, and 
the up and down quark.  The second generation contains the muon and muon neutrino, and the charm and 
strange quark, while the third generation contains the tau and tau neutrino, and the top and bottom 
quark.

There are three interactions in the SM, and they are mediated by additional particles.  Mathematically, 
these interactions are represented by gauge groups, and the mediator particles are represented by 
combinations of group generators multiplied by vector fields.  The strong interaction, represented by 
the $SU(3)_{C}$ group, occurs between quarks and is mediated 
by gluons, which are represented by vector fields multiplied by the $SU(3)$ group generators.  The proton, 
built from three quarks with a net positive charge, is stable because the strong interaction attraction 
between quarks overwhelms the electromagnetic repulsion between quarks with same sign electric charges.  
The electromagnetic interaction occurs between all particles with electric charge 
and is mediated the photon.  The $\pi^{0}$ contains a linear combination of an up anti-up quark pair and 
a down anti-down quark pair, and primarily decays through the electromagnetic interaction to two photons, 
$\pi^{0} \rightarrow \gamma\gamma$.  The weak interaction occurs 
between all particles in all generations, and is mediated by the $W^{\pm}$ and $Z$ bosons.  For reasons 
discussed later, the photon, $W^{\pm}$ and $Z$ bosons are represented mathematically by linear combinations 
of vector fields multiplied by the $SU(2)_{L}$ and $U(1)$ group generators.

The dynamics of the strong interaction is driven by a mathematical construct called color charge.  Every 
quark is produced with one of six possible color charges: red, anti-red, blue, anti-blue, green, or anti-
green.  Gluons that mediate the strong interaction carry two different color charges, like red and blue, 
or anti-blue and green.  Color charge is conserved in strong interactions, and all quasi-
stable\footnote{mean lifetime $\gtrsim 10^{-25}$ seconds} quark bound states 
are color neutral.  Thus, a pair of quarks that are red and anti-red, or a triplet of quarks that are 
red, blue and green constitute quasi-stable bound states.  When a proton-proton interaction in the LHC 
produces a quark, the strong interaction works to keep that quark in a color neutral bound state.  As the 
quark moves away from quarks in colliding protons, potential energy accumulates, similar to stretching a 
spring.  Once enough potential energy accumulates, the strong interaction converts the potential energy 
into a massless gluon that decays into two new quarks.  At least one of the two new quarks has 
the correct color to form a color neutral bound state with the original quark.  The other new quark may 
have the correct color to form a color neutral, quark triplet bound state.  If not, the process of 
potential energy accumulation followed by quark pair production continues until the sum total color 
charge of all new quarks produced is neutral.  A gluon produced by a proton-proton interaction 
decays into two quarks of different colors, and the process described for one quark ensues for both 
quarks.  The shower of color neutral hadrons that results from the production of one quark or gluon 
is called a jet for the remainder of this thesis.

The weak interaction has garnered much attention from the particle physics community for many decades.  
To explain the $\Beta$-decay process, in 1932 Fermi proposed a model for the weak interaction based on 
the electromagnetic interaction.  His model predicted the existence of a neutral lepton, the electron 
anti-neutrino, which was later proven \cite{firstNuDiscovery} and became part of the SM.  However, Fermi's 
model did not correctly predict the branching fraction of many hadrons that decay through the weak 
interaction, like $K^{+} \rightarrow 2\pi, 3\pi$.  In 1956 Lee and Yang proposed the weak interaction 
violated parity, and this was proven experimentally in the following year \cite{weakParityViolation}.  
The weak interaction in the SM violates parity, and an important consequence of this is the chirality of 
the neutrinos.  Anti-neutrinos are left-handed, and neutrinos are right-handed.  The implications of this 
constraint on neutrino masses will be discussed later.

The SM postulates that the four generators of the $SU(2)_{L} \times U(1)$ groups transform into the massless 
photon and massive $W^{\pm}$ and $Z$ bosons through the Brout-Englert-Higgs (BEH) mechanism.  This mechanism 
adds four degrees of freedom to the SM, in the form of a complex doublet $\Phi$ representing four scalar 
particles, which obey the Lagrangian $\Ell_{H}$:

\begin{align}
	\Phi &= \begin{bmatrix}
	\phi^{+} \\
	\phi^{0}
	\end{bmatrix}
\end{align}

\begin{equation}
	\Ell_{H} = (D_{\mu}\Phi)^{\dagger}D^{\mu}\Phi - V(\Phi)
\end{equation}

where $V(\Phi) = \frac{1}{2}(|\Phi|^{2} - \frac{\nu^{2}}{2})$ is the Higgs potential, and 
$D_{\mu} = \partial_{\mu} + ig_{L}\tau^{j}A^{j}_{\mu} + i\frac{g'}{2}YB_{\mu}$ describes the propagation 
of the Higgs doublet $\Phi$ and its couplings to the $SU(2)_L$ generators $\tau^{j}$ and massless vector 
fields $A^{j}_{\mu}$, and the $U(1)$ generator $Y$ and massless vector field $B_{\mu}$.  $g_{L}$ and 
$g'$ are the weak and electromagnetic interaction coupling strengths.  The Higgs doublet $\Phi$ takes an 
average value $<\Phi> = (0  \nu/sqrt{2})$ so that the Higgs potential is minimized.  When $\Phi$ is set 
to this average value and explicit matrices and values are plugged in for the group generators, the subset 
of the Lagrangian $\Ell_{H}$ without partial derivatives $\partial_{\mu}$ and the Higgs potential $V(\Phi)$:

\begin{equation}
	\Ell_{HK} = [(ig_{L}\tau^{j}A^{j}_{\mu} + i\frac{g'}{2}YB_{\mu})\Phi]^{\dagger}(ig_{L}\tau^{j}A^{j\mu} + i\frac{g'}{2}YB^{\mu})\Phi
\end{equation}

reduces to:

\begin{equation}
	\Ell_{HK} = \frac{\nu^{2}}{8}[g^{2}_{L}(A^{1}_{\mu} + iA^{2}_{\mu})(A^{1\mu} - iA^{2\mu}) + (g'B_{\mu} - g_{L}A^{3}_{\mu})^{2}]
\end{equation}

Defining the photon vector field $A_{\mu}$, and weak boson vector fields $W^{\pm}_{\mu}$ and $Z_{\mu}$ as:

\begin{equation}
	W^{\pm}_{\mu} \equiv \frac{1}{\sqrt{2}}(A^{1}_{\mu} \pm iA^{2}_{\mu}), 
	Z_{\mu} \equiv \frac{1}{\sqrt{g'^{2} + g^{2}_{L}}}(g'B_{\mu} - g_{L}A^{3}_{\mu}), 
	A_{\mu} \equiv \frac{1}{\sqrt{g'^{2} + g^{2}_{L}}}(g_{L}B_{\mu} + g'A^{3}_{\mu})
\end{equation}

yields the Lagrangian:

\begin{equation}
	\Ell_{HK} = (\frac{\nu g_{L}}{2})^{2}W^{+}_{\mu}W^{-\mu} + \frac{1}{2}(\frac{\nu \bar{g}}{2})^{2}Z_{\mu}Z^{\mu} + 0A_{\mu}A^{\mu}
\end{equation}

where $\bar{g} \equiv \sqrt{g'^{2} + g^{2}_{L}}$.  From this Lagrangian, the photon $A_{\mu}$ is massless, 
the $Z$ boson has mass $m_{Z} = \nu\bar{g}/2$, and the $W^{\pm}$ bosons have mass $m_{W} = \nu g_{L}/2$.  
Three of the four scalar fields introduced by the BEH mechanism are consumed to give mass to the $Z$ 
and $W^{\pm}$ bosons.  Recent experimental proof of the fourth scalar field, the Higgs boson, 
\cite{} proves that the $Z$ and $W^{\pm}$ bosons acquire mass through the BEH mechanism.

Following experimental evidence, quarks and charged leptons can acquire mass through two methods without 
extensions to the SM.  Their masses can be added directly to the SM Lagrangian, or additional Higgs fields 
can be added to the BEH mechanism.  In either case, mass terms of the form $-mf\bar{f}$, where the fermion 
$f$ represents a quark or charged lepton, appear in the SM Lagrangian.  In the coordinate basis where a 
fermion field $f = (\chi_{L},\chi_{R})$ consists of a right-handed component $\chi_{R}$ and left-handed 
component $\chi_{L}$, a fermion mass term in the SM Lagrangian is written as:

\begin{equation}
	\Ell_{D} = -m\bar{f}f = -m\chi^{\dagger}_{L}\chi_{R} - m\chi^{\dagger}_{R}\chi_{L}
\end{equation}

This type of mass term, called a Dirac mass, contains the product of left and right-handed fields.  Quarks 
and charged leptons have both left and right-handed fields, corresponding to opposite charge particles and 
anti-particles, so their masses can be assigned using Dirac mass terms.

Neutrinos play a special role in the SM.  The SM postulates that they are neutral, massless, and only interact 
with other particles through the weak interaction.  Due to parity violation of the weak interaction, anti-neutrinos $\bar{\nu_{l}}$ 
are always right-handed, and neutrinos $\nu_{l}$ are always 
left-handed.  Neutrino experiments \cite{NOvAresults,mainzPhaseIIResults,t2kResults} have proven that neutrinos 
have mass, but otherwise are consistent with SM predictions.  Non-zero neutrino masses can only be explained 
by extensions to the SM.


\section{Left-Right Symmetric Extensions of the Standard Model}


%%%%%%%%%%%%%%%%%%%%%%%%%%%%%%%%%%%%%%%%%%%%%%%%%%%%%%%%%%%%%%%%%%%%%%%%%%%%%%%%
