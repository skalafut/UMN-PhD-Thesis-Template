%%%%%%%%%%%%%%%%%%%%%%%%%%%%%%%%%%%%%%%%%%%%%%%%%%%%%%%%%%%%%%%%%%%%%%%%%%%%%%%%
%neutrino_physics.tex: Chapter on neutrino physics:
%%%%%%%%%%%%%%%%%%%%%%%%%%%%%%%%%%%%%%%%%%%%%%%%%%%%%%%%%%%%%%%%%%%%%%%%%%%%%%%%
\chapter{Theoretical Motivations}
\label{wrBosonAndHeavyNu}
%%%%%%%%%%%%%%%%%%%%%%%%%%%%%%%%%%%%%%%%%%%%%%%%%%%%%%%%%%%%%%%%%%%%%%%%%%%%%%%%
Experimental evidence \cite{NOvAresults,mainzPhaseIIResults,t2kResults} of neutrinos with non-zero masses motivates extensions to the Standard Model (SM).  
Before discussing SM extensions, the methods through which the weak bosons $W^{\pm},Z$ and fermions acquire mass 
in the SM are discussed.  Subsequently, a SM extension is presented to explain neutrino masses using the 
mass generation methods of the SM, and other experimentally observed phenomena not explained by the SM.  
The chapter concludes by presenting phenomenological aspects of this SM extension relevant to experimental 
detection.

\section{Particle Masses in the Standard Model}
\label{sec:massInSM}
The SM postulates that the four generators of the $SU(2)_{L} \times U(1)$ groups transform into the massless 
photon and massive $W^{\pm}$ and $Z$ bosons through the Brout-Englert-Higgs (BEH) mechanism.  This mechanism 
adds four degrees of freedom to the SM, in the form of a complex doublet $\Phi$ representing four bosonic 
particles, which obey the Lagrangian $\Ell_{H}$:

\begin{align}
	\Phi &= \begin{bmatrix}
	\phi^{+} \\
	\phi^{0}
	\end{bmatrix}
\end{align}

\begin{equation}
	\Ell_{H} = (D_{\mu}\Phi)^{\dagger}D^{\mu}\Phi - V(\Phi)
\end{equation}

where $V(\Phi) = \frac{1}{2}(|\Phi|^{2} - \frac{\nu^{2}}{2})$ is the Higgs potential, and 
$D_{\mu} = \partial_{\mu} + ig_{L}\tau^{j}A^{j}_{\mu} + i\frac{g'}{2}YB_{\mu}$ describes the propagation 
of the Higgs doublet $\Phi$ and its couplings to the $SU(2)_L$ generators $\tau^{j}$ and massless vector 
fields $A^{j}_{\mu}$, and the $U(1)$ generator $Y$ and massless vector field $B_{\mu}$.  $g_{L}$ and 
$g'$ determine the weak and electromagnetic interaction coupling strengths.  The Higgs doublet $\Phi$ takes a 
value $<\Phi> =$ (0  $\nu/\sqrt{2}$) so that the Higgs potential is minimized.  When $V(\Phi)$ is minimized 
and explicit matrices and values are plugged in for the group generators, the subset 
of the Lagrangian $\Ell_{H}$ without partial derivatives $\partial_{\mu}$ and the Higgs potential $V(\Phi)$:

\begin{equation}
	\Ell_{HK} = [(ig_{L}\tau^{j}A^{j}_{\mu} + i\frac{g'}{2}YB_{\mu})\Phi]^{\dagger}(ig_{L}\tau^{j}A^{j\mu} + i\frac{g'}{2}YB^{\mu})\Phi
\end{equation}

reduces to:

\begin{equation}
	\Ell_{HK} = \frac{\nu^{2}}{8}[g^{2}_{L}(A^{1}_{\mu} + iA^{2}_{\mu})(A^{1\mu} - iA^{2\mu}) + (g'B_{\mu} - g_{L}A^{3}_{\mu})^{2}]
\end{equation}

Defining the photon vector field $A_{\mu}$, and the weak boson vector fields $W^{\pm}_{\mu}$ and $Z_{\mu}$ as:

\begin{equation}
	W^{\pm}_{\mu} \equiv \frac{1}{\sqrt{2}}(A^{1}_{\mu} \pm iA^{2}_{\mu}), 
	Z_{\mu} \equiv \frac{1}{\sqrt{g'^{2} + g^{2}_{L}}}(g'B_{\mu} - g_{L}A^{3}_{\mu}), 
	A_{\mu} \equiv \frac{1}{\sqrt{g'^{2} + g^{2}_{L}}}(g_{L}B_{\mu} + g'A^{3}_{\mu})
\end{equation}

yields the Lagrangian:

\begin{equation}
	\Ell_{HK} = (\frac{\nu g_{L}}{2})^{2}W^{+}_{\mu}W^{-\mu} + \frac{1}{2}(\frac{\nu \bar{g}}{2})^{2}Z_{\mu}Z^{\mu} + 0A_{\mu}A^{\mu}
\end{equation}

where $\bar{g} \equiv \sqrt{g'^{2} + g^{2}_{L}}$.  Following from this Lagrangian, the photon $A_{\mu}$ is massless, 
the $Z$ boson has mass $m_{Z} = \nu\bar{g}/2$, and the $W^{\pm}$ bosons have mass $m_{W} = \nu g_{L}/2$.  
Three of the four scalar fields introduced by the BEH mechanism are consumed to give mass to the $Z$ 
and $W^{\pm}$ bosons.  Recent experimental evidence of the fourth scalar field \cite{combinedHiggsResult}, the Higgs boson, 
supports the SM prediction that the $Z$ and $W^{\pm}$ bosons acquire mass through the BEH mechanism.

Experimental evidence supports the prediction that quarks and charged leptons are massive particles.  In the SM, they can acquire mass 
through two methods.  Their masses can be added directly to the SM Lagrangian, or additional Higgs fields 
can be added to the BEH mechanism.  In either case, the result is mass terms of the form $-mf\bar{f}$, where $f$ is a fermion 
representing a quark or charged lepton, are added to the SM Lagrangian.  In the basis where a 
fermion field $f$ consists of a right-handed component $\chi_{R}$ and left-handed 
component $\chi_{L}$, $f = (\chi_{L},\chi_{R})$, a fermion mass term in the SM Lagrangian is written as:

\begin{equation}
	\Ell_{D} = -m\bar{f}f = -m\chi^{\dagger}_{L}\chi_{R} - m\chi^{\dagger}_{R}\chi_{L}
\end{equation}

This type of mass term, called a Dirac mass, contains the product of left and right-handed fields.  Experimental 
evidence substantiates that quarks and charged leptons, or their anti-particles, are massive and can be left or right-handed 
fields, so their masses are assigned using Dirac mass terms.

Neutrinos play a special role in the SM.  The SM postulates that they are neutral, massless fermions, and only interact 
with other particles through the weak interaction.  In addition, due to parity violation of the weak interaction, 
anti-neutrinos $\bar{\nu_{\ell}}$ are always right-handed, and neutrinos $\nu_{\ell}$ are always left-handed.  Fermions in 
the SM can only acquire mass through a Lagrangian of the form $\Ell_{D}$, which cannot be used to explain SM neutrino masses 
due to the multiplication of left and right-handed fields.  An extension of the SM is needed to explain neutrino masses.


\section{Standard Model Extensions}
\label{sec:lrsExtensions}
The SM can be extended in several ways to accommodate neutrinos with mass.  One of the simplest extensions is to 
make neutrinos Majorana fermions, instead of Dirac fermions used in the SM.  Majorana neutrinos, first proposed 
in 1937 \cite{majoranaTheory}, postulate that fermionic neutrinos are their own anti-particles, and have masses 
$m_{L},m_{R}$ given by the Lagrangian $\Ell_{M}$:

\begin{equation}
	\Ell_{M} = -m_{L}\chi^{\dagger}_{L}\chi_{L} - m_{R}\chi^{\dagger}_{R}\chi_{R}
\end{equation}

The Majorana masses can be generated through an extended Higgs model, or simply adding terms of the form $\Ell_{M}$ 
to the SM Lagrangian.  If light Majorana neutrinos exist, then the double beta decay process could occur 
with no neutrinos.  Experimental evidence of neutrinoless double beta decay has not been found 
\cite{igexDblBetaDecay,gerdaDblBetaDecay}, so a different SM extension is needed to explain neutrino masses.

Another class of SM extensions that explain the origin of neutrino masses are Left-Right Symmetric (LRS) extensions.  
First proposed in 1974 \cite{earlyLRSModel}, LRS models postulate that a precursor to the SM electromagnetic and weak interactions exists 
in the very early universe, and this interaction conserves parity and is mediated by seven massless gauge bosons.  
Shortly after the Big Bang an extension of the SM BEH mechanism transforms the seven massless gauge bosons 
into the massless photon, the SM $W^{\pm}$ and $Z$ bosons, and three heavier bosons $W^{\pm}_{R}$ (\WR) and $Z'$.  
The mechanism through which the SM $W^{\pm}$ and $Z$, and \WR and $Z'$ bosons acquire mass is discussed next, and 
is followed by a discussion of neutrino masses in LRS models.

The subset of LRS models considered here add a $SU(2)_{R}$ group to the SM $SU(2)_{L} \times U(1)$ groups.
Adding the $SU(2)_{R}$ group to the SM introduces three new, massless vector fields $\xi^{j}_{\mu}$.  In these models, 
the SM BEH mechanism is extended in two stages, and create six massive gauge bosons that mediate the weak interaction.  
In the first stage \cite{lrsHiggsStageOne}, a chiral, complex Higgs doublet $\chi_{L,R}$ is introduced 

\begin{align}
	\chi_{L,R} &= \begin{bmatrix}
	\chi^{+}_{L,R} \\
	\chi^{0}_{L,R}
	\end{bmatrix}
	\label{eq:stageOneVEV}
\end{align}

with bosonic fields that couple independently to left and right-handed gauge bosons.  The propagation and interaction of these 
fields with other massless bosons is described by the Lagrangian:

\begin{equation}
	\Ell_{H,LRS} = \frac{1}{2}(D_{\mu}\chi_{L})^{\dagger}D^{\mu}\chi_{L} + \frac{1}{2}(D_{\mu}\chi_{R})^{\dagger}D^{\mu}\chi_{R} - V(\chi_{L,R})
\end{equation}

where $D_{\mu} = \partial_{\mu} + ig_{L}\tau^{j}A^{j}_{L\mu} + ig_{R}\tau^{j}\xi^{j}_{R\mu} + i\frac{g'}{2}YB_{\mu}$ contains 
the massless boson fields $A^{j}_{L\mu}, \xi^{j}_{R\mu}, B_{\mu}$ multiplied by the generators of the $SU(2)_{L}, SU(2)_{R}, U(1)$ groups.  
In the models considered here the coupling strengths $g_{R}, g_{L}$ are assumed to be equal, and are denoted as $g$.  The potential $V(\chi_{L,R})$

\begin{equation}
	V(\chi_{L,R}) = -2\lambda_{1}U^{2}_{R}(\chi^{\dagger}_{R}\chi_{R} + \chi^{\dagger}_{L}\chi_{L}) + \lambda_{1}[(\chi^{\dagger}_{R}\chi_{R})^{2} + (\chi^{\dagger}_{L}\chi_{L})^{2}] + \lambda_{2}\chi^{\dagger}_{R}\chi_{R}\chi^{\dagger}_{L}\chi_{L}
\end{equation}

respects the symmetries of LRS models, has positive constants $\lambda_{1}, \lambda_{2} > 2\lambda_{1}$, and is minimized when $\chi_{L,R}$ 
takes values $<\chi^{+}_{L,R}> = 0, <\chi^{0}_{L}> = 0, <\chi^{0}_{R}> = U_{R}$.  In the early universe $V(\chi_{L,R})$ is minimized, and 
subsequently new, massive fields are created.  Identifying the new massive fields as:

\begin{equation}
	W^{\pm}_{R\mu} \equiv \frac{1}{\sqrt{2}}(\xi^{1}_{R\mu} \mp i\xi^{2}_{R\mu}), 
	Z'_{\mu} \equiv \frac{1}{\sqrt{g'^{2} + g^{2}}}(-g'B_{\mu} + g\xi^{3}_{R\mu})
\end{equation}

then substituting $<\chi^{+}_{L,R}> = 0, <\chi^{0}_{L}> = 0, <\chi^{0}_{R}> = U_{R}$ into $\Ell_{H,LRS}$, and simplifying yields the mass terms:

\begin{equation}
	\Ell_{HK,LRS} = (\frac{1}{4}U^{2}_{R}g^{2})W^{+\mu}_{R}W^{-}_{R\mu} + \frac{1}{2}[\frac{1}{4}U^{2}_{R}(g^{2} + g'^{2})]Z'_{\mu}Z'^{\mu}
\end{equation}

Thus in the first stage extension of the BEH mechanism, the bosons $W^{\pm}_{R}, Z'$ are created with masses $m_{W_{R}} = \frac{1}{2}gU_{R}$ 
and $m_{Z'} = \frac{1}{2}U_{R}\sqrt{g'^{2} + g^{2}}$, and all other bosons remain massless.

Following the first stage, the second stage extension of the BEH mechanism \cite{lrsHiggsStageOne,lrsHiggsStageTwo} introduces two complex Higgs doublets 
$\phi_{1}$ and $\phi_{2}$ represented by the multiplet $\Phi$:

\begin{align}
	\Phi &= \begin{bmatrix}
	\phi^{0}_{1} & \phi^{+}_{2} \\
	\phi^{-}_{1} & \phi^{0}_{2}
	\end{bmatrix}
\end{align}

The multiplet interacts with the left and right-handed $SU(2)$ boson fields, and the Lagrangian $\Ell_{H2,LRS}$ that 
describes\footnote{$\Ell_{H2,LRS}$ is similar to $\Ell_{H}$ described in the previous section, but with extra terms for the second Higgs doublet} these 
interactions includes a potential $V(\phi_{1},\phi_{2})$ that forces $\Phi$ to a non-zero expectation value.  When $V(\phi_{1},\phi_{2})$ is 
minimized in the early universe, the multiplet $\Phi$ takes the value:

\begin{align}
	<\Phi> &= \begin{bmatrix}
	\nu_{1} & 0 \\
	0 & \nu_{2}
	\end{bmatrix}
	\label{eq:stageTwoVEV}
\end{align}

After substituting $<\Phi>$ for $\Phi$ in $\Ell_{H2,LRS}$ and simplifying, the \WR, $Z'$, SM $W^{\pm}$ and $Z$ acquire the masses:

\begin{equation}
	m_{W_{L}} = \frac{1}{2}g\nu , m_{W_{R}} \simeq \frac{1}{2}gU_{R} , m_{Z} = \frac{1}{2}\bar{g}\nu , m_{Z'} \simeq \frac{1}{2}\bar{g}U_{R}
\end{equation}
\begin{equation}
	\nu^{2} \equiv \nu^{2}_{1} + \nu^{2}_{2} , \bar{g}^{2} \equiv g^{2} + g'^{2}
\end{equation}

where it is assumed that $U_{R} \gg \nu$, and there is negligible mixing between left and right-handed leptons, 
which is discussed later.  Thus, LRS models predict the existence of the SM weak bosons with correct masses, and three new, 
heavier bosons.  The mass splitting between the left-handed SM $W,Z$ and right-handed $\WR,Z'$ is clear evidence of 
parity violation in LRS models, whereas it is assumed in the SM without a clear theoretical source.

The addition of the $SU(2)_{R}$ group to the SM also introduces three new right-handed neutrinos $N^{l}_{R}$.  
Thanks to these right-handed neutrinos, the LRS models can assign masses to neutrinos using Dirac or Majorana mass terms:

\begin{equation}
	\Ell_{D} = -m\nu^{\dagger}_{L}N_{R} - mN^{\dagger}_{R}\nu_{L}
	\Ell_{M} = -m_{L}\nu^{\dagger}_{L}\nu_{L} - m_{R}\nu^{\dagger}_{R}\nu_{R}
\end{equation}

where $\nu_{R}$ is the anti-particle of the SM neutrino $\nu_{L}$.  LRS models considered here postulate 
that neutrino masses come from a mixture of Dirac and Majorana mass terms \cite{seeSawAndParityViolation,seeSawAndGUTs} described by the Lagrangian:

\begin{align}
	\Ell &= \frac{1}{2}(\bar{\nu}_{Li} \quad \bar{\nu}_{Ri})\begin{bmatrix}
	B'_{i} & M_{i} \\
	M_{i} & B_{i}
\end{bmatrix}(\nu_{Li} \quad \nu_{Ri})^{T}
\label{eq:nuMasses}
\end{align}

where $i$ denotes the lepton generation, and $\nu_{L}$ and $\nu_{R}$ are the pure left and right-handed 
neutrino fermion fields.  The nonzero value of $<\Phi>$ in equation \ref{eq:stageTwoVEV} creates the 
Dirac masses $M_{i}$, and the expectation values of $\chi_{L}$ and $\chi_{R}$ defined in equation \ref{eq:stageOneVEV} 
creates the Majorana masses $B'_{i}$ and $B_{i}$.  As a result, $M_{i} \sim \nu$, $B'_{i} \sim 0$, 
$B_{i} \sim U_{R}$, and $B_{i} \gg M_{i}$, consistent with $m_{W_{R}} \gg m_{W_{L}}$.  Substituting 
these values in for the Dirac and Majorana masses, equation \ref{eq:nuMasses} can be diagonalized to 
yield the mass eigenvalues assuming left-right mixing is negligible.  These mass eigenvalues are:

\begin{equation}
	\lambda_{i+} \simeq B_{i},  \quad \lambda_{i-} \simeq -\frac{M^{2}_{i}}{B_{i}}
\end{equation}

and correspond to one very heavy mass state $\lambda_{i+}$ and a very light state $\lambda_{i-}$.  The 
detectable states $N_{i}, \nu_{i}$ that participate in the weak interactions are described in terms of 
the pure left and right-handed neutrino fields as:

\begin{equation}
	\nu_{i} \simeq \frac{1}{\sqrt{M^{2}_{i} + B^{2}_{i}}}(B_{i}\nu_{Li} - M_{i}\nu_{Ri}) \simeq \nu_{Li} - \frac{M_{i}}{B_{i}}\nu_{Ri} , \quad m_{\nu_{i}} = \lambda_{i-}
	
	N_{i} \simeq \frac{1}{\sqrt{M^{2}_{i} + B^{2}_{i}}}(M_{i}\nu_{Li} + B_{i}\nu_{Ri}) \simeq \nu_{Ri} + \frac{M_{i}}{B_{i}}\nu_{Li} , \quad m_{N_{i}} = \lambda_{i+}
\end{equation}

Thus, in the LRS model, left-handed neutrinos $\nu_{i}$ are very light,  
right-handed neutrinos $N_{i}$ are very heavy, and the left-handed neutrinos become lighter as 
the right-handed neutrinos become heavier.  Furthermore, the model predicts that mixing between left 
and right-handed states is highly suppressed by $\sim \frac{M_{i}}{B_{i}}$, and this is supported by 
experimental evidence \cite{dZeroMixingLimits,theoreticalMixingLimits}.

In addition to explaining the source of parity violation in the SM weak interaction and neutrino 
masses, LRS models can also predict the correct magnitude of CP violation to explain baryon asymmetry.  
Theoretical predictions of baryon asymmetry \cite{saharov}, where there is a deficit of anti-baryons 
relative to baryons, are supported by experimental evidence of CP violation, but the amount of CP 
violation predicted by the SM is insufficient to explain the observed baryon asymmetry \cite{surveyOfExtensions}.  
CP violation in the SM occurs in weak interactions between quarks mediated by the $W^{\pm}$ boson.  
In LRS models the existence of a coupling between the \WR and SM quarks increases the amount of CP 
violation in quark-quark interactions.

\section{LRS Model Phenomenology}
\label{sec:lrsPhenomenology}
The LRS models considered here retain all aspects of the SM that are supported by experimental 
evidence, and provides explanations for light neutrinos, parity violation, and baryon asymmetry.  
In this thesis, the following assumptions are made that significantly affect the experimental 
detection of new particles predicted by LRS models:

\begin{itemize}
	\item The SM quarks and right-handed leptons have the same coupling strengths to the \WR and $Z'$ 
		as the SM quarks and left-handed leptons have to the SM $W$ and $Z$.
	\item $\frac{m_{W_{R}}}{m_{Z'}} \simeq \frac{m_{W}}{m_{Z}}$, so the $W_{R}$ is lighter than the $Z'$.
	\item The right-handed neutrino \nul is lighter than the \WR.
\end{itemize}

Using proton-proton (pp) collisions delivered by the CERN Large Hadron Collider (LHC) and detected 
by the CMS experiment, evidence of LRS models can be found in several ways.  The $W_{R}$ and $Z'$ 
bosons couple to quarks found inside the proton, so they 
can be produced in proton-proton collisions.  As the lighter boson is more likely to be 
produced in LHC collisions, this thesis presents a search for the \WR, which, 
based on previous searches \cite{cmsWRRunOneResults}, is expected to be heavier than $m_{W_{R}} \gtrsim$ 2.8 TeV.

Analogous to the SM $W$ boson decay modes, LRS models predict that the \WR can 
decay to a pair of quarks, or a right-handed charged lepton and heavy \nul.  The $\WR \rightarrow q_{1}q_{2}$ 
process has the highest cross section times branching fraction, but does not allow the mass 
\mnul to be measured, and the \WR signal is obscured by the large rate of high energy SM hadronic 
backgrounds encountered in pp collisions.  Instead, this thesis seeks evidence of LRS models using 
the $\WR \rightarrow l\nul$ decay channel to mitigate hadronic backgrounds, and facilitate 
measurements of \mnul.  In this process, the \nul can decay to a second charged 
lepton and a virtual $W^{*}_{R}$, or to a virtual $Z'^{*}$ and a virtual \nul.  To 
be consistent with experimental observations, it is assumed that the heavy neutrino decay to a charged lepton and 
virtual $W^{*}_{R}$ cannot violate lepton flavor conservation, like $N_{\mu} \rightarrow eW^{*}_{R}$.  
Considering the two \nul decay modes, any \nul decay via $Z'^{*}N^{*}_{l}$ to detectable quarks or 
charged leptons is suppressed by the weak coupling constant squared, $g^{2}$, relative to 
the decay $\nul \rightarrow l^{\pm}W^{*}_{R} \rightarrow l^{\pm}q_{1}q_{2}$.  To maximize 
the probability of finding evidence of LRS models, this thesis presents a search 
for the \WR and \nul in the decay mode $\WR \rightarrow l_{1}\thickspace \nul \rightarrow 
l_{1}\thickspace l_{2}\thickspace q_{1}\thickspace q_{2}$ shown in Figure \ref{fig:wrFeynmanDiagram}.


\begin{figure}[h]
	\centering
	\includegraphics[width=1.0\textwidth]{figures/feynman.pdf}
	\caption{Production of a \WR boson and its decay to two charged leptons and two quarks through 
	a heavy neutrino \nul.}
	\label{fig:wrFeynmanDiagram}
\end{figure}

%%%%%%%%%%%%%%%%%%%%%%%%%%%%%%%%%%%%%%%%%%%%%%%%%%%%%%%%%%%%%%%%%%%%%%%%%%%%%%%%
