%%%%%%%%%%%%%%%%%%%%%%%%%%%%%%%%%%%%%%%%%%%%%%%%%%%%%%%%%%%%%%%%%%%%%%%%%%%%%%%%
%neutrino_physics.tex: Chapter on neutrino physics:
%%%%%%%%%%%%%%%%%%%%%%%%%%%%%%%%%%%%%%%%%%%%%%%%%%%%%%%%%%%%%%%%%%%%%%%%%%%%%%%%
\chapter{Phenomenology}
\label{wrBosonAndHeavyNu}
%%%%%%%%%%%%%%%%%%%%%%%%%%%%%%%%%%%%%%%%%%%%%%%%%%%%%%%%%%%%%%%%%%%%%%%%%%%%%%%%
\section{Particles and Interaction Mediators}
The Standard Model of particle physics (SM) postulates that the universe can be reduced down to a small 
number of particles and interactions between them.  During the past four decades the SM has successfully 
explained many phenomena observed in physics and astronomy, and predicted the existence of many particles 
which were confirmed by experiments.

The theoretical framework of the SM was proposed by Glashow, Weinberg and Salam in 1967 \cite{weinbergSM,salamSM}.  
In their theory, the fundamental building blocks of matter are organized into three generations.  Each 
generation has two quarks, a charged lepton paired with a neutral lepton, and particle and anti-particle 
versions of all four.  The human body and every material or substance that humans are able to sense are 
built from the first generation of particles, which consists of the electron, electron neutrino, and 
the up and down quark.  The second generation contains the muon and muon neutrino, and the charm and 
strange quark, while the third generation contains the tau and tau neutrino, and the top and bottom 
quark.

Interactions in the SM are split into three categories, and are mediated by additional particles.  Mathematically, 
these interactions are represented by three groups, and the group generators correspond to the physical 
particles.  The strong interaction, represented by the $SU(3)_{C}$ group, occurs between quarks and 
is mediated by gluons, which are vector fields multiplied by the $SU(3)$ group generators.  The proton, 
built from three quarks with a net positive charge, is stable because the strong interaction binding 
between quarks overwhelms the electromagnetic repulsion between quarks with same sign electric charge.  
The electromagnetic interaction, represented by the $U(1)$ group, occurs between all charged particles 
and is mediated the photon.  The $\pi^{0}$ contains a linear combination of an up anti-up quark pair and 
a down anti-down quark pair, and primarily decays through the electromagnetic interaction to two photons, 
$\pi^{0} \rightarrow \gamma\gamma$.  The weak  interaction, represented by the $SU(2)_{L}$ group, occurs 
between all particles in all generations, and is mediated by the $W^{\pm}$ and $Z$ bosons.  For reasons 
discussed next, the photon, $W^{\pm}$ and $Z$ bosons are linear combinations of vector fields multiplied 
by the $SU(2)_{L}$ and $U(1)$ group generators.

The SM postulates that the four generators of the $SU(2)_{L} \times U(1)$ become the massless photon and 
massive $W^{\pm}$ and $Z$ bosons through the Brout-Englert-Higgs (BEH) mechanism.  This mechanism 
introduces four degrees of freedom to the SM, in the form of a complex doublet $\Phi$ representing four scalar 
particles, that obey the Lagrangian $\Ell_{H}$:

\begin{align}
	\Phi &= \begin{bmatrix}
	\phi^{+} \\
	\phi^{0}
	\end{bmatrix}
\end{align}

\begin{equation}
	\Ell_{H} = (D_{\mu}\Phi)^{\dagger}D^{\mu}\Phi - V(\Phi)
\end{equation}

where $V(\Phi) = \frac{1}{2}(|\Phi|^{2} - \frac{\nu^{2}}{2})$ is the Higgs potential, and 
$D_{\mu} = \partial_{\mu} + ig_{L}\tau^{j}A^{j}_{\mu} + i\frac{g'}{2}YB_{\mu}$ describes the propagation 
of the Higgs doublet $\Phi$ and its couplings to the $SU(2)_L$ generators $\tau^{j}$ and massless vector 
fields $A^{j}_{\mu}$, and the $U(1)$ generator $Y$ and massless vector field $B_{\mu}$.  $g_{L}$ and 
$g'$ are the weak and electromagnetic interaction coupling strengths.  The Higgs doublet $\Phi$ takes an 
average value $<\Phi> = (0  \nu/sqrt{2})$ so that the Higgs potential is minimized.  When this average 
value is taken, the subset of the Lagrangian $\Ell_{H}$ without partial derivatives $\partial_{\mu}$ and 
the Higgs potential $V(\Phi)$:

\begin{equation}
	\Ell_{HK} = [(ig_{L}\tau^{j}A^{j}_{\mu} + i\frac{g'}{2}YB_{\mu})\Phi]^{\dagger}(ig_{L}\tau^{j}A^{j\mu} + i\frac{g'}{2}YB^{\mu})\Phi
\end{equation}

reduces to:

\begin{equation}
	\Ell_{HK} = \frac{\nu^{2}}{8}[g^{2}_{L}(A^{1}_{\mu} + iA^{2}_{\mu})(A^{1\mu} - iA^{2\mu}) + (g'B_{\mu} - g_{L}A^{3}_{\mu})^{2}]
\end{equation}

Defining the photon vector field $A_{\mu}$, and weak boson vector fields $W^{\pm}_{\mu}$ and $Z_{\mu}$ as:

\begin{equation}
	W^{\pm}_{\mu} \equiv \frac{1}{\sqrt{2}}(A^{1}_{\mu} \pm iA^{2}_{\mu}), 
	Z_{\mu} \equiv \frac{1}{\sqrt{g'^{2} + g^{2}_{L}}}(g'B_{\mu} - g_{L}A^{3}_{\mu}), 
	A_{\mu} \equiv \frac{1}{\sqrt{g'^{2} + g^{2}_{L}}}(g_{L}B_{\mu} + g'A^{3}_{\mu})
\end{equation}

yields the Lagrangian:

\begin{equation}
	\Ell_{HK} = (\frac{\nu g_{L}}{2})^{2}W^{+}_{\mu}W^{-\mu} + \frac{1}{2}(\frac{\nu \bar{g}}{2})^{2}Z_{\mu}Z^{\mu} + 0A_{\mu}A^{\mu}
\end{equation}

where $\bar{g} \equiv \sqrt{g'^{2} + g^{2}_{L}}$.  From this Lagrangian, the photon $A_{\mu}$ is massless, 
the $Z$ boson has mass $m_{Z} = \nu\bar{g}/2$, and the $W^{\pm}$ bosons have mass $m_{W} = \nu g_{L}/2$.

The .







Three of these fields merge with the fields associated with the weak interaction, and thus the 
massive $W^{\pm}$ and $Z$ bosons are created.  The fourth field remains as .

Neutrinos play a special role in the SM.  The SM postulates that they are massless, and thus cannot decay 
to other particles.  In addition, they only interact with matter through the weak interaction.  Neutrino 
experiments.

%%%%%%%%%%%%%%%%%%%%%%%%%%%%%%%%%%%%%%%%%%%%%%%%%%%%%%%%%%%%%%%%%%%%%%%%%%%%%%%%
