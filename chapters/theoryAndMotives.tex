%%%%%%%%%%%%%%%%%%%%%%%%%%%%%%%%%%%%%%%%%%%%%%%%%%%%%%%%%%%%%%%%%%%%%%%%%%%%%%%%
%neutrino_physics.tex: Chapter on neutrino physics:
%%%%%%%%%%%%%%%%%%%%%%%%%%%%%%%%%%%%%%%%%%%%%%%%%%%%%%%%%%%%%%%%%%%%%%%%%%%%%%%%
\chapter{Theory of the Standard Model and Extensions}
\label{wrBosonAndHeavyNu}
%%%%%%%%%%%%%%%%%%%%%%%%%%%%%%%%%%%%%%%%%%%%%%%%%%%%%%%%%%%%%%%%%%%%%%%%%%%%%%%%
\section{Standard Model Particles and Interactions}
The Standard Model of particle physics (SM) postulates that the universe can be reduced down to a small 
number of particles and interactions between them.  During the past four decades the SM has successfully 
explained many phenomena observed in physics and astronomy, and predicted the existence of many particles 
which were later confirmed by experiments.

The theoretical framework of the SM was proposed by Glashow, Weinberg and Salam in 1967 \cite{weinbergSM,salamSM}.  
In their theory, the fundamental building blocks of matter are organized into three generations.  Each 
generation has two quarks, a charged lepton paired with a neutral lepton, and particle and anti-particle 
versions of all four.  The human body and every material or substance that humans are able to sense are 
built from the first generation of particles, which consists of the electron and electron neutrino, and 
the up and down quark.  The second generation contains the muon and muon neutrino, and the charm and 
strange quark, while the third generation contains the tau and tau neutrino, and the top and bottom 
quark.

There are three interactions in the SM, and they are mediated by additional particles.  Mathematically, 
these interactions are represented by gauge groups, and the mediator particles are represented by 
combinations of group generators multiplied by vector fields.  The strong interaction, represented by 
the $SU(3)_{C}$ group, occurs between quarks and is mediated 
by gluons, which are represented by vector fields multiplied by the $SU(3)$ group generators.  The proton, 
built from three quarks with a net positive charge, is stable because the strong interaction attraction 
between quarks overwhelms the electromagnetic repulsion between quarks with same sign electric charges.  
The electromagnetic interaction occurs between all particles with electric charge 
and is mediated the photon.  The $\pi^{0}$ contains a linear combination of an up anti-up quark pair and 
a down anti-down quark pair, and primarily decays through the electromagnetic interaction to two photons, 
$\pi^{0} \rightarrow \gamma\gamma$.  The weak interaction occurs 
between all particles in all generations, and is mediated by the $W^{\pm}$ and $Z$ bosons.  For reasons 
discussed later, the photon, $W^{\pm}$ and $Z$ bosons are represented mathematically by linear combinations 
of vector fields multiplied by the $SU(2)_{L}$ and $U(1)$ group generators.

The dynamics of the strong interaction is driven by a mathematical construct called color charge.  Every 
quark is produced with one of six possible color charges: red, anti-red, blue, anti-blue, green, or anti-
green.  Gluons that mediate the strong interaction carry two different color charges, like red and blue, 
or anti-blue and green.  Color charge is conserved in strong interactions, and all quasi-
stable\footnote{mean lifetime $\gtrsim 10^{-25}$ seconds} quark bound states 
are color neutral.  Thus, a pair of quarks that are red and anti-red, or a triplet of quarks that are 
red, blue and green constitute quasi-stable bound states.  When a proton-proton interaction in the LHC 
produces a quark, the strong interaction works to keep that quark in a color neutral bound state.  As the 
quark moves away from quarks in colliding protons, potential energy accumulates, similar to stretching a 
spring.  Once enough potential energy accumulates, the strong interaction converts the potential energy 
into a massless gluon that decays into two new quarks.  At least one of the two new quarks has 
the correct color to form a color neutral bound state with the original quark.  The other new quark may 
have the correct color to form a color neutral, quark triplet bound state.  If not, the process of 
potential energy accumulation followed by quark pair production continues until the sum total color 
charge of all new quarks produced is neutral.  A gluon produced by a proton-proton interaction 
decays into two quarks of different colors, and the process described for one quark ensues for both 
quarks.  The shower of color neutral hadrons that results from the production of one quark or gluon 
is called a jet for the remainder of this thesis.

The weak interaction has garnered much attention from the particle physics community for many decades.  
To explain the $\beta$-decay process, in 1932 Fermi proposed a model for the weak interaction based on 
the electromagnetic interaction.  His model predicted the existence of a neutral lepton, the electron 
anti-neutrino, which was later supported by experimental evidence \cite{firstNuDiscovery} and became part of the SM.  However, Fermi's 
model did not correctly predict the branching fraction of many hadrons that decay through the weak 
interaction, like $K^{+} \rightarrow 2\pi, 3\pi$.  In 1956 Lee and Yang proposed the weak interaction 
violated parity, and this was substantiated with experimental evidence in the following year \cite{weakParityViolation}.  
The weak interaction in the SM violates parity, and an important consequence of this is the chirality of 
the neutrinos.  Anti-neutrinos are left-handed, and neutrinos are right-handed.  The implications of this 
constraint on neutrino masses will be discussed later.

The SM postulates that the four generators of the $SU(2)_{L} \times U(1)$ groups transform into the massless 
photon and massive $W^{\pm}$ and $Z$ bosons through the Brout-Englert-Higgs (BEH) mechanism.  This mechanism 
adds four degrees of freedom to the SM, in the form of a complex doublet $\Phi$ representing four bosonic 
particles, which obey the Lagrangian $\Ell_{H}$:

\begin{align}
	\Phi &= \begin{bmatrix}
	\phi^{+} \\
	\phi^{0}
	\end{bmatrix}
\end{align}

\begin{equation}
	\Ell_{H} = (D_{\mu}\Phi)^{\dagger}D^{\mu}\Phi - V(\Phi)
\end{equation}

where $V(\Phi) = \frac{1}{2}(|\Phi|^{2} - \frac{\nu^{2}}{2})$ is the Higgs potential, and 
$D_{\mu} = \partial_{\mu} + ig_{L}\tau^{j}A^{j}_{\mu} + i\frac{g'}{2}YB_{\mu}$ describes the propagation 
of the Higgs doublet $\Phi$ and its couplings to the $SU(2)_L$ generators $\tau^{j}$ and massless vector 
fields $A^{j}_{\mu}$, and the $U(1)$ generator $Y$ and massless vector field $B_{\mu}$.  $g_{L}$ and 
$g'$ determine the weak and electromagnetic interaction coupling strengths.  The Higgs doublet $\Phi$ takes a 
value $<\Phi> =$ (0  $\nu/\sqrt{2}$) so that the Higgs potential is minimized.  When $V(\Phi)$ is minimized 
and explicit matrices and values are plugged in for the group generators, the subset 
of the Lagrangian $\Ell_{H}$ without partial derivatives $\partial_{\mu}$ and the Higgs potential $V(\Phi)$:

\begin{equation}
	\Ell_{HK} = [(ig_{L}\tau^{j}A^{j}_{\mu} + i\frac{g'}{2}YB_{\mu})\Phi]^{\dagger}(ig_{L}\tau^{j}A^{j\mu} + i\frac{g'}{2}YB^{\mu})\Phi
\end{equation}

reduces to:

\begin{equation}
	\Ell_{HK} = \frac{\nu^{2}}{8}[g^{2}_{L}(A^{1}_{\mu} + iA^{2}_{\mu})(A^{1\mu} - iA^{2\mu}) + (g'B_{\mu} - g_{L}A^{3}_{\mu})^{2}]
\end{equation}

Defining the photon vector field $A_{\mu}$, and weak boson vector fields $W^{\pm}_{\mu}$ and $Z_{\mu}$ as:

\begin{equation}
	W^{\pm}_{\mu} \equiv \frac{1}{\sqrt{2}}(A^{1}_{\mu} \pm iA^{2}_{\mu}), 
	Z_{\mu} \equiv \frac{1}{\sqrt{g'^{2} + g^{2}_{L}}}(g'B_{\mu} - g_{L}A^{3}_{\mu}), 
	A_{\mu} \equiv \frac{1}{\sqrt{g'^{2} + g^{2}_{L}}}(g_{L}B_{\mu} + g'A^{3}_{\mu})
\end{equation}

yields the Lagrangian:

\begin{equation}
	\Ell_{HK} = (\frac{\nu g_{L}}{2})^{2}W^{+}_{\mu}W^{-\mu} + \frac{1}{2}(\frac{\nu \bar{g}}{2})^{2}Z_{\mu}Z^{\mu} + 0A_{\mu}A^{\mu}
\end{equation}

where $\bar{g} \equiv \sqrt{g'^{2} + g^{2}_{L}}$.  From this Lagrangian, the photon $A_{\mu}$ is massless, 
the $Z$ boson has mass $m_{Z} = \nu\bar{g}/2$, and the $W^{\pm}$ bosons have mass $m_{W} = \nu g_{L}/2$.  
Three of the four scalar fields introduced by the BEH mechanism are consumed to give mass to the $Z$ 
and $W^{\pm}$ bosons.  Recent experimental evidence of the fourth scalar field \cite{combinedHiggsResult}, the Higgs boson, 
supports the SM prediction that the $Z$ and $W^{\pm}$ bosons acquire mass through the BEH mechanism.

Experimental evidence supports the prediction that quarks and charged leptons are massive particles.  In the SM, they can acquire mass 
through two methods.  Their masses can be added directly to the SM Lagrangian, or additional Higgs fields 
can be added to the BEH mechanism.  In either case, the result is mass terms of the form $-mf\bar{f}$, where $f$ is a fermion 
representing a quark or charged lepton, are added to the SM Lagrangian.  In the basis where a 
fermion field $f$ consists of a right-handed component $\chi_{R}$ and left-handed 
component $\chi_{L}$, $f = (\chi_{L},\chi_{R})$, a fermion mass term in the SM Lagrangian is written as:

\begin{equation}
	\Ell_{D} = -m\bar{f}f = -m\chi^{\dagger}_{L}\chi_{R} - m\chi^{\dagger}_{R}\chi_{L}
\end{equation}

This type of mass term, called a Dirac mass, contains the product of left and right-handed fields.  Quarks 
and charged leptons can exist as left or right-handed fields, so their masses can be assigned using Dirac 
mass terms.

Neutrinos play a special role in the SM.  The SM postulates that they are neutral, massless fermions, and only interact 
with other particles through the weak interaction.  In addition, due to parity violation of the weak interaction, 
anti-neutrinos $\bar{\nu_{l}}$ are always right-handed, and neutrinos $\nu_{l}$ are always left-handed.  Results 
from neutrino experiments \cite{NOvAresults,mainzPhaseIIResults,t2kResults} support the proposition that neutrinos 
are massive fermions.  In the SM, fermions can only acquire mass through a Lagrangian of the form $\Ell_{D}$, 
which requires a fermion to have left and right-handed fields.  Parity violation in the weak interaction prevents 
neutrinos from having left and right-handed fields, so an extension of the SM is needed to explain the existence 
of neutrinos with mass.


\section{Left-Right Symmetric Extensions of the Standard Model}
Parity violation in the weak interaction constrains SM neutrinos to be massless, but this constraint is removed 
in SM extensions where parity is conserved.  First proposed in 1974 \cite{earlyLRSModel}, Left-Right Symmetric (LRS) 
extensions of the SM postulate that all interactions conserved parity in the very early universe.  In LRS models, shortly after 
the Big Bang a BEH mechanism brought several massive scalar particles into existence.  These scalar particles merged 
with massless mediator particles of an interaction, and created a parity violating weak interaction mediated by 
massive particles.  Many LRS models predict the early universe evolved in this manner.  This thesis focuses on the 
LRS extension model that replaces the SM $SU(2)_{L} \times U(1)$ groups with $SU(2)_{R} \times SU(2)_{L} \times U(1)$, 
and extends the SM BEH mechanism to give mass to neutrinos and to create additional, massive bosons.

Adding the $SU(2)_{R}$ group to the SM introduces three new, massless vector fields $\xi^{j}_{\mu}$.  The SM BEH mechanism 
is extended in two stages, which result in six massive gauge bosons that mediate the weak interaction.  In the first 
stage \cite{lrsHiggsStageOne}, a chiral, complex Higgs doublet $\chi_{L,R}$ is introduced 

\begin{align}
	\chi_{L,R} &= \begin{bmatrix}
	\chi^{+}_{L,R} \\
	\chi^{0}_{L,R}
	\end{bmatrix}
	\label{eq:stageOneVEV}
\end{align}

with bosonic fields that couple independently to left and right-handed gauge bosons.  Propagation and interaction of these 
fields with other massless bosons is described by the Lagrangian:

\begin{equation}
	\Ell_{H,LRS} = \frac{1}{2}(D_{\mu}\chi_{L})^{\dagger}D^{\mu}\chi_{L} + \frac{1}{2}(D_{\mu}\chi_{R})^{\dagger}D^{\mu}\chi_{R} - V(\chi_{L,R})
\end{equation}

where $D_{\mu} = \partial_{\mu} + ig_{L}\tau^{j}A^{j}_{L\mu} + ig_{R}\tau^{j}\xi^{j}_{R\mu} + i\frac{g'}{2}YB_{\mu}$ contains 
the massless boson fields $A^{j}_{L\mu}, \xi^{j}_{R\mu}, B_{\mu}$ multiplied by the generators of the $SU(2)_{L}, SU(2)_{R}, U(1)$ groups.  
In the model considered here the coupling strengths $g_{R}, g_{L}$ are assumed to be equal, and are denoted as $g$.  The potential $V(\chi_{L,R})$

\begin{equation}
	V(\chi_{L,R}) = -2\lambda_{1}U^{2}_{R}(\chi^{\dagger}_{R}\chi_{R} + \chi^{\dagger}_{L}\chi_{L}) + \lambda_{1}[(\chi^{\dagger}_{R}\chi_{R})^{2} + (\chi^{\dagger}_{L}\chi_{L})^{2}] + \lambda_{2}\chi^{\dagger}_{R}\chi_{R}\chi^{\dagger}_{L}\chi_{L}
\end{equation}

respects the symmetries of the LRS model with positive constants $\lambda_{1}, \lambda_{2} > 2\lambda_{1}$, and is minimized when $\chi_{L,R}$ 
takes values $<\chi^{+}_{L,R}> = 0, <\chi^{0}_{L}> = 0, <\chi^{0}_{R}> = U_{R}$.  In the early universe $V(\chi_{L,R})$ is minimized, and 
subsequently new, massive fields are created.  Identifying the new massive fields as:

\begin{equation}
	W^{\pm}_{R\mu} \equiv \frac{1}{\sqrt{2}}(\xi^{1}_{R\mu} \mp i\xi^{2}_{R\mu}), 
	Z'_{\mu} \equiv \frac{1}{\sqrt{g'^{2} + g^{2}}}(-g'B_{\mu} + g\xi^{3}_{R\mu})
\end{equation}

then substituting $<\chi^{+}_{L,R}> = 0, <\chi^{0}_{L}> = 0, <\chi^{0}_{R}> = U_{R}$ into $\Ell_{H,LRS}$, dropping the the potential $V(\chi_{L,R})$ 
and $\partial_{\mu}$ terms, and multiplying the generator matrices $\tau^{j}$ by the fields $\xi^{j}_{R\mu},\A^{j}_{L\mu}$ yields:

\begin{equation}
	\Ell_{HK,LRS} = (\frac{1}{4}U^{2}_{R}g^{2})W^{+\mu}_{R}W^{-}_{R\mu} + \frac{1}{2}[\frac{1}{4}U^{2}_{R}(g^{2} + g'^{2})]Z'_{\mu}Z'^{\mu}
\end{equation}

Thus in the first stage extension of the BEH mechanism, the bosons $W^{\pm}_{R}, Z'$ are created with masses $m_{W_{R}} = \frac{1}{2}gU_{R}$ 
and $m_{Z'} = \frac{1}{2}U_{R}\sqrt{g'^{2} + g^{2}}$, and all other bosons remain massless.

Following the first stage, the second stage extension of the SM BEH mechanism \cite{lrsHiggsStageOne,lrsHiggsStageTwo} introduces two complex Higgs doublets 
$\phi_{1}$ and $\phi_{2}$ represented by the multiplet $\Phi$:

\begin{align}
	\Phi &= \begin{bmatrix}
	\phi^{0}_{1} & \phi^{+}_{2} \\
	\phi^{-}_{1} & \phi^{0}_{2}
	\end{bmatrix}
\end{align}

The multiplet interacts with the left and right-handed $SU(2)$ boson fields, and the Lagrangian $\Ell_{H2,LRS}$ that 
describes\footnote{$\Ell_{H2,LRS}$ is similar to $\Ell_{H}$ described in the previous section, but with extra terms for the second Higgs doublet} these 
interactions includes a potential $V(\phi_{1},\phi_{2})$.  In the early universe this potential is minimized, and 
the multiplet $\Phi$ takes the value:

\begin{align}
	<\Phi> &= \begin{bmatrix}
	\nu_{1} & 0 \\
	0 & \nu_{2}
	\end{bmatrix}
	\label{eq:stageTwoVEV}
\end{align}

Following the procedure used for the first stage extension of the SM BEH mechanism, $<\Phi>$ is substituted for $\Phi$ in 
the Lagrangian $\Ell_{H2,LRS}$, and algebraic simplifications are applied.  The second stage extension of the SM BEH 
mechanism yields the SM weak bosons $W^{\pm}_{L}$ and $Z$, and the new $W^{\pm}_{R}$ and $Z'$ bosons with masses:

\begin{equation}
	m_{W_{L}} = \frac{1}{2}g\nu , m_{W_{R}} \simeq \frac{1}{2}gU_{R} , m_{Z} = \frac{1}{2}\bar{g}\nu , m_{Z'} \simeq \frac{1}{2}\bar{g}U_{R}
\end{equation}
\begin{equation}
	\nu^{2} \equiv \nu^{2}_{1} + \nu^{2}_{2} , \bar{g}^{2} \equiv g^{2} + g'^{2}
\end{equation}



where it is assumed that $U_{R} \gg \nu$, and there is negligible mixing between left and right-handed leptons, 
which is discussed later.  Thus LRS model yields the SM weak bosons with the correct masses, and three new, 
heavier bosons.  The parity violation of the SM weak interaction is retained in the LRS model.

The addition of the $SU(2)_{R}$ group to the SM also introduces three new generations of right-handed leptons - 
three charged leptons $l^{\pm}_{R}$, and three neutral leptons (neutrinos) $N^{l}_{R}$.  The existence of 
right-handed neutrinos means the LRS model Lagrangian can assign neutrino masses using Dirac mass terms like:

\begin{equation}
	\Ell_{D} = -m\nu^{\dagger}_{L}N_{R} - mN^{\dagger}_{R}\nu_{L}
\end{equation}

in addition to using Majorana neutrino mass terms that do not require independent left and right-handed 
neutrinos, like:

\begin{equation}
	\Ell_{M} = -m_{L}\nu^{\dagger}_{L}\nu_{L} - m_{R}\nu^{\dagger}_{R}\nu_{R}
\end{equation}

where $\nu_{R}$ is the anti-particle of the SM neutrino $\nu_{L}$.  The LRS model considered here postulates 
that neutrino masses come from a mixture of Dirac and Majorana mass terms \cite{seeSawAndParityViolation,seeSawAndGUTs} described by the Lagrangian:

\begin{align}
	\Ell &= \frac{1}{2}(\bar{\nu}_{Li} \quad \bar{\nu}_{Ri})\begin{bmatrix}
	B'_{i} & M_{i} \\
	M_{i} & B_{i}
\end{bmatrix}(\nu_{Li} \quad \nu_{Ri})^{T}
\label{eq:nuMasses}
\end{align}

where $i$ denotes the lepton generation, and $\nu_{L}$ and $\nu_{R}$ are the pure left and right-handed 
neutrino fermion fields.  The nonzero value of $<\Phi>$ in equation \ref{eq:stageTwoVEV} creates the 
Dirac masses $M_{i}$, and the expectation values of $\chi_{L}$ and $\chi_{R}$ defined in equation \ref{eq:stageOneVEV} 
creates the Majorana masses $B'_{i}$ and $B_{i}$.  As a result, $M_{i} \sim \nu$, $B'_{i} \sim 0$, 
$B_{i} \sim U_{R}$, and $B_{i} \gg M_{i}$, consistent with $m_{W_{R}} \gg m_{W_{L}}$.  Substituting 
these values in for the Dirac and Majorana masses, equation \ref{eq:nuMasses} can be diagonalized to 
yield the mass eigenvalues assuming left-right mixing is negligible.  These mass eigenvalues are:

\begin{equation}
	\lambda_{i+} \simeq B_{i},  \quad \lambda_{i-} \simeq -\frac{M^{2}_{i}}{B_{i}}
\end{equation}

which yields one very heavy mass state $\lambda_{i+}$ and a very light state $\lambda_{i-}$.  The 
detectable states $N_{i}, \nu_{i}$ that participate in the weak interactions can be described in terms of 
the pure left and right-handed neutrino fields as:

\begin{equation}
	\nu_{i} \simeq \frac{1}{\sqrt{M^{2}_{i} + B^{2}_{i}}}(B_{i}\nu_{Li} - M_{i}\nu_{Ri}) \simeq \nu_{Li} - \frac{M_{i}}{B_{i}}\nu_{Ri}
\end{equation}

with mass $\lambda_{i-}$, and:

\begin{equation}
	N_{i} \simeq \frac{1}{\sqrt{M^{2}_{i} + B^{2}_{i}}}(M_{i}\nu_{Li} + B_{i}\nu_{Ri}) \simeq \nu_{Ri} + \frac{M_{i}}{B_{i}}\nu_{Li}
\end{equation}

with mass $\lambda_{i+}$.  Thus, in the LRS model, left-handed neutrinos $\nu_{i}$ are very light, and 
right-handed neutrinos $N_{i}$ are very heavy.  Furthermore, the model predicts that mixing between left 
and right-handed states is highly suppressed by $\sim \frac{M_{i}}{B_{i}}$, and this is supported by 
experimental evidence \cite{dZeroMixingLimits,theoreticalMixingLimits}.  Experimental evidence of this 
LRS model can explain why neutrinos have mass.


\section{Phenomenology of LRS Model}
Using proton-proton collisions delivered by the CERN Large Hadron Collider and the CMS experiment, 
evidence of the LRS model under consideration can be found in several ways.  The $W_{R}$ and $Z'$ 
bosons originating from the $SU(2)_{R}$ group couple to quarks found inside the proton, so they 
can be produced in proton-proton collisions.  In this thesis it is assumed that the coupling 
strengths between the $W_{R}$ and $Z'$ bosons and SM quarks and right-handed leptons are equal 
to the coupling strengths between the SM $W$ and $Z$ bosons and the SM quarks and left-handed 
leptons.  In addition, it is assumed that the ratio of masses $\frac{m_{W_{R}}}{m_{Z'}} \simeq \frac{m_{W}}{m_{Z}}$, 
and thus the $Z'$ is heavier than the $W_{R}$.  The $Z'$ mass may exceed the 13 TeV collision 
energy achieved .




%%%%%%%%%%%%%%%%%%%%%%%%%%%%%%%%%%%%%%%%%%%%%%%%%%%%%%%%%%%%%%%%%%%%%%%%%%%%%%%%
