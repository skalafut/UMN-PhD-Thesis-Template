%%%%%%%%%%%%%%%%%%%%%%%%%%%%%%%%%%%%%%%%%%%%%%%%%%%%%%%%%%%%%%%%%%%%%%%%%%%%%%%%
% app_wrMC.tex: Appendix on WR MC generation:
%%%%%%%%%%%%%%%%%%%%%%%%%%%%%%%%%%%%%%%%%%%%%%%%%%%%%%%%%%%%%%%%%%%%%%%%%%%%%%%%
\chapter{Bayesian Limits}
\label{app_bayesLimits}
%%%%%%%%%%%%%%%%%%%%%%%%%%%%%%%%%%%%%%%%%%%%%%%%%%%%%%%%%%%%%%%%%%%%%%%%%%%%%%%%
The \WR cross section $\times$ branching ratio ($\sigma(\WR) \times BR(\WR \rightarrow \ell\ell jj)$) limits are calculated 
using several quantities.  Each limit for a specific \mWR signal is calculated using the number of measured events $G$ and 
predicted signal ($S$) and background ($B$) events in the $\Mlljj$ window, and their uncertainties $\delta S$ and $\delta B$.  
Expected limits, calculated by setting $G = B$, were used to determine the sizes of the $\Mlljj$ windows.  Expected limits were 
calculated at 95\% confidence level (CL) using a Poisson model of the $S \plus B$ events:

\begin{equation}
	Poisson(\mu S(\pmb{\theta}) \thickspace \plus \thickspace B(\pmb{\theta}))
	\label{eq:poissonModel}
\end{equation}
where $\mu$ is the dimensionless \WR signal strength, and $\pmb{\theta}$ represent the uncertainties $\delta S$ and $\delta B$ 
described later in Section \ref{sec:uncertainties}.  Using Bayesian statistics, the probability distribution for $\mu$ given 
$G$ measured events, $p(\mu|G)$, is obtained by evaluating the integral \cite{bayesianDataAnalysis}:

\begin{equation}
	p(\mu|G) = \int p(\mu|\pmb{\theta},G)p(\pmb{\theta}|G)d\pmb{\theta}
	\label{eq:sigStrngthProbDist}
\end{equation}
where $p(\pmb{\theta}|G)$ are the probability distributions for the uncertainties given the measurement $G$ (``marginal 
posterior distributions''), and $p(\mu|\pmb{\theta},G)$ are the probability distributions for $\mu$ given the uncertainties, 
and the measurement $G$ (``conditional posterior distributions'').  Functional forms of the marginal posterior distributions 
are either log-normal or Gamma distributions depending on the uncertainty, and are identified later for specific uncertainties.  
Functional forms of the conditional posterior distributions are derived from uniform prior distributions and Equation 
\ref{eq:poissonModel}.  The integrals in Equation \ref{eq:sigStrngthProbDist} were evaluated numerically using \MC methods 
to derive $p(\mu|G)$.  Then, $p(\mu|G)$ was integrated from $\mu =$ 0 to $\mu = \mu_{max}$ such that the normalized integral 
equaled 0.95.  The value $\mu_{max}$ is the 95\% CL upper limit on the signal strength.  The value of $\sigma(\WR) \times 
BR(\WR \rightarrow \ell\ell jj)$ obtained from simulations is multiplied by $\mu_{max}$ to calculate the upper limit on 
$\sigma(\WR) \times BR(\WR \rightarrow \ell\ell jj)$.


%%%%%%%%%%%%%%%%%%%%%%%%%%%%%%%%%%%%%%%%%%%%%%%%%%%%%%%%%%%%%%%%%%%%%%%%%%%%%}}}
