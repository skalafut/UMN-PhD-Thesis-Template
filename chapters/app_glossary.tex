%%%%%%%%%%%%%%%%%%%%%%%%%%%%%%%%%%%%%%%%%%%%%%%%%%%%%%%%%%%%%%%%%%%%%%%%%%%%%%%%
% app_glossary.tex: Glossary Appendix:
%%%%%%%%%%%%%%%%%%%%%%%%%%%%%%%%%%%%%%%%%%%%%%%%%%%%%%%%%%%%%%%%%%%%%%%%%%%%%%%%
\chapter{Glossary and Acronyms}
\label{app_glossary}
%%%%%%%%%%%%%%%%%%%%%%%%%%%%%%%%%%%%%%%%%%%%%%%%%%%%%%%%%%%%%%%%%%%%%%%%%%%%%%%%
Care has been taken in this thesis to minimize the use of jargon and
acronyms, but this cannot always be achieved.  This appendix defines
jargon terms in a glossary, and contains a table of acronyms and their
meaning.
%%%%%%%%%%%%%%%%%%%%%%%%%%%%%%%%%%%%%%%%%%%%%%%%%%%%%%%%%%%%%%%%%%%%%%%%%%%%%%%%

%%%%%%%%%%%%%%%%%%%%%%%%%%%%%%%%%%%%%%%%%%%%%%%%%%%%%%%%%%%%%%%%%%%%%%%%%%%%%%%%
% Glossary {{{
%%%%%%%%%%%%%%%%%%%%%%%%%%%%%%%%%%%%%%%%%%%%%%%%%%%%%%%%%%%%%%%%%%%%%%%%%%%%%%%%
\section{Glossary}
\label{jargonapp}
%%%%%%%%%%%%%%%%%%%%%%%%%%%%%%%%%%%%%%%%%%%%%%%%%%%%%%%%%%%%%%%%%%%%%%%%%%%%%%%%
\begin{itemize}

%\item \textbf{Cosmic-Ray Muon} (\textbf{CR $\mu$}) -- A muon coming from
%the abundant energetic particles originating outside of the Earth's
%atmosphere.
	\item \textbf{barrel} -- The central $\eta$ region of each sub-detector is 
		called the barrel.
	\item \textbf{endcap} -- The high $\eta$ region of each sub-detector is 
		called the endcap.
	\item \textbf{hadronization} -- The process through which bare quarks and 
		gluons become color neutral hadrons.
	\item \textbf{interaction point} -- The region at the center of the detector 
		where proton-proton interactions occur.
	\item \textbf{minimum bias collisions} -- A collection of randomly selected 
		proton-proton collision events where energy is detected in CMS.  Due to 
		its high cross section, inelastic and diffractive proton-proton collisions 
		constitute the majority of minimum bias collisions.
	\item \textbf{nuclear interaction length} -- On average a relativistic hadron 
		will experience one nuclear interaction after travelling through one 
		nuclear interaction length of material.
	\item \textbf{pileup} -- Additional proton-proton interactions that occur 
		in the same collision event.
	\item \textbf{reconstruction} -- The process through which detector information 
		is transformed into particles used in physics analyses.
	\item \textbf{Supercluster} (\textbf{SC}) -- A 5 $\times$ 5 crystal region 
		in the Electromagnetic Calorimeter.
	\item \textbf{Supercluster seed crystal} (\textbf{SC seed}) -- The highest 
		energy crystal in a Supercluster.
\end{itemize}
%%%%%%%%%%%%%%%%%%%%%%%%%%%%%%%%%%%%%%%%%%%%%%%%%%%%%%%%%%%%%%%%%%%%%%%%%%%%%}}}

%%%%%%%%%%%%%%%%%%%%%%%%%%%%%%%%%%%%%%%%%%%%%%%%%%%%%%%%%%%%%%%%%%%%%%%%%%%%%%%%
% Acronyms {{{
%%%%%%%%%%%%%%%%%%%%%%%%%%%%%%%%%%%%%%%%%%%%%%%%%%%%%%%%%%%%%%%%%%%%%%%%%%%%%%%%
\section{Acronyms}
\label{acronymsec}
%%%%%%%%%%%%%%%%%%%%%%%%%%%%%%%%%%%%%%%%%%%%%%%%%%%%%%%%%%%%%%%%%%%%%%%%%%%%%%%%

% Table formatting

% Heading for the first page
\begin{longtable}{p{0.25\textwidth} p{0.75\textwidth}}
\caption{Acronyms} \label{tab:acronyms} \\

\toprule
Acronym & Meaning \\
\midrule
\endfirsthead

% Heading for all subsequent pages
\multicolumn{2}{l}{\textit{\tablename\ \thetable{} -- Continued from previous page}} \\
\toprule
Acronym & Meaning \\
\midrule
\endhead

% Footer for each page that wraps over to the next
\multicolumn{2}{r}{\textit{Continued on next page}} \\
\bottomrule
\endfoot

% Footer for the end of the table
\bottomrule
\endlastfoot

% End table formatting

CR$\mu$ & Cosmic-Ray Muon \\

\end{longtable}
%%%%%%%%%%%%%%%%%%%%%%%%%%%%%%%%%%%%%%%%%%%%%%%%%%%%%%%%%%%%%%%%%%%%%%%%%%%%%}}}
