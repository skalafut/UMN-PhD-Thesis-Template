%%%%%%%%%%%%%%%%%%%%%%%%%%%%%%%%%%%%%%%%%%%%%%%%%%%%%%%%%%%%%%%%%%%%%%%%%%%%%%%%
% app_wrMC.tex: Appendix on WR MC generation:
%%%%%%%%%%%%%%%%%%%%%%%%%%%%%%%%%%%%%%%%%%%%%%%%%%%%%%%%%%%%%%%%%%%%%%%%%%%%%%%%
\chapter{Trigger and Offline Selection Criteria}
\label{app_trgOfflId}
%%%%%%%%%%%%%%%%%%%%%%%%%%%%%%%%%%%%%%%%%%%%%%%%%%%%%%%%%%%%%%%%%%%%%%%%%%%%%%%%
During collisions, in the high-level trigger the electron, muon, and HCAL cluster reconstruction algorithms are used in the electron and 
muon HLT algorithms to select events with at least one lepton that is isolated from large HCAL energies.  The triggers used for the $\mu\mu$-
channel and the $ee$-channel were chosen to maximize the \WR selection efficiency, and select a large fraction of $Z \rightarrow \ell\ell$ 
events to minimize the statistical uncertainty on the \DY background prediction discussed later.

Single muon and double-muon triggers select $\WR \rightarrow \mu\mu jj$ events with different efficiencies.  The single muon 
triggers require one track segment to be reconstructed in multiple muon detectors and have $\pt$ above some threshold $A$, while the 
double-muon triggers require two track segments that have $\pt$ above some threshold $B < A$.  In simulated $\WR \rightarrow \mu\mu jj$ 
events the highest efficiency single muon trigger selected 5\% more \WR events than the highest efficiency double muon trigger (Table 
\ref{tab:singleVsDblMuHlt}).  The double-muon trigger selected fewer \WR events because the efficiency to reconstruct a muon with 
$\pt > B$ as a track segment was below 100\% \cite{cmsMuonRecoRunTwo}.  To maximize the signal selection efficiency, a single muon 
trigger was used to select collision events with muons.

\begin{table}[h]
	\caption{The efficiency of a single and double-muon trigger in simulated $\WR \rightarrow \mu\mu jj$ events with $\mWR = 800$ $\GeV$ 
		and $\mnul = 400$ $\GeV$.}
	\label{tab:singleVsDblMuHlt}
	\centering
	\begin{tabular}{c|c|c}
		trigger type & $\pt$ criteria & efficiency (\%) \\  \hline
		single $\mu$ & $\pt >50$ $\GeV$ & 98  \\ 
		double-$\mu$ & one $\pt >17$, other $\pt >8$ $\GeV$ & 93  \\
	\end{tabular}
\end{table}

During collisions, events with muons were selected using a Level-1 trigger that required one track segment with $\pt > 16$ $\GeV$.  The 
segment was required to have signals in DT or CSC chambers from at least 2 stations, and signals in at least 4 layers of each chamber.  
Then, events were required to pass the following single muon HLT selection criteria:

\begin{itemize}
	\item A track was reconstructed in the silicon tracker with $\pt > 50$ $\GeV$ and $|\eta| < 2.4$.
	\item In the plane perpendicular to the beam axis, the distance between the silicon tracker track origin and its 
		reconstructed vertex was $< 1$ mm.
	\item The muon detector track that passed the L1 trigger extrapolated back to the silicon tracker track $(\eta,\phi)$ 
		position to within 3 cm.
\end{itemize}

Muons reconstructed in simulated events passed the trigger criteria with a different efficiency than muons reconstructed in data.  This 
efficiency difference was corrected by multiplying the weight of every simulated event (default weight is 1.0) by a $\pt$,$\eta$-
dependent value between 0.95 (5\% decrease) and 1.04 (4\% increase) for the muon that fired the trigger.

In events selected by the single muon trigger, tracks in the muon detectors and silicon tracker were identified as being caused by muons 
using algorithms described previously.  Then, the following identification criteria were applied to select promptly produced muons 
that were isolated from other particles and reconstructed in multiple muon stations:

\begin{itemize}
	\item The muon track reconstructed in the silicon tracker:
	\begin{itemize}
		\item Was reconstructed from signals in at least 1 silicon pixel detector layer, and signals in at least 
			5 layers in the entire tracker.
		\item Within a cone of radius $\Delta R = 0.3$ centered on the track, the $\sum \pt$ of all other 
			reconstructed tracks was low compared to the muon $\pt$, $\frac{\sum \pt}{muon \pt} < 0.1$.
	\end{itemize}
	\item The muon's track segment went through a muon chamber in at least 2 muon stations.  Track segments in each DT 
		chamber were required to have signals in all 4 r-$z$ layers, and at least 7 of 8 r-$\phi$ layers.  Track segments 
		in each CSC were required to have signals in all 6 layers.
	\item The origin of the muon's silicon tracker track was within 2 mm of the muon's reconstructed vertex 
		position along the $z$ axis.
\end{itemize}

Simulated muons are reconstructed and pass the offline muon identification selection criteria with a higher 
efficiency than muons produced in real collisions.  This efficiency difference was corrected by multiplying the weight of every 
simulated event by a $\pt$,$\eta$-dependent value between 0.985 and 1.0 for each offline muon selected in the event.


Similar to the muon triggers, single electron and double-electron triggers select $\WR \rightarrow eejj$ events with different efficiencies.  
The single electron triggers require that one track reconstructed in the silicon tracker extrapolates to a 5 $\times$ 5 ECAL crystal 
cluster with $\Et$ above some threshold $D$, while the double-electron triggers require two such track-ECAL cluster combinations each 
with $\Et$ above some threshold $G < D$.  In simulated $\WR \rightarrow eejj$ events the highest efficiency single and double-electron 
triggers selected similar numbers of signal events (Table \ref{tab:singleVsDblEleHlt}), so a trigger was chosen using different criteria.  To 
reduce the statistical uncertainty on the \DY background prediction, the electron trigger should select a large fraction of $Z \rightarrow ee$ 
events.  The single electron trigger required $\Et > 105$ $\GeV$, thereby excluding a significant fraction of $Z \rightarrow ee$ events.  To 
select a larger portion of $Z \rightarrow ee$ events, a double-electron trigger that required two electrons that had $\Et > 33$ $\GeV$ 
was used to select collision events with electrons.

\begin{table}[h]
	\caption{The efficiency of a single and double-electron trigger in simulated $\WR \rightarrow \mu\mu jj$ events with $\mWR = 800$ $\GeV$ 
		and $\mnul = 400$ $\GeV$.}
	\label{tab:singleVsDblEleHlt}
	\centering
	\begin{tabular}{c|c|c}
		trigger type & $\Et$ criteria & efficiency (\%) \\  \hline
		single $e$ & $\Et >105$ $\GeV$ & 94  \\ 
		double-$e$ & both $\Et >33$ $\GeV$ & 92  \\
	\end{tabular}
\end{table}

During collisions, events with electrons were selected using single and double-electron Level-1 triggers.  These triggers required 
one 5 $\times$ 5 ECAL crystal cluster that had $\Et > 40$ $\GeV$, or two 5 $\times$ 5 ECAL clusters that had 
$\Et > 22$ $\GeV$ and $\Et > 10$ $\GeV$.  Then, events were required to pass the following double-electron HLT selection criteria:

\begin{itemize}
	\item Two 5 $\times$ 5 ECAL crystal clusters separated by $\Delta R > 0.1$ were required to have $\Et > 33$ $\GeV$.
	\item For each ECAL cluster (energy E):
	\begin{itemize}
		\item The hadronic energy behind the cluster was $<$ 15\% of E in the barrel, and $<$ 10\% of E in the endcap. 
		\item Ninety percent of E was measured in an area that was two crystals wide in $\eta$.
		\item If the cluster was in the barrel, a reconstructed track with signals in at least two pixel tracker layers 
			extrapolated close to the cluster position.  The track extrapolated from the pixel tracker to within $2.3$ cm 
			of the cluster $z$ position, and to within 1 ECAL crystal area of the cluster $(\eta,\phi)$ position.
	\end{itemize}
\end{itemize}

Electrons reconstructed in simulations passed the trigger criteria with the same efficiency as electrons that were reconstructed in data, 
so no electron trigger efficiency correction was applied to simulated events.

In events selected by the double-electron trigger, reconstructed tracks and ECAL SCs were identified as being caused by electrons using 
algorithms described previously.  Then, the following identification criteria were applied to select the promptly produced electrons, 
excluding electrons within a jet, or in the ECAL transition region $1.44 < |\eta| < 1.57$:

\begin{itemize}
	\item The electron $\Et$ is the calibrated ECAL SC energy $E_{SC}$.
	\item For a SC in the barrel, at least 94\% of $E_{SC}$ was measured in an area that was 2 crystals wide in $\eta$.
	\item The hadronic energy (H) behind the SC was $\frac{H}{E_{SC}}< 0.05 +\thickspace \frac{1 \GeV}{E_{SC}}$ 
		in the barrel, and $\frac{H}{E_{SC}}< 0.05 +\thickspace \frac{5 \GeV}{E_{SC}}$ in the endcap.
	\item In a $\Delta R =$ 0.3 radius cone centered on the electron's $(\eta, \phi)$ trajectory:
	\begin{itemize}
		\item The $\sum \pt$ of all tracks excluding the electron's track was low, $\sum \pt < 5$ $\GeV$.
		\item The total calorimeter energy $E_{ECAL + HCAL}$ not associated with the electron was 
			$E_{ECAL + HCAL} < 2 + 0.03\alpha + 0.28\rho$.  $\rho$ is the neutral particle energy per unit $\eta,\phi$ area, 
			$\alpha$ in the barrel is $E_{SC}$, and $\alpha$ in the endcap is $E_{SC} - 50$.
	\end{itemize}
	\item For a SC in the endcap, the electron track extrapolated from the outermost silicon tracker measurement to the SC 
		seed crystal position to within $\sim$3 crystal widths in $\phi$.
	\item The electron track was reconstructed from signals in every silicon pixel and inner strip detector layers, or all but 1 layer.
	\item The electron track's origin was separated from its vertex by a small distance $\Delta_{xy}$ in the $x-y$ 
		plane: $\Delta_{xy} < 0.2$ mm in the tracker barrel, and $\Delta_{xy} < 0.5$ mm in the tracker endcap.
\end{itemize}

Simulated electrons are reconstructed and pass the offline electron identification selection criteria with a higher efficiency than 
electrons that were reconstructed in data.  This efficiency difference was corrected by multiplying the weight of every 
simulated event by 0.982, the reconstruction efficiency correction, and by 0.989, the ID efficiency correction, for each offline electron 
selected in the event.

In events selected by the lepton triggers, jets were reconstructed offline from tracks and calorimeter energy clusters using the 
algorithms described previously.  Then, the following identification criteria were applied to select jets that contained at least one 
charged hadron, and whose energies were not dominated by electron or photon SCs:

\begin{itemize}
	\item The jet had at least 2 constituents.
	\item The jet had at least 1 charged hadron constituent.
	\item More than 0\% of the total jet energy came from charged hadrons.
	\item Less than 90\% of the total jet energy came from neutral hadrons.
	\item Less than 90\% of the total jet energy came from photons.
	\item Less than 99\% of the total jet energy came from electrons.
\end{itemize}

The efficiency with which jets are reconstructed and selected using the jet ID criteria is identical in data and simulations, so 
no jet efficiency corrections were applied in simulations.



%old material saved for reference
%Events from \WR interactions, characterized by different \mnul and \mWR, are simulated in the same three steps used to simulate 
%background interactions, described in Chapter \ref{sec:backgroundEstimation}.  The \PYTHIA generator implements a flexible, generic 
%\WR signal model that captures the main characteristics of theoretical models discussed in the literature 
%\cite{earlyLRSModel,lrsHiggsStageOne,lrsHiggsStageTwo,seeSawAndParityViolation,seeSawAndGUTs,lrsMassConstraints}, so \PYTHIA was used 
%to simulate $\WR \rightarrow \ell\ell jj$ events.
%
%Two classes of \WR signal events were simulated using \PYTHIA: one class was simulated through all three simulation steps, the other 
%was simulated only through the first step.  Events simulated through all three steps were produced with \mWR stepping from 0.8 to 6 
%$\TeV$ in increments of 0.2 $\TeV$, and $\mnul = \frac{1}{2}\mWR$.  Those events were used to calculate cross section $\times$ branching 
%ratio limits on the $\WR \rightarrow \ell\ell jj$ process as a function of \mWR.  The other class of events were produced with \mWR 
%stepping from 0.8 to 4.0 $\TeV$ in increments of 0.1 $\TeV$, and several \mnul values between $0.1 \leq \mnul < \mWR$ $\TeV$ at each 
%value of \mWR.  If no \WR signal is found, these events are used to transform the cross section $\times$ branching ratio limits for 
%$\mnul = \frac{1}{2}\mWR$ into \mWR and \mnul exclusion limits for 100 $\GeV$ $\lesssim \mnul < \mWR$.


%%%%%%%%%%%%%%%%%%%%%%%%%%%%%%%%%%%%%%%%%%%%%%%%%%%%%%%%%%%%%%%%%%%%%%%%%%%%%}}}
