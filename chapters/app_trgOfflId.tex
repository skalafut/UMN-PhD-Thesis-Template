%%%%%%%%%%%%%%%%%%%%%%%%%%%%%%%%%%%%%%%%%%%%%%%%%%%%%%%%%%%%%%%%%%%%%%%%%%%%%%%%
% app_wrMC.tex: Appendix on WR MC generation:
%%%%%%%%%%%%%%%%%%%%%%%%%%%%%%%%%%%%%%%%%%%%%%%%%%%%%%%%%%%%%%%%%%%%%%%%%%%%%%%%
\chapter{Trigger and Offline Selection Criteria}
\label{app_trgOfflId}
%%%%%%%%%%%%%%%%%%%%%%%%%%%%%%%%%%%%%%%%%%%%%%%%%%%%%%%%%%%%%%%%%%%%%%%%%%%%%%%%


%old material saved for reference
%Events from \WR interactions, characterized by different \mnul and \mWR, are simulated in the same three steps used to simulate 
%background interactions, described in Chapter \ref{sec:backgroundEstimation}.  The \PYTHIA generator implements a flexible, generic 
%\WR signal model that captures the main characteristics of theoretical models discussed in the literature 
%\cite{earlyLRSModel,lrsHiggsStageOne,lrsHiggsStageTwo,seeSawAndParityViolation,seeSawAndGUTs,lrsMassConstraints}, so \PYTHIA was used 
%to simulate $\WR \rightarrow \ell\ell jj$ events.
%
%Two classes of \WR signal events were simulated using \PYTHIA: one class was simulated through all three simulation steps, the other 
%was simulated only through the first step.  Events simulated through all three steps were produced with \mWR stepping from 0.8 to 6 
%$\TeV$ in increments of 0.2 $\TeV$, and $\mnul = \frac{1}{2}\mWR$.  Those events were used to calculate cross section $\times$ branching 
%ratio limits on the $\WR \rightarrow \ell\ell jj$ process as a function of \mWR.  The other class of events were produced with \mWR 
%stepping from 0.8 to 4.0 $\TeV$ in increments of 0.1 $\TeV$, and several \mnul values between $0.1 \leq \mnul < \mWR$ $\TeV$ at each 
%value of \mWR.  If no \WR signal is found, these events are used to transform the cross section $\times$ branching ratio limits for 
%$\mnul = \frac{1}{2}\mWR$ into \mWR and \mnul exclusion limits for 100 $\GeV$ $\lesssim \mnul < \mWR$.


%%%%%%%%%%%%%%%%%%%%%%%%%%%%%%%%%%%%%%%%%%%%%%%%%%%%%%%%%%%%%%%%%%%%%%%%%%%%%}}}
