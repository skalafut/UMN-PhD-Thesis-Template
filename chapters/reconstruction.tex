%%%%%%%%%%%%%%%%%%%%%%%%%%%%%%%%%%%%%%%%%%%%%%%%%%%%%%%%%%%%%%%%%%%%%%%%%%%%%%%%
% reconstruction.tex:
%%%%%%%%%%%%%%%%%%%%%%%%%%%%%%%%%%%%%%%%%%%%%%%%%%%%%%%%%%%%%%%%%%%%%%%%%%%%%%%%
\chapter{Event Reconstruction}
\label{sec:reco_chapter}
%%%%%%%%%%%%%%%%%%%%%%%%%%%%%%%%%%%%%%%%%%%%%%%%%%%%%%%%%%%%%%%%%%%%%%%%%%%%%%%%

Electrons, muons and jets expected from \WR and \nul decays were reconstructed from charged particle 
tracks measured by the silicon tracker, and energy deposits measured by the calorimeters and muon 
detectors.  The high expected energy of final state leptons and jets motivated the use of specific 
lepton and jet reconstruction algorithms described herein.

\section{Electron Reconstruction}
\label{sec:eleReco}
Electrons ($e \equiv e^{\pm}$) were the only final state particles that, on average, lost a significant 
amount of energy in the tracker.  They lost energy through bremstrahhlung, and the resulting bremsstrahlung 
photons were detected outside the tracker by the ECAL.  Electron tracks were reconstructed from hits in 
individual tracker layers using a dedicated algorithm that estimated the energy lost in and between 
successive layers.  Electron $(\eta, \phi)$ trajectories were determined from their reconstructed tracks, 
but their energies $\Et$ were dictated by ECAL measurements.

%%%%%%%%%%%%%%%%%%%%%%%%%%%%%%%%%%%%%%%%%%%%%%%%%%%%%%%%%%%%%%%%%%%%%%%%%%%%%%%%
