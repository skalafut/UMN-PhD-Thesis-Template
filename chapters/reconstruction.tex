%%%%%%%%%%%%%%%%%%%%%%%%%%%%%%%%%%%%%%%%%%%%%%%%%%%%%%%%%%%%%%%%%%%%%%%%%%%%%%%%
% reconstruction.tex:
%%%%%%%%%%%%%%%%%%%%%%%%%%%%%%%%%%%%%%%%%%%%%%%%%%%%%%%%%%%%%%%%%%%%%%%%%%%%%%%%
\chapter{Event Reconstruction}
\label{sec:reco_chapter}
%%%%%%%%%%%%%%%%%%%%%%%%%%%%%%%%%%%%%%%%%%%%%%%%%%%%%%%%%%%%%%%%%%%%%%%%%%%%%%%%

Electrons, muons and jets expected from \WR and \nul decays were reconstructed from charged particle 
tracks measured by the silicon tracker, and energy deposits measured by the calorimeters and muon 
detectors.  The high expected energy of final state leptons and jets motivated the use of specific 
lepton and jet reconstruction algorithms described herein.


\section{Electron Reconstruction}
\label{sec:eleReco}
Electrons ($\equiv e^{\pm}$) were the only particles considered in the \WR search that, on average, lost non-negligible 
amounts of energy in the tracker.  They lost energy through bremstrahhlung, and the resulting bremsstrahlung 
photons were detected outside the tracker by the ECAL.  Electron tracks were reconstructed from hits in 
individual tracker layers along helical paths using a dedicated algorithm that also estimated the energy lost.  
Reconstructed tracks determined electron $(\eta, \phi)$ trajectories, but ECAL measurements dictated their 
energies.

After traversing the tracker, electrons impinged on ECAL crystals and lost substantially all of their 
energy through radiative processes.  These processes resulted in showers of lower energy particles 
that enabled measurements of initial electron energies.  Showers from individual electrons were contained 
in 5 $\times$ 5 crystal regions centered on the first crystal hit by an electron, so electron energies 
were measured in 5 $\times$ 5 crystal superclusters (SCs).  The large SC size also facilitated 
bremsstrahlung photon recovery, and thus more precise measurements of the energy lost by the electron 
before the ECAL.  Following SC reconstruction, the $(\eta, \phi)$ positions of SCs were compared to 
the $(\eta, \phi)$ trajectories of electron candidate tracks.  Each reconstructed electrons was identified 
as a SC and at least one track with matching $(\eta, \phi)$ coordinates.


\section{Muon Reconstruction}
\label{sec:muReco}
Muon ($\equiv \mu^{\pm}$) reconstruction started with reconstructing tracks in the silicon tracker.  
Hits in individual tracker layers were reconstructed into helical tracks, whose $(\eta, \phi)$ 
trajectories determined the directions of reconstructed muons.  Reconstructed tracks also provided 
initial muon momentum estimates that were further refined by muon detector measurements.

Muons, on average, lost negligible amounts of energy before entering the muon detectors.  Hits in 
muon chambers were reconstructed into helical tracks, and their radii of curvature enabled more precise 
momentum measurements.  Muon detector tracks were extrapolated back to the silicon tracker, and 
then the $\eta$ and $\phi$ of silicon tracker and muon detector tracks were compared.  Each 
reconstructed muon was identified as a muon detector track whose trajectory matched a silicon 
tracker track. 

The momentum of muons was determined using silicon tracker and muon detector measurements.  Four 
reconstruction algorithms fitted four continuous tracks that represented a muon's trajectory from 
the silicon tracker through the muon detectors \cite{cmsMuonRecoRunOne}.  Each algorithm combined 
muon detector and silicon tracker measurements in a different way, and could exclude measurements 
with large uncertainty.  Each continuous track fitted to silicon tracker and muon detector data was 
distinguished by a fit uncertainty $\chi^{2}/nDOF$ and momentum uncertainty $\sigma(\pt)/\pt$.  The 
fitted track with the lowest fit and momentum uncertainties determined the reconstructed muon momenta.


\section{Jet Reconstruction}
\label{sec:jetReco}
As a result of hadronization, photon radiation and leptonic weak decays, quarks produced in pp interactions 
created showers of photons, hadrons and charged leptons that carried the initial quark energies.  

Jets produced in proton-proton collisions contained long lived charged and neutral hadrons, photons from 
$\pi^{0} \rightarrow \gamma\gamma$ decays, and charged leptons from heavy quark decays.  Jets reconstruction began 
by reconstructing electrons, muons, photons, and charged and neutral hadrons.  Photons were reconstructed using the same algorithm 
used to reconstruct electrons, excluding matching ECAL SCs to reconstructed tracks.  On average, a hadron that 
impinged on the ECAL had one significant interaction with the PbWO$_{4}$ before entering 
the HCAL.  As a result, reconstructing charged and neutral hadrons started with the ECAL.  A modified version of 
the electron SC reconstruction algorithm, with looser requirements on the $\eta$ distribution of energy 
in the SC, was used to reconstruct hadronic energy clusters in the ECAL.  Then, energy deposited in individual 
HCAL towers was reconstructed into tower clusters (TC).  Independently, tracks not consistent with muons or 
electrons were reconstructed using silicon tracker measurements.  Every neutral hadron was built from an HCAL TC, and 
an ECAL SC if a geometric match was found.  Hadron tracks were extrapolated from the tracker to the HCAL, and 
each charged hadron was built from a TC and geometrically matching track, and an ECAL SC if a geometric match 
with the track was found.  After all particles were reconstructed, jets were reconstructed starting with 
charged particle tracks.  In each event, charged particles that traced back to the interaction 
vertex with the highest track $\Sigma \pt$ (the primary vertex) were used to tag different jets and their 
directions.  Then, all charged particles from the primary vertex, and all photons and neutral hadrons in the event 
were placed in a list of jet constituent candidates, which was subsequently analyzed using the anti-$k_{T}$ 
algorithm \cite{antikt}.  Jet constituents were iteratively clustered into jets based on the $\pt$ of each 
constituent and its $(\eta,\phi)$ distance from the jet axis defined by a charged particle track.  The majority 
of jet constituents fell within $\Delta R < 0.4$ of the jet axis, but it was possible for low energy constituents to 
be futher away.  A jet's energy was defined as the total energy of its constituents, and its $(\eta,\phi)$ 
trajectory was the $\pt$-averaged trajectory of all charged constituents.



%%%%%%%%%%%%%%%%%%%%%%%%%%%%%%%%%%%%%%%%%%%%%%%%%%%%%%%%%%%%%%%%%%%%%%%%%%%%%%%%
