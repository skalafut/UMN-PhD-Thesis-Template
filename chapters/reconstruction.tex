%%%%%%%%%%%%%%%%%%%%%%%%%%%%%%%%%%%%%%%%%%%%%%%%%%%%%%%%%%%%%%%%%%%%%%%%%%%%%%%%
% reconstruction.tex:
%%%%%%%%%%%%%%%%%%%%%%%%%%%%%%%%%%%%%%%%%%%%%%%%%%%%%%%%%%%%%%%%%%%%%%%%%%%%%%%%
\chapter{Event Reconstruction}
\label{sec:reco_chapter}
%%%%%%%%%%%%%%%%%%%%%%%%%%%%%%%%%%%%%%%%%%%%%%%%%%%%%%%%%%%%%%%%%%%%%%%%%%%%%%%%

Electrons, muons and jets expected from \WR and \nul decays were reconstructed from charged particle 
tracks measured by the silicon tracker, and energy deposits measured by the calorimeters and muon 
detectors.  The high expected energy of final state leptons and jets motivated the use of specific 
lepton and jet reconstruction algorithms described herein.


\section{Electron Reconstruction}
\label{sec:eleReco}
Electrons ($\equiv e^{\pm}$) were the only final state particles that, on average, lost a non-negligible 
amount of energy in the tracker.  They lost energy through bremstrahhlung, and the resulting bremsstrahlung 
photons were detected outside the tracker by the ECAL.  Electron tracks were reconstructed from hits in 
individual tracker layers along helical paths using a dedicated algorithm that also estimated the energy lost.  
Later in reconstruction, tracks determined electron $(\eta, \phi)$ trajectories, but ECAL measurements 
dictated their energies ($\Et$).

After traversing the tracker, electrons impinged on ECAL crystals and lost substantially all of their 
energy through radiative processes.  These processes produced produced showers of lower energy particles 
that enabled measurements of initial electron energies.  Showers from individual electrons were contained 
in 5 $\times$ 5 crystal regions centered on the first crystal hit by an electron, so electron energies 
were measured in 5 $\times$ 5 crystal superclusters (SCs).  The large SC size also facilitated 
bremsstrahlung photon recovery, and thus more precise measurements of the energy lost by the electron 
before the ECAL.  Following SC reconstruction, the $(\eta, \phi)$ positions of SCs were compared to 
the $(\eta, \phi)$ trajectories of electron candidate tracks.  Pairs of SCs and tracks with matching 
spatial coordinates were identified as reconstructed electrons.


\section{Muon Reconstruction}
\label{sec:muReco}
Muons ($\equiv \mu^{\pm}$) were .

%%%%%%%%%%%%%%%%%%%%%%%%%%%%%%%%%%%%%%%%%%%%%%%%%%%%%%%%%%%%%%%%%%%%%%%%%%%%%%%%
