%%%%%%%%%%%%%%%%%%%%%%%%%%%%%%%%%%%%%%%%%%%%%%%%%%%%%%%%%%%%%%%%%%%%%%%%%%%%%%%%
% reconstruction.tex:
%%%%%%%%%%%%%%%%%%%%%%%%%%%%%%%%%%%%%%%%%%%%%%%%%%%%%%%%%%%%%%%%%%%%%%%%%%%%%%%%
\chapter{Multiple Choice}
\label{Multiple Choice}
%%%%%%%%%%%%%%%%%%%%%%%%%%%%%%%%%%%%%%%%%%%%%%%%%%%%%%%%%%%%%%%%%%%%%%%%%%%%%%%%
problem 1: e 
\newline
use energy conservation, equating the initial spring potential energy to the final gravitational\newline
potential energy of the mass. If the spring constant is doubled, the initial kinetic energy of\newline
the box will be twice the kinetic energy with the original spring. In addition, since the final\newline
height of the box is linearly related to the initial kinetic energy, doubling the spring constant\newline
will double the final box height. Finally, doubling the spring constant will increase the initial\newline
velocity of the box just after it leaves the spring by $\sqrt{2}$.\newline
problem 2: c
\newline
problem 3: b
\newline
problem 4: a
\newline
integrate the force F over a distance from 0 to 2 meters to find the word done\newline
$\int_0^2 \!(-3x^{2} + 6x) dx = 4.0$ Joules\newline
problem 5: b\newline

use energy conservation to determine the maximum distance H above ground a projectile can travel\newline
given an initial velocity of $2.78 \frac{m}{s}$ directed straight up. At any launch angle other than\newline
$90$ degrees (straight up) the maximum height will be smaller.\newline
$(0.5)mv_{0}^{2} = mgH \rightarrow H = \frac{v_{0}^{2}}{2g} = 0.394$ meters\newline
%%%%%%%%%%%%%%%%%%%%%%%%%%%%%%%%%%%%%%%%%%%%%%%%%%%%%%%%%%%%%%%%%%%%%%%%%%%%%%%%
