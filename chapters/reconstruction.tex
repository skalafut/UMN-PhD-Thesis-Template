%%%%%%%%%%%%%%%%%%%%%%%%%%%%%%%%%%%%%%%%%%%%%%%%%%%%%%%%%%%%%%%%%%%%%%%%%%%%%%%%
% reconstruction.tex:
%%%%%%%%%%%%%%%%%%%%%%%%%%%%%%%%%%%%%%%%%%%%%%%%%%%%%%%%%%%%%%%%%%%%%%%%%%%%%%%%
\chapter{Multiple Choice}
\label{Multiple Choice}
%%%%%%%%%%%%%%%%%%%%%%%%%%%%%%%%%%%%%%%%%%%%%%%%%%%%%%%%%%%%%%%%%%%%%%%%%%%%%%%%
problem 1: e 
\newline
use energy conservation, equating the initial spring potential energy to the final gravitational\newline
potential energy of the mass. If the spring constant is doubled, the initial kinetic energy of\newline
the box will be twice the kinetic energy with the original spring. In addition, since the final\newline
height of the box is linearly related to the initial kinetic energy, doubling the spring constant\newline
will double the final box height. Finally, doubling the spring constant will increase the initial\newline
velocity of the box just after it leaves the spring by $\sqrt{2}$.\newline
problem 2: c
\newline
problem 3: b
\newline
problem 4: a
\newline
integrate the force F over a distance from 0 to 2 meters to find the word done\newline
$\int_0^2 \!(-3x^{2} + 6x) dx = 4.0$ Joules\newline
problem 5: d\newline

use kinematics to determine the maximum horizontal distance X a projectile\newline
can travel given an initial velocity of $2.78 \frac{m}{s}$ directed\newline
at an angle of $45$ degrees above the horizontal.\newline
At any launch angle other than $45$ degrees the maximum horizontal\newline
distance X will be smaller.\newline
first find the time spent in the air $t_{air}$\newline
$y(t_{air}) = 0 = v_{0}\sin(45)t_{air} - \frac{g}{2}t_{air}^{2}$\newline
$t_{air} = \frac{-1}{g}(-v_{0}\sin(45) - \sqrt{(v_{0}\sin(45))^{2}})$\newline
$t_{air} = 0.401$ seconds\newline
now find the max horizontal distance X\newline
X = $v_{0}\cos(45)t_{air} = 0.789$ meters\newline

%%%%%%%%%%%%%%%%%%%%%%%%%%%%%%%%%%%%%%%%%%%%%%%%%%%%%%%%%%%%%%%%%%%%%%%%%%%%%%%%
