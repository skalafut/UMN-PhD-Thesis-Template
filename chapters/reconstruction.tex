%%%%%%%%%%%%%%%%%%%%%%%%%%%%%%%%%%%%%%%%%%%%%%%%%%%%%%%%%%%%%%%%%%%%%%%%%%%%%%%%
% reconstruction.tex:
%%%%%%%%%%%%%%%%%%%%%%%%%%%%%%%%%%%%%%%%%%%%%%%%%%%%%%%%%%%%%%%%%%%%%%%%%%%%%%%%
\chapter{Event Reconstruction}
\label{sec:reco_chapter}
%%%%%%%%%%%%%%%%%%%%%%%%%%%%%%%%%%%%%%%%%%%%%%%%%%%%%%%%%%%%%%%%%%%%%%%%%%%%%%%%

Electrons, muons and jets expected from \WR and \nul decays traversed multiple CMS sub-detectors, 
as shown in Figure \ref{fig:particleTrajectories}.  Their trajectories and energies were measured 
from charged particle tracks reconstructed by the silicon tracker, and energy deposits reconstructed 
by the calorimeters and the muon detectors.  The high energy of expected leptons and jets motivated 
the use of specific lepton and jet reconstruction algorithms described herein.

\begin{figure}[h]
	\centering
	\includegraphics[width=0.75\textwidth]{figures/flowOfParticlesThroughCMS.png}
	\caption{Typical trajectories of particles travelling through CMS, from CERN.}
	\label{fig:particleTrajectories}
\end{figure}


\section{Electron Reconstruction}
\label{sec:eleReco}
Electrons ($e^{\pm}$) were the only particles expected from \WR decays that, on average, lost non-negligible 
energy in the tracker.  They lost energy by emitting bremsstrahlung photons in $\phi$, which were detected 
by the ECAL.  A dedicated electron track algorithm reconstructed helical electron tracks from hits in tracker 
layers, and estimated the bremsstrahlung energy lost.  Electron $(\eta, \phi)$ trajectories were measured 
from reconstructed tracks, and the ECAL measured their energies.

After traversing the tracker, electrons impinged on ECAL crystals and showered into lower energy particles, 
through which electron energies were measured.  Approximately 94\% of the energy of an electron 
incident on ECAL was measured in a 3 $\times$ 3 crystal area, but the energy lost in the tracker was spread 
over a larger area in $\phi$.  To measure the total energy of each electron, ECAL energy deposits were 
reconstructed in dynamically sized superclusters (SCs), as shown in Figure \ref{fig:eleTrackAndSC}.  Following 
SC reconstruction, the $(\eta, \phi)$ positions of SCs were compared to the $(\eta, \phi)$ trajectories of 
electron candidate tracks.  Each reconstructed electron was identified as a SC that matched at least one track.

\begin{figure}[h]
	\centering
	\includegraphics[width=0.75\textwidth]{figures/electronTrackAndSupercluster.png}
	\caption{The trajectory of a typical electron through the tracker and the ECAL.}
	\label{fig:eleTrackAndSC}
\end{figure}


\section{Muon Reconstruction}
\label{sec:muReco}
Muon ($\mu^{\pm}$) reconstruction started with reconstructing tracks in the silicon tracker.  
Hits in individual tracker layers were reconstructed into helical tracks, and their $(\eta, \phi)$ 
trajectories determined reconstructed muon trajectories.  Muon momenta were measured with 
reconstructed tracks, and were refined by muon detector measurements.

On average, muons lost negligible energy before entering the muon detectors.  Hits in 
the muon chambers were reconstructed into helical tracks, and their radii of curvature were 
measured to improve the precision of muon momenta measurements.  Muon detector tracks were 
extrapolated back to the silicon tracker where the $(\eta, \phi)$ trajectories of silicon tracker 
and muon detector tracks were compared.  Each reconstructed muon was identified as a muon detector 
track whose trajectory matched a silicon tracker track.

The momentum of muons was determined using silicon tracker and muon detector measurements.  Four 
reconstruction algorithms fitted four continuous tracks \cite{cmsMuonRecoRunTwo} to silicon tracker 
and muon detector hits to estimate a muon's trajectory through CMS, represented in Figure 
\ref{fig:particleTrajectories}.  Each algorithm combined muon detector and silicon tracker measurements 
in a different way, and could exclude measurements with large uncertainty.  The precision of each 
continuous track was identified by a fit uncertainty $\chi^{2}/nDOF$ and momentum uncertainty 
$\sigma(\pt)/\pt$.  The track with the lowest uncertainties determined the reconstructed muon momenta.  
This procedure improved the momentum resolution for muons with $\pt > 200$ $\GeV$, which were 
expected in a significant fraction of $\WR \rightarrow \mu\mu jj$ events, as shown in Table 
\ref{tab:wrHighPtMuons}.

\begin{table}[h]
	\caption{Fraction of expected $\WR \rightarrow \mu\mu jj$ events that had at least one muon with $\pt > 200$ $\GeV$. 
	($\mnul = \frac{1}{2}\mWR$)}
	\label{tab:wrHighPtMuons}
	\centering
	\begin{tabular}{c|c}
		\mWR ($\TeV$) & Fraction of events with at least one high-$\pt$ muon (\%) \\  \hline
		1.0 &  80.  \\
		2.0 &  95.  \\ 
		3.0 &  98.  \\ \hline
	\end{tabular}
\end{table}


\section{Jet Reconstruction}
\label{sec:jetReco}
Quarks, through hadronization, photon radiation, and leptonic weak decays, produced jets of photons, 
hadrons, and leptons.  Jet reconstruction began by reconstructing photons, 
electrons and muons, and charged and neutral hadrons.  Each charged hadron was reconstructed in a 
similar way to electrons, by geometrically matching an energy measured in an HCAL tower to a 
reconstructed track.  A charged hadron could also contain an ECAL SC if the SC $(\eta, \phi)$ position 
matched the HCAL tower.  Each neutral hadron was reconstructed as an HCAL tower, possibly associated with 
an ECAL SC, whose $(\eta, \phi)$ position did not match any reconstructed track.  Similarly, 
each photon was reconstructed as an ECAL SC whose position did not match any reconstructed track.  
After reconstructing individual particles, jets were reconstructed as clusters of individual particles, 
as shown in Figure \ref{fig:jetClustering}.

\begin{figure}[h]
	\centering
	\includegraphics[width=0.75\textwidth]{figures/jetClusteringInCMS.png}
	\caption{A cone of reconstructed particles clustered into a jet, with the reconstructed vertex on the right.  
	From the CMS Experiment.}
	\label{fig:jetClustering}
\end{figure}

In each collision event, every reconstructed particle was considered a jet candidate, except charged 
hadrons that did not come from the reconstructed vertex with the highest $\sum \pt$.  Using the 
anti-$k_{T}$ algorithm \cite{antikt}, reconstructed particles were clustered into jets 
based on their individual energies and trajectories.  The jet energies were the total energies of 
all constituents, and the jet $(\eta, \phi)$ trajectories were the energy-weighted average trajectories 
of all constituents.  The majority of each jet's energy was contained in a cone with radius $\Delta R = 0.4$.


\section{Conclusion}
\label{sec:recoConclusion}
Energetic electrons, muons and jets produced in collisions were reconstructed using several CMS sub-detectors 
and dedicated reconstruction algorithms.  Additional selections were applied to these leptons and jets to 
increase sensitivity to the \WR signal relative to ST processes that produced $\ell\ell jj$ events.

%%%%%%%%%%%%%%%%%%%%%%%%%%%%%%%%%%%%%%%%%%%%%%%%%%%%%%%%%%%%%%%%%%%%%%%%%%%%%%%%
