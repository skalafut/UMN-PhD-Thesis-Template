%%%%%%%%%%%%%%%%%%%%%%%%%%%%%%%%%%%%%%%%%%%%%%%%%%%%%%%%%%%%%%%%%%%%%%%%%%%%%%%%
% event_selection.tex: 
%%%%%%%%%%%%%%%%%%%%%%%%%%%%%%%%%%%%%%%%%%%%%%%%%%%%%%%%%%%%%%%%%%%%%%%%%%%%%%%%
\chapter{Event Selection}
\label{sec:event_selection_chapter}
%%%%%%%%%%%%%%%%%%%%%%%%%%%%%%%%%%%%%%%%%%%%%%%%%%%%%%%%%%%%%%%%%%%%%%%%%%%%%%%%

LRS models presented earlier predict the existence of a \WR boson and heavy neutrinos \nul, and that their 
production and decay yields events where two high energy jets and two high energy charged leptons are 
reconstructed.  Using collision data recorded by CMS from September through November 2015, evidence of 
\WR and \nul production was searched for in events with $\ell\ell jj$ ($\ell = e,\mu$) final states.  
Events were selected using lepton and jet selections developed from features of expected ST backgrounds 
and $\WR \rightarrow \ell\nul rightarrow \ell\ell jj$ decays.


\section{\WR and \nul Signal Features}
\label{sec:signalFeatures}
Lepton and jet selections were driven by several model-independent features of \WR and \nul decays.  \mWR 
was expected to be large, 2 $\TeV$ or more, relative to the center-of-mass collision energy of 13 $\TeV$, 
so \WR bosons would have been produced with low momentum.  Therefore, the lighter lepton and jet decay 
products, on average, were distributed uniformly in $\phi$, and concentrated in the region $|\eta| < 2.5$, 
as indicated in Figures \ref{fig:wrLeptonEtas} and \ref{fig:wrJetEtas}.  Furthermore, the large expected 
\mWR yielded reconstructed leptons and jets with several hundred $\GeV$ or more of $\pt$, as shown in 
Figures \ref{fig:wrLeptonPts} and \ref{fig:wrJetPts}.  The decay of the \WR into a charged lepton $\ell$ 
with negligible mass $m_{\ell} \ll \mnul$ and a \nul with non-negligible mass $\mnul < \mWR$ yielded 
additional, distinguishing features of $\WR \rightarrow \ell\ell jj$ decays.  To conserve momentum in 
the $\WR \rightarrow \ell \nul$ decay, the charged lepton $\ell$ recoiled against the neutrino \nul.  The 
second charged lepton produced by the \nul decay often recoiled against the first charged lepton, and the 
lepton pair, on average, had a large dilepton mass $\Mll$ (Figure \ref{fig:wrSigMll}).


\begin{figure}[btp]
	\centering
	\subfigure{
		\includegraphics[width=0.65\textwidth]{figures/missingImage.png}
	}
	\subfigure{
		\includegraphics[width=0.65\textwidth]{figures/missingImage.png}
	}
	\label{fig:wrLeptonEtas}
	\caption{The $\eta$ distribution of the leading (sub-leading) reconstructed muon is shown on the left (right) for 
		$\WR \rightarrow \mu\mu jj$ events with $\mWR = 2.0 \TeV$ and $\mnul = 1.0 \TeV$.  The electron 
	distributions are identical, except for the ECAL barrel-endcap gap where electrons are not reconstructed.}
\end{figure}

\begin{figure}[btp]
	\centering
	\subfigure{
		\includegraphics[width=0.65\textwidth]{figures/missingImage.png}
	}
	\subfigure{
		\includegraphics[width=0.65\textwidth]{figures/missingImage.png}
	}
	\label{fig:wrJetEtas}
	\caption{The $\eta$ distribution of the leading (sub-leading) reconstructed jet is shown on the left (right) for 
		$\WR \rightarrow \mu\mu jj$ events with $\mWR = 2.0 \TeV$ and $\mnul = 1.0 \TeV$.}
\end{figure}

\begin{figure}[btp]
	\centering
	\subfigure{
		\includegraphics[width=0.65\textwidth]{figures/missingImage.png}
	}
	\subfigure{
		\includegraphics[width=0.65\textwidth]{figures/missingImage.png}
	}
	\label{fig:wrLeptonPts}
	\caption{The $\pt$ distribution of the leading (sub-leading) reconstructed muon is shown on the left (right) for 
		$\WR \rightarrow \mu\mu jj$ events with $\mWR = 2.0 \TeV$ and $\mnul = 1.0 \TeV$.}
\end{figure}

\begin{figure}[btp]
	\centering
	\subfigure{
		\includegraphics[width=0.65\textwidth]{figures/missingImage.png}
	}
	\subfigure{
		\includegraphics[width=0.65\textwidth]{figures/missingImage.png}
	}
	\label{fig:wrJetPts}
	\caption{The $\pt$ distribution of the leading (sub-leading) reconstructed jet is shown on the left (right) for 
		$\WR \rightarrow \mu\mu jj$ events with $\mWR = 2.0 \TeV$ and $\mnul = 1.0 \TeV$.}
\end{figure}

\begin{figure}[btp]
	\centering
	\includegraphics[width=0.65\textwidth]{figures/missingImage.png}
	\label{fig:wrSigMll}
	\caption{The distribution of the dilepton mass $\Mll$ is shown for $\WR \rightarrow \mu\mu jj$ events with 
	$\mWR = 2.0 \TeV$ and $\mnul = 1.0 \TeV$.}
\end{figure}







\section{Online Event Selection}
\label{sec:triggers}
Charged leptons that came from proton-proton interactions were identified by the trigger system.  Based 
on the multiplicity and energy of charged leptons, the trigger system selected events during 
collisions, and saved these events to permanent storage for further analysis.  As described in Chapter 
\ref{sec:experiment_chapter}, triggers selected events in two steps.  In the first step, one or more Level-1 triggers were 
required to fire.  Then, in regions where Level-1 triggers fired, local reconstruction was run, and 
selections were applied to reconstructed particles.  Events from data and simulations were only 
used if they passed a High Level trigger.  The specific triggers used in this search are presented 
here.

In $\WR \rightarrow \ell\ell jj$ processes predicted by LRS models, the expected mass of the \WR ($\gtrsim 1 \TeV$) was 
so large that the final state leptons were expected to have $\pt \gtrsim 60 \GeV$.  ST processes that 
produced two charged leptons with such high $\pt$ were predicted to have very low rate in 2015, so the 
triggers used in this search sought one or two high $\pt$ charged leptons.

Events used in the ee channel \WR search ($pp \rightarrow \WR \rightarrow eejj$) were first selected by Level-1 triggers 
that required: 

\begin{itemize}
	\item At least 40 GeV of energy was measured in one ECAL supercluster (SC), defined as a 5 $\times$ 5 crystal region.
	\item Or, at least 22 GeV of energy was measured in one SC, and at least 10 GeV of energy was measured in 
		another, non-overlapping SC.
\end{itemize}

Then, events were saved to permanent storage if the following double electron High Level trigger requirements 
were met: 

\begin{itemize}
	\item Two non-overlapping ECAL SCs were detected with energy $>$ 33 GeV.
	\item For each SC:
	\begin{itemize}
		\item The ratio of hadronic energy in the HCAL tower behind the SC to the SC energy was low, $\frac{E_{HCAL}}{E_{SC}} < 0.15$ in the barrel, $< 0.1$ in the endcap.
		\item For SCs in the barrel or endcap, 90\% of the SC energy was measured in an $(\eta, \phi)$ region that was two crystals wide in $\eta$.
		\item For SCs in the barrel, a reconstructed track with hits in at least two pixel tracker layers extrapolated to the $z_{SC}$ 
			SC center within 2.3 \cm, and the $(\eta_{SC}, \phi_{SC})$ SC center within the $(\eta, \phi)$ area of one ECAL crystal.
	\end{itemize}
\end{itemize}

The High Level trigger selections for electrons differ between the barrel and endcap regions because the ECAL 
crystal sizes and orientations relative to the $x$ and $y$ axes differ between the barrel and endcap, as 
explained in Chapter \ref{sec:experiment_chapter}.

A second set of ee channel events were used only to estimate backgrounds.  These events were first 
selected online using a Level-1 trigger that required $>$ 30 GeV of energy be measured in an ECAL SC 
with $|\eta| < 2.1$.  Following the Level-1 selection, events were saved to permanent storage if the 
following double electron High Level trigger requirements were met:

\begin{itemize}
	\item One SC was detected with energy $>$ 30 GeV.
	\item For the SC with energy $>$ 30 GeV:
	\begin{itemize}
		\item For SCs in the barrel or endcap, 90\% of the SC energy was measured in an $(\eta, \phi)$ region that was two crystals wide in $\eta$.
		\item The ratio of hadronic energy in the HCAL tower behind the SC to the SC energy was low, $\frac{E_{HCAL}}{E_{SC}} < 0.055$ in the barrel, $< 0.07$ in the endcap.
		\item In a cone of radius $\Delta R =$ 0.3 centered on the SC ($\thicksim$900 ECAL crystals, $\thicksim$35 HCAL towers in the cone):
		\begin{itemize}
			\item The fraction of the total ECAL energy in the cone not associated with the SC is low, $\frac{E_{ECAL}}{E_{SC}} < 0.225$ in the barrel, $< 0.121$ in the endcap.
			\item The total HCAL energy in the cone is small compared to the SC energy, $\frac{E_{HCAL}}{E_{SC}} < 0.155$ in the barrel, $< 0.16$ in the endcap.
		\end{itemize}
		\item For SCs in the barrel or endcap, a reconstructed track with hits in at least two pixel tracker layers extrapolates to the 
			$z_{SC}$ SC center within 1 \cm, and the $(\eta_{SC}, \phi_{SC})$ SC center within the $(\eta, \phi)$ area of $\frac{1}{2}$ ECAL crystal.
		\item For SCs in the barrel or endcap, the SC energy and the matching reconstructed track momentum cannot differ by more than 50\%
	\end{itemize}
	\item A second SC was detected with energy $>$ 4 GeV.
\end{itemize}


Events used in the muon channel \WR search ($pp \rightarrow \WR \rightarrow \mu\mu jj$) were first selected online using 
a Level-1 trigger that required $>$ 16 GeV of momentum be measured in a muon DT or CSC detector.  Following 
the Level-1 selection, events were saved to permanent storage if the following single muon High Level trigger 
requirements were met: 

\begin{itemize}
	\item The same requirements were applied to muon candidates in the barrel and endcap.
	\item A global curve representing a muon candidate was fit to a reconstructed track and at least one muon detector hit with $\chi^{2}/nDOF <$ 20.
	\item In the $(x,y)$ plane, the distance between the origin of the muon track and the primary vertex was $<$ 1 \mm.
	\item The reconstructed muon track had $p_{T} >$ 50 GeV.
\end{itemize}

A second set of $\mu\mu$ channel events were used only to estimate backgrounds.  These events were first 
selected online by a Level-1 trigger, which required $>$ 20 GeV of momentum be measured in a muon 
DT or CSC detector.  Following the Level-1 selection, events were saved to permanent storage if the 
following single muon High Level trigger requirements were met:

\begin{itemize}
	\item Unless noted otherwise, the same requirements were applied to muon candidates in the barrel and endcap.
	\item A global curve representing a muon candidate was fit to a reconstructed track and at least one muon detector hit with $\chi^{2}/nDOF <$ 20.
	\item In the $(x,y)$ plane, the distance between the origin of the muon track and the primary vertex was $<$ 1 \mm.
	\item The reconstructed muon track had $p_{T} >$ 22 GeV.
	\item In a cone of radius $\Delta R =$ 0.3 centered on the muon detector energy cluster ($\thicksim$900 ECAL crystals, $\thicksim$35 HCAL towers in the cone):
	\begin{itemize}
		\item The total ECAL energy in the cone is small compared to the muon cluster energy, $\frac{E_{ECAL}}{E_{\mu}} < 0.11$ in the barrel, $< 0.08$ in the endcap.
		\item The total HCAL energy in the cone is small compared to the muon cluster energy, $\frac{E_{HCAL}}{E_{\mu}} < 0.21$ in the barrel, $< 0.22$ in the endcap.
		\item The sum $p_{T,other}$ of all tracks in the cone excluding the muon track is small compared to the muon track $p_{T,\mu}$, 
			$\frac{p_{T,other}}{p_{T,\mu}} < 0.09$ in the barrel and endcap.
	\end{itemize}
\end{itemize}


As stated in Chapter \ref{sec:lrsPhenomenology}, it is assumed that the $\WR$ decay cannot violate lepton 
flavor conservation.  As a result, the search presented here did not seek evidence of the LRS model in 
events with one electron, one muon and two jets in the final state.  However, events in the $e\mu$ channel 
($e\mu jj$ final state) were used to estimate top quark backgrounds using a procedure described later.  The $e\mu$ 
channel events were first selected by a Level-1 trigger that required $>$ 16 GeV of momentum be 
measured in one muon DT or CSC detector.  Events that passed the Level-1 trigger were saved to permanent 
storage if the following electron $+$ muon High Level trigger requirements were met:

\begin{itemize}
	\item A global curve representing a muon candidate was fit to a reconstructed track and at least one muon detector hit with $\chi^{2}/nDOF <$ 20.
	\item In the $(x,y)$ plane, the distance between the origin of the reconstructed muon track and the primary vertex was $<$ 1 \mm.
	\item The reconstructed muon track had $p_{T} >$ 30 GeV.
	\item One ECAL SC was detected with energy $>$ 30 GeV.
	\item For the SC:
	\begin{itemize}
		\item The ratio of hadronic energy in the tower behind the SC to the SC energy must be low, $\frac{E_{HCAL}}{E_{SC}} < 0.15$ in the barrel, $< 0.1$ in the endcap.
		\item For SCs in the barrel or endcap, 90\% of the SC energy must be measured in an $(\eta, \phi)$ region that is two crystals wide in $\eta$.
		\item For SCs in the barrel, a reconstructed track with hits in at least two pixel tracker layers extrapolated to the 
			$z_{SC}$ SC center within 2.3 \cm, and the $(\eta_{SC}, \phi_{SC})$ SC center within the $(\eta, \phi)$ area of one ECAL crystal.
	\end{itemize}
\end{itemize}


\subsection{Data}
\label{sec:collisionData}

%The LHC started colliding protons at $\sqrt{s} = 13\TeV$ center-of-mass energy in April 2015.  From 
%April until mid August, proton bunches in each beam were separated by 50 \ns.  The data collected 
%in this period, $\thicksim$0.2 fb$^{-1}$, was used to calibrate and align all CMS subdetector systems, but 
%was not used in the search presented here for the following reason.  Before 2015 collisions began it 
%was known that the amount of 50 \ns data collected would be small compared to the data collected with 
%25 \ns bunch spacing.  As a result, the vast resources needed for physics analyses, like Monte Carlo simulations 
%of SM processes used to estimate backgrounds, were only produced for data collected with 25 \ns bunch 
%spacing.  The person-power needed to develop the same resources for 50 \ns data would have been 
%detrimental to the quality of results produced with 25 \ns data.
%
%The LHC stopped collisions in the second half of August and early September, evidenced by the plateau 
%in Figure \ref{fig:lhc2015IntegLumi} \cite{lumi} during this time, to reconfigure the CERN accelerator system to 
%deliver proton-proton (pp) collisions with 25 \ns between proton bunches.  The bunch spacing was decreased to 
%increase the rate of pp collisions without increasing the number of interactions per pp collision event, 
%which makes particle reconstruction and identification more difficult.  Collisions at $\sqrt{s} = 13\TeV$ 
%resumed in September and continued until November 2015. During this period the LHC delivered approximately 
%4.0 fb$^{-1}$ of data, but problems with the CMS magnet limited the amount of data recorded 
%by CMS with the full 3.8 $\unit{T}$ magnetic field strength to 2.6 fb$^{-1}$.  The full 2.6 fb$^{-1}$ was used in this analysis, and was 
%divided into two run periods - Run2015C and Run2015D.  All data from September until mid October 
%was collected with similar beam conditions, like average instantaneous luminosity and the number of bunches 
%per beam, and constituted run era Run2015C.  In mid October collisions stopped for $\thicksim$1.5 weeks for 
%LHC maintenance, and to upgrade the LHC to deliver collisions with higher instantaneous luminosities, 
%approaching $6 \times 10^{33} \frac{protons}{cm^{2}s}$.  Data collected after this upgrade and until the end 
%of pp collisions in November constituted run era Run2015D.  As in Run2015C, all data in Run2015D was collected 
%with similar beam conditions.  Comparing the two run eras in Table \ref{tab:collisionDatasets}, most of the 
%data used in this analysis was collected in Run2015D.

%\begin{figure}[h]
%	\centering
%	\includegraphics[width=1.0\textwidth]{figures/int_lumi_per_day_cumulative_pp_2015.pdf}
%	\caption{Integrated luminosity delivered by the LHC and recorded by CMS in 2015.  Only data collected after 
%	September 1st, corresponding to 25 \ns bunch spacing, was used in this search.}
%	\label{fig:lhc2015IntegLumi}
%\end{figure}
%
%\begin{table}[h]
%\caption{The amount of data collected in each run era.}
%\label{tab:collisionDatasets}
%\centering
%\begin{tabular}{c|c}
%Run Era & Int. Lumi (fb$^{-1}$) \\  \hline
%	2015C &  0.02  \\
%	2015D &  2.62  \\ \hline
%\end{tabular}
%\end{table}

The collision data used in this search was delivered by the LHC at $\sqrt{s} = 13 \TeV$, and recorded by the CMS experiment 
from September until November 2015.  The amount of data collected corresponded to 2.6 fb$^{-1}$ \cite{lumi}, and events 
used in this search were selected during collisions using the High Level triggers described previously.  Following trigger 
selection, detector information was reconstructed into charged leptons, photons, and jets using procedures described later.  
%The fully reconstructed dataset representing all of 2015 was enormous ($\gtrsim 10^{4}$ terabytes), and contained much 
%more information than what was needed by any individual physics analysis.  To expedite the transformation of 
%collision data into a public physics result, reconstructed collision events from each run era were split into smaller 
%datasets distinguished by the High Level triggers that selected the events.  As discussed earlier in 
%Section \ref{sec:triggers}, collision events used in this analysis were selected if they had energy deposits consistent 
%with at least one muon, at least two electrons, or at least one muon and one electron.  Events selected by the single 
%muon triggers were assigned to the "SingleMuon" dataset, while those selected by the double electron and electron $+$ 
%muon triggers were assigned to the "DoubleEG" (EG for electron gamma) and "MuonEG" datasets, respectively.  
%The storage space consumed by these datasets largely came from object collections, representing quantities 
%like individual hits in all tracker or calorimeter cells, that were not needed by the majority of physics 
%analyses, including the one presented here.  Slimmed datasets were made by removing these object collections, and moving 
%their important information into more general object collections, like the sole collection representing all 
%reconstructed electrons.  Slimmed versions of the three datasets mentioned earlier were used in this 
%analysis, and in each run era individual datasets were $\thicksim$5 terabytes or smaller.

%In addition to the trigger requirements discussed above, collision datasets are cleaned of events
%in which global reconstruction problems occurred within the detector.  These include events
%where no primary vertex is reconstructed, and when anomalous noise appears in the
%tracker, calorimeters, or muon detectors.  Furthermore, events identified as coming from
%interactions between a single proton beam and beam pipe gas or other foreign material are
%removed.

\section{Monte Carlo}
\label{sec:MC}

Monte Carlo (\MC) simulations were used in this search to model two types of processes.  The first 
type was ST processes that resulted in the reconstruction of two charged leptons 
and two jets.  These included processes that produced two real charged leptons and two jets, like 
$pp \rightarrow Z+jets \rightarrow ll+jets$, and processes that produced multiple jets that 
were incorrectly reconstructed as charged leptons, like $pp \rightarrow W+jets \rightarrow l\nu+jets$.  
The second type was the \WR signal process $pp \rightarrow \WR \rightarrow l\nul \rightarrow lljj$ 
with different \mWR and \mnul masses.  
\MC simulations of both types of processes were produced using a multi-step procedure.  In the first step, 
a \MC generator, like \PYTHIA or \MADGRAPH, simulated 
collision events between two protons in a vacuum, without any magnetic field or CMS detector 
components.  In simulations, the generator accounted for:

\begin{itemize}
	\item The energy carried by each proton into the collision (6.5 \TeV).
	\item How the incoming proton energy was distributed amongst constituent quarks and gluons according 
		to parton distribution functions (PDFs).
	\item The masses of the \WR, \nul and all particles in the ST.
	\item The coupling strengths between fermions and the bosons that mediated interactions.
\end{itemize}

Any unstable particles produced in the interaction, like a $Z$ or \WR boson, decayed according to 
their branching fractions to other particles, and any free quarks or gluons that were produced were 
run through a hadronization simulation that emulates jet production described in Chapter \ref{intro_chapter}.  In the second 
simulation step, the effect of multiple pp interactions observed in real collisions is simulated by 
overlaying simulations of randomly chosen ST interactions onto the primary events simulated in the first 
step.  All pp interaction processes in the ST were assigned a probability proportional to their cross section 
times branching fraction, then several processes were chosen randomly, and the events they produced after 
the first simulation stage were overlaid on the primary events of interest.  The processes with the highest cross section times 
branching fraction by several orders of magnitude were inelastic pp scattering and 
other gluon mediated interactions that produced hadrons, so most of the randomly chosen events only added 
hadronic jets to each event.  After this overlaying procedure, the second simulation step used GEANT4 \cite{geant4} to simulate 
the propagation of all particles in the CMS magnetic field, their interactions with everything in the CMS detector, 
and the tracker, calorimeters and muon detectors to these particles.  Using the simulated detector response, 
the last step simulated the Level-1 and High Level triggers, and the reconstruction of interaction 
vertices, charged and neutral particles, and jets.  The same reconstruction software 
was used in simulations and collision data, and pertinent details of jet, electron and muon reconstruction 
algorithms are discussed later.  This multi-step procedure 
was used to simulate the \WR signal process at different \mWR and \mnul masses, and the ST processes 
summarized in Table \ref{tab:centrallyProducedMC}.


\begin{table}[bt]
\caption{Fully reconstructed \MC samples used in the search described here.  The \DY and t$\bar{t}$+jets events 
	were produced with a dilepton mass $M_{LL} > 50 \GeV$ selection applied to the two leptons coming from the hard interaction.  
Here, "size" represents the equivalent integrated luminosity of collision data needed to collect the same number 
of events in data.  For the \WR datasets the size is estimated based on the coupling assumptions stated in Chapter \ref{sec:lrsPhenomenology}.}
\label{tab:centrallyProducedMC}

\centering
\resizebox{\textwidth}{!}{
	\begin{tabular}{ |c|c|c|c| } 
	\hline
	Dataset         & Step 1 Generator & cross section (pb) & Size (fb$^{-1}$)   \\
		\hline
		Inclusive DY+jets, $DY \rightarrow ll$ & \MADGRAPH   & 5991    & 1.51 \\ \hline
		DY+jets HT 100-200, $DY \rightarrow ll$ & \MADGRAPH   & 181.3    & 15.0 \\ \hline
		DY+jets HT 200-400, $DY \rightarrow ll$ & \MADGRAPH   & 50.42    & 19.3 \\ \hline
		DY+jets HT 400-600, $DY \rightarrow ll$ & \MADGRAPH   & 6.984    & 153. \\ \hline
		DY+jets HT $>$ 600, $DY \rightarrow ll$ & \MADGRAPH   & 2.704    & 369. \\ \hline
		\ttbar+jets $\rightarrow ll$+jets & \MADGRAPH  & 85.67    & 286. \\ \hline
		single t $\rightarrow$ leptons+jets  & \POWHEG & 80.95 & 20.8 \\ \hline
		single $\bar{t}$ $\rightarrow$ leptons+jets & \POWHEG & 136.0 & 24.3 \\ \hline
		$\bar{t}$+W   & \POWHEG & 35.85 & 27.6 \\ \hline
		t+W   & \POWHEG & 35.85 & 27.8 \\ \hline
		WW  & \PYTHIA & 113.8   & 8.73   \\ \hline
		ZZ  & \PYTHIA & 10.15   & 98.2   \\ \hline
		WZ  & \PYTHIA & 23.4   & 41.8   \\ \hline
		W+jets $\rightarrow l\nu$+jets & \MADGRAPH & 50270   & 1.44 \\ \hline
		$\WR \rightarrow l\nul$  & \PYTHIA & 1$\times 10^{-5}$ - 4.3 & 5$\times 10^{6}$ - 11.6   \\ \hline
		\end{tabular}
}
\end{table}

In simulations of ST processes, different \MC generators were used in the first simulation step.  
The \DY (DY)+jets, W+jets, and t$\bar{t}$+jets simulations used the \MADGRAPH \cite{madgraph} generator, 
the single top and top+W simulations used the \POWHEG \cite{powheg} generator, and the diboson (WW, WZ, ZZ) 
simulations used the \PYTHIA \cite{pythia8,Sjostrand:2006za} generator.  In all of these 
simulations, \PYTHIA was used to hadronize free quarks and gluons into jets with the NNPDF23 PDF set 
\cite{nnpdf}.  Events used from these \MC datasets were required to pass at least one of the High 
Level triggers described earlier.

The \WR signal process was simulated using the \PYTHIA generator and NNPDF23 PDF set, and following 
the assumptions stated in Chapter \ref{sec:lrsPhenomenology}.  \WR signals in the $\mu\mu jj$ and $eejj$ 
final states were simulated independently, with \mWR increasing from 0.8 to 6 $\TeV$ in increments of 
0.2 \TeV, and $\mnul = \frac{1}{2}\mWR$.  Events from the $\WR \rightarrow \mu\mu jj$ samples were 
required to pass the single $\mu$, $\pt > 50 \GeV$ High Level trigger described previously, while 
$\WR \rightarrow eejj$ events were required to pass the double electron $\Et > 33 \GeV$ High Level 
trigger.

As stated previously, in the second step of \MC simulations, simulated events from randomly chosen ST 
processes were mixed into the main events being simulated.  For each event simulated for the main 
process (\WR or ST), the number of secondary events mixed in were pulled from a Poisson distribution 
with mean 12.  The secondary simulated events emulated multiple reconstructed interaction vertices, 
or pileup, in real data, but the PU distribution observed in data did not match that in simulated 
events.  To correct for the PU discrepancy between data and simulations, a PU dependent weight was 
applied to every simulated event.

Additional $\WR \rightarrow lljj$ simulated datasets using \PYTHIA were produced for limit setting.  
Only the first simulation step was run, and the \PYTHIA configuration was 
identical to what was used for the fully reconstructed \MC events.  Datasets were produced with 
$0.8 \leq \mWR \leq 4.0 \TeV$ increasing in \mWR increments of 0.1 $\TeV$, and $0.025 \leq \mnul < \mWR$ \TeV.  
Using a procedure described later, these datasets were used to extrapolate \WR cross section limits 
versus \mWR to \WR and \nul exclusion limits as functions of \mWR and \mnul.


\section{Offline Muon Reconstruction and Selection}
\label{sec:muonRecoAndSelection}
Muons were reconstructed using a combination of silicon tracker and muon detector information.  Muons 
that traversed the tracker layers were reconstructed as charged particles.  Their radii of curvature 
and measured rate of energy loss helped distinguish muons from other particles, and improved the 
resolution with which their momenta were measured.  Muons that reached the muon detectors were reconstructed 
as tracks, with opposite sign radii of curvature relative to the silicon tracker due to the magnetic 
field in the muon detectors.  Following track reconstruction in the silicon tracker and muon detectors, 
algorithms \cite{cmsMuonRecoRunOne} were used to identify muons as combinations of tracks in both detector 
systems that had similar trajectories.  The high-$\pt$ muons studied in this search were primarily 
reconstructed by starting with tracks in the muon detectors, and looking for matching tracks in the 
silicon tracker.

%One algorithm started with a seed track in the muon detectors, and 
%extrapolated the track back to the silicon tracker.  Then, tracks in the silicon tracker identified as 
%muon tracks were compared to the extrapolated muon detector track.  If a silicon tracker track had a 
%trajectory that matched the extrapolated muon detector track, the algorithm built a muon from the 
%combination of the two tracks.  This algorithm was developed to maximize reconstruction efficiency of 
%high $p_{T}$, and was complemented by a second algorithm that started with a seed track identified as a 
%muon in the silicon tracker.  The second algorithm extrapolated the seed track to the muon detectors in 
%the magnet return yoke, and compared the seed track to the muon detector tracks.  If a muon detector track 
%had a trajectory that matched the extrapolated seed track, the algorithm built a muon by combining the two 
%tracks.  The second algorithm was developed to maximize reconstruction efficiency of low $p_{T}$ muons, which 
%were not always able to traverse multiple layers of the iron return yoke to be detected in multiple 
%muon chambers.  In both algorithms, if multiple matching tracks were found, the track assigned to the muon 
%was the one whose trajectory was the closest match to the extrapolated track.

The momenta of reconstructed muons were determined using the "Tune-P" algorithm 
\cite{cmsMuonRecoRunTwo}, which was developed to improve high-$\pt$ muon momentum resolution.  For every muon, 
this algorithm simultaneously runs four algorithms that try to measure 
the momentum and fit a track to the entire reconstructed muon with the lowest $\chi^{2}/nDOF$.  All algorithms used all available 
information from the silicon tracker, and combined muon detector measurements and tracker measurements in 
different ways.  After all four algorithms were run, the "Tune-P" algorithm identified the muon momentum  
from the algorithm that fitted a global track to the muon detector and silicon tracker trajectories with the 
lowest $\chi^{2}/nDOF$ and $\sigma(\pt)/\pt$.  This procedure improved muon momentum resolution for $\pt > 200$ \GeV, 
which, as shown in Table \ref{tab:wrHighPtMuons}, were expected in a significant fraction of 
$\WR \rightarrow \mu\mu jj$ events.

\begin{table}[h]
	\caption{Fraction of expected $\WR \rightarrow \mu\mu jj$ events that had at least one muon with $\pt > 200$ \GeV. ($\mnul = \frac{1}{2}\mWR$)}
	\label{tab:wrHighPtMuons}
	\centering
	\begin{tabular}{c|c}
		\mWR (\TeV) & Fraction of events with at least one high-$\pt$ muons (\%) \\  \hline
		1.0 &  80.  \\
		2.0 &  95.  \\ 
		3.0 &  98.  \\ \hline
	\end{tabular}
\end{table}


%The first algorithm used all muon detector measurements, but gave greater weight to silicon 
%tracker measurements in the track fit and momentum measurement.  The second algorithm used equally weighted 
%tracker and muon detector measurements to fit a track, and used infrequent but distinct electromagnetic showers 
%in all muon chambers to determine the muon momentum.  The third algorithm only used measurements made in the 
%first layer of muon chambers closest to the magnet solenoid, where the muon has traversed the least amount of 
%material before being detected, in combination with the tracker measurements to fit a curve to the global muon 
%path, and determine the muon momentum.  The fourth algorithm dynamically truncated the length of the curve 
%fitted to the global muon path when a large energy loss caused the muon to dramatically change direction, 
%and as a result used as few as one layer of muon chambers to determine the trajectory and momentum of a muon.  
%After all four algorithms were run, the "Tune-P" algorithm assigned the momentum to the muon from the algorithm 
%that yielded a track fit with the lowest $\chi^{2}/nDOF$ and $\sigma(p_{T})/p_{T}$.  This procedure improved 
%the muon momentum resolution for $p_{T} > 200$ \GeV, which in general degrades as muon $p_{T}$ increases as 
%shown in Chapter \ref{sec:experiment_chapter}.

Following the momentum determination, muons were required to pass a $\pt$ selection, and 
identification selections to reject muons that were not produced by \WR decays.  These requirements were applied identically to events 
in data and simulations.  LRS models predicted $\mWR \gtrsim 1 \TeV$, and as a result the charged leptons 
produced by \WR decays had average $\pt$ in excess of 100 \GeV.  The kinematics of the expected final state leptons 
motivated a $\pt > 53 \GeV$ cut on muons, to increase sensitivity to \WR signals relative to ST backgrounds.  
As shown in Table \ref{tab:lowerMuonPtCuts}, the $\pt$ cut was not lowered to maintain good \WR sensitivity.  After the $\pt$ cut, muons 
were required to pass the following identification selections:

\begin{itemize}
	\item The track fit obtained from "Tune-P" included at least one muon chamber hit.
	\item The origin of the track fit obtained from "Tune-P" was within 2 (5) \mm of the primary vertex in the $x,y$ plane ($z$ axis).
	\item Muon detector tracks were reconstructed in at least two muon chambers.
	\item The relative $\pt$ error on the muon momentum from "Tune-P" was below 30\%, $\sigma(\pt)/\pt < 0.3$.
	\item The track obtained from the initial muon reconstruction had at least one hit in the silicon pixel detector, and had 
		hits in at least five layers in the entire tracker.
	\item In a $\Delta R = 0.3$ cone centered on the muon track in the silicon tracker, the $\Sigma \pt$ of all 
		tracks in the cone not matched with the muon must be low compared to the muon $\pt$, $\frac{\Sigma \pt}{muon \pt} < 0.1$.
\end{itemize}

\begin{table}[h]
	\caption{Signal/$\sqrt{Background}$ (S/$\sqrt{B}$) for $\mu$ $\pt$ 
	cuts using simulated \DY, t$\bar{t}$ and $\WR \rightarrow \mu\mu jj$ events with $\mWR = 2.2 \TeV$ and $\mnuR = 1.1 \TeV$.  
Lowering the $\mu$ $\pt$ cut worsened S/$\sqrt{B}$.}
	\label{tab:lowerMuonPtCuts}
	\centering
	\begin{tabular}{c|c}
		$\mu$ $p_{T}$ cut (\GeV) & S/$\sqrt{B}$ \\  \hline
		40 &  11.9  \\
		53 &  12.6  \\ \hline
	\end{tabular}
\end{table}

There were known differences between simulations and data that effected the reconstruction and selection efficiencies 
of muons.  These differences were corrected by applying event weights to muons in simulated events, and energy 
corrections to muons in data and simulated events.  In general the muon triggers were more efficient in simulations than in 
collisions, so the weight of every simulated event was scaled down by a factor dependent on the $\pt$ and $\eta$ of the 
triggering muon.  For the $\pt > 50 \GeV$ trigger used in the $\WR \rightarrow \mu\mu jj$ channel search, the 
event weight (a weight of 1 is equivalent to no correction) applied to triggering muons in simulated events was between 
0.95 (5\% decrease) and 0.97 (3\% decrease) for $\pt < 140 \GeV$, and between 0.97 (3\% decrease) and 1.04 (4\% increase) for $\pt > 140 \GeV$.  
The uncertainty on these weights was purely statistical, and the small impact of muon trigger weight uncertainties is discussed in 
Chapter \ref{sec:leptonRecoTriggerEffUnc}.  Based on the peak position and width of the dilepton mass distribution in $Z \rightarrow \mu\mu$ 
events in data and simulations, it was known that the muon energy scale and resolution differed between simulations 
and data\footnote{the peak position differed due to different muon energy scale calibrations, and the width differed 
because the muon energy resolution differed between simulations and data}.  The $\pt$ of muons in data and simulations 
were corrected based on their $\eta$, $\phi$, charge and initial $\pt$ to bring the muon energy scale and resolution 
into agreement between data and simulations.  These corrections were applied 
before the $\pt > 53 \GeV$ cut mentioned earlier.  Additional weights were applied to every simulated muon 
to eliminate the difference in muon identification selection and reconstruction efficiency between data and simulations.  
For all muons this weight was always between 1.0 (0\% change) and 0.985 (1.5\% decrease), and the effect of weight uncertainties 
are discussed in Chapter \ref{sec:uncertainties}.

%\begin{table}[htp]
%  \caption{The correction factor applied to the triggering muon, as a function of its $\pt$ and $\eta$, in simulated events.}
%  \label{tab:muTrgCorrs}
%  \centering
%  \begin{tabular}{ccccc}
%	  \hline
%	  $53 < \pt < 140$ & & & & \\
%	  $|\eta|$         & $< 0.9$ & 0.9 to 1.2 & 1.2 to 2.1 & 2.1 to 2.4 \\
%	  \MC correction  & $0.971$ & $0.964$ & $0.961$ & $0.951$  \\
%	  \MC correction uncertainty & $0.00079$ & $0.0021$ & $0.0013$ & $0.0043$  \\
%	  \hline
%	  $140 < \pt$ & & & & \\
%	  $|\eta|$         & $< 0.9$ & 0.9 to 1.2 & 1.2 to 2.1 & 2.1 to 2.4 \\
%	  \MC correction  & $0.971$ & $0.999$ & $0.999$ & $1.04$  \\
%	  \MC correction uncertainty & $0.0076$ & $0.025$ & $0.015$ & $0.051$  \\
%	  \hline
%  \end{tabular}
%\end{table}


\section{Offline Electron Reconstruction and Selection}
\label{sec:electronRecoAndSelection}
Electron reconstruction started with the reconstruction of ECAL energy clusters.  Electrons that impinged 
on the ECAL showered and lost substantially all of their energy in 5 $\times$ 5 crystal, and these 25 crystals 
were reconstructed as individual superclusters (SCs).  In tandem with ECAL SC reconstruction, electron tracks 
were reconstructed from silicon tracker measurements using a dedicated electron algorithm.  On average an electron 
lost 33\% of its initial energy to bremsstrahlung in the tracker, and the electron track reconstruction algorithm 
estimated this energy loss for tracks indicative of electrons.  Tracks were subsequently extrapolated to the 
ECAL, and electrons were identified as tracks that geometrically matched ECAL SCs.  The electron energy was 
taken from the SC, but its $\eta$ and $\phi$ were obtained from the track.
%The SC reconstruction algorithm was optimized to detect real electron energy deposits, typically two crystals 
%wide or smaller in $\eta$, but large in $\phi$ due to electron bremsstrahlung in the magnetic field and energy 
%lost to interactions with the tracker, with high efficiency.  
%Independent of SC reconstruction, electron 
%tracks were reconstructed from silicon tracker measurements using a specialized electron track reconstruction 
%algorithm.  On average a relativistic electron that traverses the entire silicon tracker will lose 33\% of 
%its initial energy to bremsstrahlung in the tracker hardware.  The dedicated electron track reconstruction 
%algorithm took this significant average energy loss into consideration when building tracks from tracker 
%measurements.  Following track and SC reconstruction, tracks were extrapolated to the ECAL, and each electron 
%was formed from one SC and one or three tracks, whose extrapolated positions matched the ECAL SC.  Three track 
%combinations were allowed because a hard scatter between an electron and the tracker could produce an $e^{+}e^{-}$ 
%pair in the tracker with sufficient energy to reach the ECAL.  The energy of each electron was taken from 
%the ECAL, but its $\eta$ and $\phi$ were obtained from the reconstructed track.

Following electron reconstruction and energy measurement, electrons were required to pass $\Et$ and identification 
selections.  These requirements were applied identically to events in data and simulations.  
Similar to muon $\pt$s, electrons were required to have $\Et > 53 \GeV$ to increase sensitivity to a \WR signal.  
After the $\Et$ cut, electrons were required to pass the following identification selections:

\begin{itemize}
	\item The extrapolated track position and SC seed crystal position differed by less than 1 crystal width in $\eta$, and 
		about 3 crystal widths in $\phi$.
	\item For endcap electrons ($1.567 < |\eta| < 2.5$), at least 90\% of the ECAL SC energy was measured in a 2 crystal wide 
		region in $\eta$.
	\item For barrel electrons ($|\eta| < 1.44$), at least 94\% (83\%) of the SC energy was measured in a 2 (1) crystal wide 
		region in $\eta$.
	\item The ratio of hadronic energy in the HCAL tower behind the SC to the SC energy was low, $\frac{E_{HCAL}}{E_{SC}} < 0.05 + 1/E_{SC}$ in the barrel, $< 0.05 + 5/E_{SC}$ in the endcap.
	\item In a cone of radius $\Delta R =$ 0.3 centered on the electron:
	\begin{itemize}
		\item The $\Sigma \pt$ of all tracks excluding the electron track was low, $\Sigma \pt < 5$ \GeV.
		\item The calorimeter energy $E_{ECAL + HCAL}$ in the cone not associated with the electron was low, 
			$E_{ECAL + HCAL} < 2 + 0.03\alpha + 0.28\rho$, where $\rho =$ the neutral hadron energy per unit $\eta,\phi$ area, 
			and in the barrel $\alpha = \Et$ of the electron, and in the endcap $= \Et - 50$ of the electron.
	\end{itemize}
	\item The number of silicon pixel or inner silicon strip layers that were not hit by the electron track was one or less. 
	\item The distance $\Delta_{xy}$ between the electron track origin and primary vertex in the $x$, $y$ plane was small, 
		$\Delta_{xy} < 0.2$ \mm in the barrel tracker, or $\Delta_{xy} < 0.5$ \mm in the endcap tracker.
\end{itemize}

There were known differences between simulations and data that effected the electron reconstruction and selection 
efficiencies.  These differences were corrected by applying event weights to electrons in simulated events, and energy 
corrections to electrons in data and simulated events.  The double electron $\Et > 33 \GeV$ trigger had the same 
efficiency in data and simulations for reconstructed electrons with $\Et > 53$ \GeV, so no trigger correction 
was applied.  Based on the peak position and width of the dilepton mass distribution in $Z \rightarrow ee$ 
events in data and simulations, it was known that the electron energy scale and resolution differed between simulations 
and data.  The $\Et$ of electrons in data and simulations were corrected based on their $\eta$, $\phi$, initial $\Et$ 
and ECAL shower shape to bring the electron energy scale and resolution in data and simulations into agreement.  
These corrections were applied before the $\Et > 53 \GeV$ cut mentioned earlier.  Additional weights were applied 
to every simulated electron to eliminate the difference in electron reconstruction and identification selection 
efficiencies between data and simulations.  A weight of 0.982 (1.8\% decrease) was applied to all simulated electrons to account for 
differences in reconstruction efficiency, and an additional weight of 0.989 (1.1\% decrease) was applied to all simulated electrons to 
account for differences in identification selection efficiency.  The effect of uncertainties on these weights was 
small, and is discussed in Chapter \ref{sec:uncertainties}.


\section{Offline Jet Reconstruction and Selection}
\label{sec:jetRecoAndSelection}
Jets produced in proton-proton collisions contained long lived charged and neutral hadrons, photons from 
$\pi^{0} \rightarrow \gamma\gamma$ decays, and charged leptons from heavy quark decays.  Jets reconstruction began 
by reconstructing electrons, muons, photons, and charged and neutral hadrons.  Photons were reconstructed using the same algorithm 
used to reconstruct electrons, excluding matching ECAL SCs to reconstructed tracks.  On average, a hadron that 
impinged on the ECAL had one significant interaction with the PbWO$_{4}$ before entering 
the HCAL.  As a result, reconstructing charged and neutral hadrons started with the ECAL.  A modified version of 
the electron SC reconstruction algorithm, with looser requirements on the $\eta$ distribution of energy 
in the SC, was used to reconstruct hadronic energy clusters in the ECAL.  Then, energy deposited in individual 
HCAL towers was reconstructed into tower clusters (TC).  Independently, tracks not consistent with muons or 
electrons were reconstructed using silicon tracker measurements.  Every neutral hadron was built from an HCAL TC, and 
an ECAL SC if a geometric match was found.  Hadron tracks were extrapolated from the tracker to the HCAL, and 
each charged hadron was built from a TC and geometrically matching track, and an ECAL SC if a geometric match 
with the track was found.  After all particles were reconstructed, jets were reconstructed starting with 
charged particle tracks.  In each event, charged particles that traced back to the interaction 
vertex with the highest track $\Sigma \pt$ (the primary vertex) were used to tag different jets and their 
directions.  Then, all charged particles from the primary vertex, and all photons and neutral hadrons in the event 
were placed in a list of jet constituent candidates, which was subsequently analyzed using the anti-$k_{T}$ 
algorithm \cite{antikt}.  Jet constituents were iteratively clustered into jets based on the $\pt$ of each 
constituent and its $(\eta,\phi)$ distance from the jet axis defined by a charged particle track.  The majority 
of jet constituents fell within $\Delta R < 0.4$ of the jet axis, but it was possible for low energy constituents to 
be futher away.  A jet's energy was defined as the total energy of its constituents, and its $(\eta,\phi)$ 
trajectory was the $\pt$-averaged trajectory of all charged constituents.

%This algorithm took the $i^{th}$ particle from the list of jet constituents, and
%calculated the distance parameter $d_{ij}$
%
%\begin{equation}
%	$d_{ij} = min(k_{Ti}^{-2},k_{Tj}^{-2})\frac{\Delta^{2}}{R^{2}}$
%\end{equation}
%
%using the $j^{th}$ particle in the same list, where $\Delta$ is the angular separation between the $i^{th}$ 
%and $j^{th}$ particle, R is 0.4, and $k_{Ti}$ is the $p_{T}$ of the $i^{th}$ particle.  If the $i^{th}$ particle 
%had $d_{ij} > k_{Ti}^{-2}$ for all j, then the $i^{th}$ particle was a jet constituent and was removed from the 
%constituent list and placed in a new list $L_{1}$.  This procedure was repeated for all particles in the list of 
%jet constituents, then repeated for all elements in the new list $L_{1}$.  The iteration over all particles in 
%the newly created list continued N times, creating N lists, until the lists $L_{N}$ and $L_{N-1}$ had the same 
%number of elements.  At the end, jets were produced from clusters of individually reconstructed particles.

Once the jets were clustered, their raw energies were corrected in several steps.  
Neutral hadrons produced by pileup interactions can still be clustered into jets, so the first correction reduces 
the energy of every jet in a data or simulations based on the jet area and the total neutral hadron energy density 
in the event \cite{pileup1,pileup2}.  Subsequently, energy corrections based on jet $\eta$ and $\pt$ 
were applied.  These corrections, derived from \MC and applied to jets in data and simulated events, accomplished 
two tasks.  First, the $\pt$ of reconstructed jets in simulated events was brought into better agreement with the 
true (quark, gluon, photon and lepton) jet $\pt$.  Second, the response of the calorimeters to reconstructed jet 
constituents in simulations and data were brought into better agreement.  More details on jet energy corrections, 
such as the types of \MC events used to derive the corrections, can be found elsewhere \cite{jetpaper}.

After jets in data and simulation were reconstructed and corrected, jet identification and energy selections 
were applied to increase sensitivity to jets coming from \WR decays.  The identification criteria required each 
jet to have:

\begin{itemize}
	\item $|\eta| \leq 2.4$
	\item less than 90\% of its energy came from neutral hadrons
	\item less than 90\% of its energy came from photons
	\item at least 2 constituents (from reconstructed charged or neutral particles)
	\item more than 0\% of its energy came from charged hadrons
	\item at least one constituent was a charged hadron
	\item less than 99\% of its energy came from electrons
\end{itemize}

These requirements increased the fraction of selected jets that originated from real hadronic activity expected in the 
$\WR \rightarrow \ell\ell jj$ decay.  As shown in Table \ref{tab:wrHighPtJets}, LRS models predicted that real jets 
produced in \WR decays had high $\pt$.  Jets were required to have $\pt > 40 \GeV$ to increase sensitivity to 
a \WR signal relative to expected ST backgrounds.  As shown in Table \ref{tab:lowerJetPtCuts}, a lower jet $\pt$ cut 
increased ST backgrounds without a corresponding increase in \WR signal.

\begin{table}[h]
	\caption{Fraction of expected $\WR \rightarrow \ell\ell jj$ events that had two jets with $\pt > 40$ \GeV. ($\mnul = \frac{1}{2}\mWR$)}
	\label{tab:wrHighPtJets}
	\centering
	\begin{tabular}{c|c}
		\mWR (\TeV) & Fraction of events with two high-$\pt$ jets (\%) \\  \hline
		1.0 &  80.  \\
		2.0 &  92.  \\
		3.0 &  94.  \\ \hline
	\end{tabular}
\end{table}


\begin{table}[h]
	\caption{Signal/$\sqrt{Background}$ (S/$\sqrt{B}$) for subleading jet $\pt$ 
		cuts using simulated \DY, t$\bar{t}$ and $\WR \rightarrow \mu\mu jj$ events 
	with $\mWR = 2.2 \TeV$ and $\mnuR = 1.1 \TeV$.  Lowering the cut would worsen S/$\sqrt{B}$.}
	\label{tab:lowerJetPtCuts}
	\centering
	\begin{tabular}{c|c}
		jet $p_{T}$ cut (\GeV) & S/$\sqrt{B}$ \\  \hline
		30 &  12.1  \\
		40 &  12.6  \\ \hline
	\end{tabular}
\end{table}


\section{\WR Candidate Selection}
\label{sec:wrCandSelection}
Following the reconstruction and selection of electrons, muons and jets, a \WR candidate in every $\mu\mu$ (ee) 
channel event was built from the two highest $\pt$ ($\Et$) muons (electrons), and the two 
highest $\pt$ jets passing the $\pt$ or $\Et$ selections described earlier.  In the $e\mu$ channel used for top quark background estimation, the highest $\pt$ muon and highest $\Et$ 
electron were used to build an $e\mu jj$ candidate.  LRS models predicted one lepton from \WR decays to have 
higher energy than the other, so ,to increase sensitivity to a \WR signal, one lepton was required to have $\pt$ or 
$\Et > 60$ \GeV.  Furthermore, due to the heavy $\nul$ recoiling off the $\ell$ daughter of the \WR, it was predicted that 
the two leptons in \WR decays would have a high dilepton mass (Table \ref{tab:wrMll}).  To further increase sensitivity to a \WR signal over expected 
backgrounds, especially \DY+jets, the two selected leptons were required to have mass $\Mll > 200$ \GeV.  In events 
that passed the high $\pt$ lepton and $\Mll$ selections, one or both leptons could have come from jets with $\pt \gg 100 \GeV$ 
that produced an energetic lepton in the tracker.  To reduce the fraction of such events passing the kinematic 
selection, the two selected jets were required to be separated from the two selected leptons by $\Delta R > 0.4$:

\begin{itemize}
	\item $\Delta R(\ell_{1}, j_{1}) > 0.4$
	\item $\Delta R(\ell_{1}, j_{2}) > 0.4$
	\item $\Delta R(\ell_{2}, j_{1}) > 0.4$
	\item $\Delta R(\ell_{2}, j_{2}) > 0.4$
\end{itemize}

\begin{table}[h]
	\caption{Fraction of expected $\WR \rightarrow \ell\ell jj$ events with dilepton mass $\Mll > 200$ \GeV. ($\mnul = \frac{1}{2}\mWR$)}
	\label{tab:wrMll}
	\centering
	\begin{tabular}{c|c}
		\mWR (\TeV) & Fraction of events with high \Mll (\%) \\  \hline
		1.0 &  88.  \\
		2.0 &  98.  \\
		3.0 &  99.  \\ \hline
	\end{tabular}
\end{table}


As there were only charged leptons and jets in the final states studied in this search, the dilepton plus dijet mass 
of the four final state particles (\Mlljj) averaged over many events was consistent with \mWR predicted by 
LRS models.  To further increase sensitivity to a \WR signal relative to expected backgrounds, an effective \WR mass 
cut was applied by requiring the final state lepton plus jet mass $\Mlljj > 600$ \GeV.  Nearly all events from \WR \MC 
datasets passed this cut with high efficiency, as shown in Table \ref{tab:wrMlljj}.

The full trigger and kinematic selection efficiency in fully reconstructed \WR signal \MC events exceeded 50\% (70\%) 
in the ee ($\mu\mu$) channel, as shown in Figure \ref{fig:wrRecoSelectionEff}.  The ee channel had lower efficiency due 
to the gap in electron detection for $1.44 < |\eta| < 1.57$, and tighter identification requirements for electrons 
relative to muons.

\begin{table}[h]
	\caption{Fraction of expected $\WR \rightarrow \ell\ell jj$ events with dilepton plus dijet mass $\Mlljj > 600$ \GeV. ($\mnul = \frac{1}{2}\mWR$)}
	\label{tab:wrMlljj}
	\centering
	\begin{tabular}{c|c}
		\mWR (\TeV) & Fraction of events with high \Mlljj (\%) \\  \hline
		1.0 &  99.  \\
		2.0 &  100.  \\
		3.0 &  100.  \\ \hline
	\end{tabular}
\end{table}


\begin{figure}[h]
	\centering
	\includegraphics[width=1.0\textwidth]{figures/wrRecoSelectionEfficiency.png}
	\caption{The efficiency that $\WR \rightarrow \ell\nul \rightarrow \ell\ell jj$ events are selected using 
	the trigger and full offline selection in the ee ($\mu\mu$) channel on the left (right).  Different curves represent 
events where both leptons were in the barrel $\eta$ region (BB), one was in the endcap (EB), or both were in the endcap (EE).}
	\label{fig:wrRecoSelectionEff}
\end{figure}

LRS models predicted a signal from a \WR boson and heavy neutrino \nul manifested as an excess of events relative 
to expected backgrounds in distributions of several lepton and jet kinematic variables, like $\pt$ and multi-particle 
invariant masses.  Models differed in the shapes and magnitudes of predicted excesses, but all agreed on one point: 
a \WR boson existed, and evidence of it appeared as a peak in the \Mlljj distribution, whose peak position was consistent 
with the predicted \mWR.  To look for evidence in support of LRS models without being sensitive to nuances of different 
models, the \Mlljj mass was used as the search variable of merit.  Shown in Figure \ref{fig:mlljjVariableOfMerit} was 
proof of this variable's efficacy: a simulated \WR signal with $\mWR = 1.0, \mnul = 0.5 \TeV$ appeared as a sharp peak 
in the \Mlljj distribution centered near $\Mlljj = 1.0 \TeV$ that was clearly differentiated from expected ST backgrounds.

\begin{figure}[h]
	\centering
	\includegraphics[width=0.7\textwidth]{figures/useOfLLJJMassAsFigureOfMerit.pdf}
	\caption{The $\Mlljj$ distribution from simulations of expected ST backgrounds and \WR signal in the ee channel.  
		The normalization of the \WR signal distribution is scaled down by 70\% only to better visualize the difference 
	between \WR signal and expected backgrounds.}
	\label{fig:mlljjVariableOfMerit}
\end{figure}



%%%%%%%%%%%%%%%%%%%%%%%%%%%%%%%%%%%%%%%%%%%%%%%%%%%%%%%%%%%%%%%%%%%%%%%%%%%%%%%%
