%%%%%%%%%%%%%%%%%%%%%%%%%%%%%%%%%%%%%%%%%%%%%%%%%%%%%%%%%%%%%%%%%%%%%%%%%%%%%%%%
% event_selection.tex: 
%%%%%%%%%%%%%%%%%%%%%%%%%%%%%%%%%%%%%%%%%%%%%%%%%%%%%%%%%%%%%%%%%%%%%%%%%%%%%%%%
\chapter{Event Selection}
\label{event_selection_chapter}
%%%%%%%%%%%%%%%%%%%%%%%%%%%%%%%%%%%%%%%%%%%%%%%%%%%%%%%%%%%%%%%%%%%%%%%%%%%%%%%%

By studying events with two jets and two same flavor leptons (e,$\mu$), this search
seeks evidence of potential \WR signals which decay via $\WR \rightarrow l\Nell \rightarrow lljj$.
Events with e$\mu$jj final state discussed here are only used for top quark background
estimation.  This chapter describes the procedures through which events are selected, and the
subsequent reconstruction of and selection applied to jets, muons and electrons, and
combinations thereof.  This chapter concludes by explaining the methodology used
to interpret the four-object invariant mass of the two same flavor leptons and two jets in the
context of different \WR mass hypotheses.

\section{Data and Monte Carlo}

\subsection{Data}
\label{data}
The data used by this analysis was collected by the CMS experiment from May until December 2015, at
the center of mass energy of $\sqrt{s} = 13\TeV$.  As this was the first year of
collisions at $\sqrt{s} = 13\TeV$, the LHC cautiously operated at a much lower average
instantaneous luminosity than in 2012, and spent the first half of the data-taking period
using 50ns spacing between proton bunches, as was done for all of 2012.  During the second
half of the data-taking period, the LHC collided proton bunches using 25ns spacing between
bunches, and delivered approximately YYY fb$^{-1}$, as shown in
Figure \ref{fig:lhc2015IntegLumi}.  The data was split into four run eras -
Run2015A, B, C and D - which can be identified in Figure \ref{fig:lhc2015IntegLumi} as
the periods between plateaus in integrated luminosity.  Each run era corresponds
to a period in which all LHC fills used similar spacing between individual proton bunches, and
the structure of each beam, in terms of how large groups of bunches were injected, was
relatively constant.  The plateaus separating run eras were periods when the
LHC stopped physics collisions for maintenance or minor upgrades to increase instantaneous
luminosity.  CMS collected data during all four run eras, and the collision datasets
are named accordingly.  Run eras A and B correspond to collisions with 50ns spacing between
proton bunches, and are not used in this analysis.  Problems with the CMS magnet cooling system reduced the
amount of data available for physics analyses from YYY fb$^{-1}$ to 2.6 fb$^{-1}$.

The raw dataset collected by CMS is too large ($\gtrsim 10^{4}$ terabytes) for analyses
which want to run from start to finish in $\lessim 3$ days, and contains more information
than what is needed by any individual physics analysis.  To expedite the process of
transforming collision data into a public physics result, collision data from each run
era is split into several smaller datasets which are distinguished by the presence of one
or more objects in the final state, such as events with at least one muon, or least two
electrons.  The HLT decides which dataset an event should be assigned to based on the
individual triggers which were fired; in some instances one event can be assigned to several
datasets.  As the average instantaneous luminosity of the LHC increased dramatically in the 
last 3 months of data-taking, datasets in Run2015D were split into two pieces such that the size
of each dataset remained small ($\sim 1$ terabyte).  The collision events used in this analysis
were collected in the "SingleMuon", "DoubleEG", and "MuonEG" datasets summarized in Table
\ref{tab:collisionDatasets}.  This analysis uses the datasets which were reconstructed, the process
by which raw detector outputs are transformed into distinguishable objects like muons and
electrons, in the late summer, fall and early winter of 2015.

\begin{table}[h]
\caption{The collision datasets used in this analysis, their total size (all dataset pieces for 2015D) and the run era in which they were made.}
\label{tab:collisionDatasets}
\centering
\begin{tabular}{c|c|c|c|c}
Run Era & Int. Lumi (pb$^{-1}$) & eejj dataset & $\mu\mu$jj dataset & e$\mu$jj dataset \\  \hline
	2015C &  CLUMI  &  DoubleEG  &  SingleMuon  &  MuonEG  \\
	2015D &  DLUMI  &  DoubleEG  &  SingleMuon  &  MuonEG  \\ \hline
\end{tabular}
\end{table}



\subsection{Monte Carlo}
\label{MC}


%%%%%%%%%%%%%%%%%%%%%%%%%%%%%%%%%%%%%%%%%%%%%%%%%%%%%%%%%%%%%%%%%%%%%%%%%%%%%%%%
