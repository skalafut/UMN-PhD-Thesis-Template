%%%%%%%%%%%%%%%%%%%%%%%%%%%%%%%%%%%%%%%%%%%%%%%%%%%%%%%%%%%%%%%%%%%%%%%%%%%%%%%%
% event_selection.tex: Select of showering and tracking events:
%%%%%%%%%%%%%%%%%%%%%%%%%%%%%%%%%%%%%%%%%%%%%%%%%%%%%%%%%%%%%%%%%%%%%%%%%%%%%%%%
\chapter{Problem 2}
\label{Problem 2}
%%%%%%%%%%%%%%%%%%%%%%%%%%%%%%%%%%%%%%%%%%%%%%%%%%%%%%%%%%%%%%%%%%%%%%%%%%%%%%%%
the ramp is inclined at an angle $\theta$ above horizontal\newline
define the X axis as parallel to the surface of the ramp\newline
and positive X points up the ramp\newline
define the Y axis as perpendicular to the ramp surface\newline
and negative Y points into the ramp\newline

a. max static coefficient = $1.37$\newline
first find the normal force by using the fact that\newline
the net force in the Y direction is zero\newline
the applied force $F_{app} = 200$ Newtons
0 = $F_{normal} - mg\cos(\theta) - F_{app}\sin(\theta)$\newline
so $F_{normal} = mg\cos(\theta) + F_{app}\sin(\theta)$\newline

the maximum static coefficient of friction is\newline
found when the force of friction plus gravity pulling\newline
the box down the ramp is equal to the component of the\newline
applied force parallel to the ramp surface\newline
0 = $F_{app}\cos(\theta) - mg\sin(\theta) - \mu F_{normal}$\newline
0 = $F_{app}\cos(\theta) - mg\sin(\theta) - \mu(mg\cos(\theta) + F_{app}\sin(\theta))$\newline
rearranging for $\mu$ yields\newline
$\mu = \frac{F_{app}\cos(\theta) - mg\sin(\theta)}{mg\cos(\theta) + F_{app}\sin(\theta)}$\newline
$\theta = 10$ degrees  $m = 10.0$ kg\newline

b. $accel_{x} = 15.4 \frac{m}{s^{2}}$ \newline
once motion begins $\mu = 0.20$\newline
the net force in the X direction is given by\newline
$maccel_{x} = F_{app}\cos(\theta) - mg\sin(\theta) - \mu(mg\cos(\theta) + F_{app}\sin(\theta))$\newline
$accel_{x} = \frac{F_{app}\cos(\theta) - mg\sin(\theta) - \mu(mg\cos(\theta) + F_{app}\sin(\theta))}{m}$\newline

c. $v_{fx} = \sqrt{2accel_{x}(5.00 meters)} = 12.4 \frac{m}{s}$ up the ramp\newline

d. $985$ Joules\newline
work W = applied force F times the distance traveled\newline
the only motion is $5.00$ meters along the X axis\newline
W = $F_{app}\cos(10)(5.00 m) = 985$ Joules\newline

%%%%%%%%%%%%%%%%%%%%%%%%%%%%%%%%%%%%%%%%%%%%%%%%%%%%%%%%%%%%%%%%%%%%%%%%%%%%%%%%
