%%%%%%%%%%%%%%%%%%%%%%%%%%%%%%%%%%%%%%%%%%%%%%%%%%%%%%%%%%%%%%%%%%%%%%%%%%%%%%%%
% experiment.tex: Chapter describing the experiment
%%%%%%%%%%%%%%%%%%%%%%%%%%%%%%%%%%%%%%%%%%%%%%%%%%%%%%%%%%%%%%%%%%%%%%%%%%%%%%%%
\chapter{Experiment}
\label{experiment_chapter}
%%%%%%%%%%%%%%%%%%%%%%%%%%%%%%%%%%%%%%%%%%%%%%%%%%%%%%%%%%%%%%%%%%%%%%%%%%%%%%%%
\subsection{The Large Hadron Collider}
\begin{itemize}
	\item sqrt s, inst lumi
	\item 27 km circumference
	\item need high energy and inst lumi to search for weakly coupled interactions with heavy mediating particles
	\item high energy and lumi drive dramatically increase rate of low energy and other abundant processes. explain these
	\item QCD multijet event rate at LHC is O(100 kHz), CMS readout capability is O(100 Hz) $\rightarrow$ need fast, efficient trigger system
	\item describe 2015 run conditions - cause of PU, PU range observed in data, consequences on trigger, offline reco and analysis
\end{itemize}
Situated near Geneva Switzerland, the Large Hadron Collider (LHC) is the highest energy particle
collider in the world.  Operated by the European Organization for Nuclear Research (CERN), the LHC resides in
a 27 km circumference tunnel previously used for the Large Electron Positron Collider (LEP).  Collisions between
protons in the LHC begin with low energy hydrogen atoms.  Hydrogen gas is heated into a plasma from which protons are
extracted.  Protons then travel through several accelerator stages, gaining energy after every stage, and end in
the LHC.  In the LHC, strong magnetic and electric fields created by superconducting magnets and RF cavities 
create two collimated beams of high energy protons which collide at four interaction points.  The LHC was
designed to accelerate both beams of protons such that collisions occur at a center of mass energy of 14 TeV,
initially at an instantaneous luminosity of $2x10^{33} \frac{1}{cm^{2}s}$.  After a few years of successful running
the instantaneous luminosity would increase to $1x10^{34} \frac{1}{cm^{2}s}$.
After the LHC startup with collisions at a lower center of mass energy and lower instantaneous luminosity than design, the center of mass
energy was increased to 13 TeV for the 2015 data taking period.



\subsection{Compact Muon Solenoid Detector Overview}
\begin{itemize}
	\item general design goals which ultimately focus down to WR search
	\item brief description of each subdetector and magnet
	\item trigger description, followed by particle flow event reconstruction
\end{itemize}

\subsection{Track and vertex reconstruction}

\subsection{Muon Reconstruction and Identification}
\begin{itemize}
	\item importance of muons, challenges with reco and energy measurements
	\item muon detector technologies
	\item muon reconstruction algorithms
	\item challenge with pT measurement of high pT muons
	\item explain TuneP algorithm and show performance plots
	\item explain ID variables, isHighPt ID and show performance plots
	\item explain usefulness of muon isolation (reject muons from QCD) and how it is calculated
	\item muon L1 and HLT: design and performance plots
\end{itemize}

\subsection{Electron Reconstruction and Identification}
\begin{itemize}
	\item importance of electrons, challenges with reco and energy measurements
	\item ECAL
	\item electron reconstruction algorithms
	\item challenge with pT measurement of high pT muons
	\item explain TuneP algorithm and show performance plots
	\item explain ID variables, HEEP ID and show performance plots
	\item explain usefulness of electron isolation (reject eles from QCD) and how it is calculated
	\item electron L1 and HLT: design and performance plots
\end{itemize}



\subsection{Particle Flow and Jet Reconstruction}



