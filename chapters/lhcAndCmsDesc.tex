%%%%%%%%%%%%%%%%%%%%%%%%%%%%%%%%%%%%%%%%%%%%%%%%%%%%%%%%%%%%%%%%%%%%%%%%%%%%%%%%
% experiment.tex: Chapter describing the experiment
%%%%%%%%%%%%%%%%%%%%%%%%%%%%%%%%%%%%%%%%%%%%%%%%%%%%%%%%%%%%%%%%%%%%%%%%%%%%%%%%
\chapter{Experiment}
\label{experiment_chapter}
%%%%%%%%%%%%%%%%%%%%%%%%%%%%%%%%%%%%%%%%%%%%%%%%%%%%%%%%%%%%%%%%%%%%%%%%%%%%%%%%
%\subsection{Overview of the Compact Muon Solenoid and Large Hadron Collider}
The Compact Muon Solenoid (CMS) experiment was built to detect all Standard Model (SM) particles, and to
search for any non-SM interactions that produce SM particles, such as the interaction described in 
chapter ~\ref{wrBosonAndHeavyNu}.  Two high-energy proton beams collided at the geometric center of
the CMS experiment, and interactions between colliding protons were recorded.

Interactions between protons were identified by the detection of:
\begin{itemize}
	\item one or more energetic charged leptons (electron, muon, tau or their anti-particles)
	\item one or more energetic photons
	\item one or more energetic hadronic jets
	\item one or more energetic neutrinos\footnote{the presence of neutrinos is inferred from the vector sum of momentum of all charged leptons, photons and hadronic jets detected in a collision event.}
	\item combinations thereof
\end{itemize}
Particles produced by proton-proton (pp) interactions were detected using a 3.8$\unit{T}$ magnet and four CMS 
sub-detector systems.  The full strength magnetic field volume enveloped the nominal point where pp interactions occurred, and three of the four 
sub-detector systems used in particle detection.  As shown in Figure \ref{fig:layersOfCMS}, surrounding the pp interaction
point was the silicon tracker in which charged particles were first detected.  As charged particles traversed
the tracker, mobile electron-hole pairs were ionized in several layers of silicon along the trajectories of charged
particles.  Charged particles were detected in the tracker by measuring the charge ionized in all silicon layers.

\begin{figure}[h]
	\centering
	\includegraphics[width=0.6\textwidth]{figures/missingImage.png}
	\caption{Cross section view of the entire CMS detector in a plane perpendicular to the beam axis.  Closest to the
	proton-proton interaction point is the silicon tracker, followed by the Electromagnetic Calorimeter, then the Hadronic 
Calorimeter, and the magnet.  Outside the magnet are the muon detectors.}
	\label{fig:layersOfCMS}
\end{figure}


Surrounding the silicon tracker was the second of three sub-detector systems enclosed in the full strength magnetic field, the
lead-tungstate crystal (PbWO$_{4}$) Electromagnetic Calorimeter (ECAL).  The ECAL was an absorption calorimeter that
detected photons produced by pp interactions, and distinguished electrons and positrons from other charged 
particles detected by the tracker.  Electrons, positrons and photons interacted with lead-tungstate nuclei 
through bremsstrahlung and electron-positron pair production processes, and produced showers of lower energy
electrons and positrons.  Through Coulomb interactions, these lower energy particles excited atomic electrons
of lead-tungstate nuclei to higher energy, meta-stable bound states.  When these atomic electrons transitioned
back to the ground state, visible light photons were produced.  High energy photons, electrons and positrons 
produced by pp interactions were detected by measuring the visible light produced in lead-tungstate crystals.

Wrapped around the ECAL was the last of three sub-detector systems enclosed in the full strength magnetic field, the brass and 
plastic scintillator Hadronic Calorimeter (HCAL).  Built from alternating layers of brass absorber and scintillating
plastic tiles, the HCAL was a sampling calorimeter that detected charged and neutral hadrons.  Charged and 
neutral hadrons that travelled through the tracker and the ECAL interacted with bound nucleons in brass absorber layers
through diffractive and inelastic scattering processes.  Scattering events produced showers of lower energy hadrons, which
travel through nearby scintillating plastic tiles.  Interactions between low energy hadrons and molecules in
the plastic excited the molecules to meta-stable states.  When excited molecules transitioned back to the ground 
state, visible light photons were emitted.  High energy charged and neutral hadrons produced by pp interactions 
were detected by measuring the visible light produced in scintillating plastic tiles.

Outside the HCAL was the 3.8$\unit{T}$ magnet.  The magnetic field was generated by running a current of 
$\lesssim$18 thousand amps through superconducting wire wound into a solenoid.  The wire, made from a niobium alloy, was 
cooled to a superconducting state using a liquid helium adsorption refrigerator and liquid nitrogen precooling 
system.  The cryogenic system and magnet were supported by a separate structure built from iron, steel and titanium.  The
large volume of material linked to the magnet and support systems provided a barrier through which only muons 
could penetrate.

Surrounding the magnet was the fourth sub-detector system used to detect particles, the muon detectors, which resided 
in the magnet return yoke where the magnetic field strength was 1$\unit{T}$.  Three 
different technologies - drift tubes (DT) in the lowest radiation regions, cathode strip chambers (CSC) in higher 
radiation regions, and resistive plate chambers (RPC) in all regions - were used to detect muons.  In detectors of all 
three technologies, muons travelled through pressurized gas chambers and ionized electrons from the gas along their 
trajectories.  Within the chambers, electric fields accelerated the ionized electrons towards conducting wires.  Muons 
were detected by measuring the charge collected by conducting wires.

Due to the enormous rate of pp collisions in which only elastic or diffractive scattering occurred, a two tiered trigger 
system was used to select events where energetic charged leptons, hadronic jets, photons, neutrinos or combinations 
thereof were detected.  The first tier of the trigger system processed information from the ECAL, the HCAL and the muon 
detectors in small regions where high energy particles were detected.  Regions in which the detected energy exceeded 
a preset threshold were then processed by the second tier of the trigger system.  In the second tier, reconstruction software 
was run in selected regions to identify charged leptons, photons, jets and neutrinos.  If a pp collision event produced one 
or more of these reconstructed particles that passed quality and energy cuts defined in the second tier, then all data 
from all four sub-detector systems for that event was read out and written to disk.

The accelerator which collided the proton beams, the coordinate system used to characterize the kinematics of detected 
particles, the silicon tracker, both calorimeters, the muon detectors and the trigger system are discussed in this chapter.

\section{The Large Hadron Collider}
\label{sec:lhcDescription}
Situated near Geneva, Switzerland and spanning the French-Swiss border, the CERN Large Hadron Collider (LHC) accelerated 
two counter-rotating proton beams in a 27 km circular tunnel (CITE) up to an energy of 6.5 TeV in 2015.  At several points along the tunnel 
the two beams collided, and surrounding one of these points the CMS experiment was built.

Several accelerator stages were used to create a proton beam in the LHC.  Summarized in Figure \ref{fig:accelComplex} (CITE), each 
proton beam began as isolated proton bunches ionized from hydrogen gas.  Proton bunches were first accelerated by a linear 
accelerator, LINAC 2, and were then injected into the circular Proton Synchrotron (PS) accelerator.  Protons were accelerated 
to an energy of 26 GeV by the PS, and then were injected into the circular Super Proton Synchrotron (SPS) accelerator.  
The SPS accelerated proton bunches to an energy of 450 GeV, which were subsequently injected into the LHC.  Once the SPS 
injected the desired number of proton bunches to fill two beams, the LHC accelerated both beams to the desired energy, 
and focused the beams to increase the rate of pp collisions.  In 2015 each beam was accelerated to an energy of 6.5 TeV, and 
contained as many as 2300 proton bunches.

\begin{figure}[ht]
	\centering
	\includegraphics[width=0.6\textwidth]{figures/missingImage.png}
	\caption{CERN accelerator systems that feed in to the LHC.}
	\label{fig:accelComplex}
\end{figure}


In the LHC tunnel, superconducting radio-frequency (RF) cavities accelerated proton bunches, and magnets steered and 
focused the beams towards the collision point at the heart of CMS.  Sixteen RF cavities (8 per beam) accelerated proton 
bunches to an energy of 6.5 TeV, and also served as a control system in which protons in any bunch moving too quickly 
were slowed, while protons moving too slowly experienced greater acceleration.  Dipole, quadrupole and higher order 
magnets operated at magnetic field strengths up to 8$\unit{T}$ steered the beams around the LHC, and focused them 
on the nominal interaction point inside CMS to increase the rate of pp interactions.

DISCUSS INST AND INTEGRATED LUMI, CONNECT TO 2015 COLLISION DATA AND SEARCH FOR LOW XSXN WR PRODUCTION

\section{The CMS Coordinate System and Kinematic Variables}
\label{sec:coordinateSystemAndKinematicVars}
The coordinate system and variables used to characterize the kinematics of detected particles were defined in 
the following way.  The right-handed coordinate system was defined with the $z$ axis pointed in the direction 
of the counter-clockwise rotating beam, the $y$ axis pointed up towards the surface, and the $x$ axis pointed towards 
the center of the LHC ring.  In this coordinate system particle trajectories were measured using two angular 
variables, $\phi$ and $\eta$, that were defined in terms of angles in two coordinate planes.  $\phi$ was defined 
as the angle in the $x-y$ plane, and took values between 0 and $2\pi$.  The angle $\theta$ in the $y-z$ plane, 
which was 0 ($\frac{\pi}{2}$) when pointing along the $z$ ($y$) axis, was transformed into pseudorapidity 
$\eta$ through the following function:

\begin{equation}
	\eta \equiv -\ln{\tan{\frac{\theta}{2}}}
\end{equation}

Pseudorapidity of 0 was the $y$ axis, and pseudorapidity of $\infty$ was the $z$ axis.
An equivalent definition of pseudorapidity in terms of the energy E and longitudinal 
momentum $p_{z}$ of a particle in the massless particle limit was:

\begin{equation}
	\eta \equiv \frac{1}{2}\ln{\frac{E+p_{z}}{E-p_{z}}}
\end{equation}

Pseudorapidity was used to describe particle trajectories instead of the angle $\theta$ for several reasons.  Particles 
produced by pp interactions were often boosted along the $z$ axis, but the degree of boost was never well 
defined because the momenta of the interacting partons could never be measured.  This boost affected the 
trajectories of particles in $\theta$, but did not affect particle trajectories measured in $\eta$.  In 
addition, the particles that caused radiation damage in CMS sub-detector systems were produced at 
a rate that scaled linearly with $\eta$.  Thus, in discussions about topics where radiation damage 
or particle isolation played a role, ideas could be more clearly articulated using $\eta$.

The main variables used to quantify the energy of a detected particle were transverse energy 
$E_{T} \equiv E/\cosh{\eta}$ and transverse momentum $p_{T} \equiv |p|/\cosh{\eta}$, while the distance between 
a particle and another point in the $(\eta, \phi)$ space was measured as $\Delta R \equiv \sqrt{\eta^{2} + \phi^{2}}$.  
Transverse momentum, the component of momentum perpendicular to the proton beam axis, and transverse energy, the 
analogous form of energy, were used because both were insensitive to the unknown momenta of the interacting partons.  
Explained in later chapters, the distance $\Delta R$ was measured between the trajectory of one particle 
and another point in the $(\eta, \phi)$ space, which often was the trajectory of another particle.

\section{The Silicon Tracker}
\label{sec:siTrackerDescription}
Built from separate silicon pixel and silicon strip trackers, the silicon tracker detected charged particles  
and measured their momenta, and identified interaction vertices where charged particles were produced.  Closest 
to the beam axis was the pixel tracker, which used small silicon pixels to pinpoint pp interaction vertices 
with $\mu$m precision.  Surrounding the pixel tracker were larger silicon strip detectors that measured the 
momenta of charged particles.  Charged particles produced within the tracker acceptance of $|\eta| < 2.5$ ionized 
charge in the silicon pixel and strip trackers, and the measurement of this charge formed the basis for all tracker 
measurements.

The pixel tracker was built from small rectangular pixel detectors ($100 \times 150 \mu$m) organized in two 
structures based on $|\eta|$.  In the barrel region ($0 < |\eta| \lesssim 1.2$) where the rate of radiation damage was 
low, individual pixel detectors were arranged into three 53 cm long concentric cylindrical shells centered on the 
$z$ axis, with radii of 4, 7 and 10 cm relative to the $z$ axis.  The face of each barrel pixel pointed towards the interaction 
point, and, for reasons described later, their faces were aligned with the magnetic field direction.  In the endcap region 
($1.2 \lesssim |\eta| \leq 2.5$) where the rate of radiation damage was high, two layers of pixel detectors were 
installed in a turbine pattern as shown in Figure \ref{fig:pixelTracker} \cite{pixelCommissioning}.  These disks 
were placed at $|z| =$ 35 and 47 cm away from the midpoint of the pixel barrel cylindrical shells, and covered 
radii between 6 and 15 cm from the $z$ axis.  For reasons discussed later the face of each endcap region pixel 
pointed 20 degrees away from the interaction point.


\begin{figure}[ht]
	\centering
	\includegraphics[width=0.8\textwidth]{figures/pixelDetectorSchematic.png}
	\caption{The barrel and endcap sections of the pixel tracker.}
	\label{fig:pixelTracker}
\end{figure}

Charged particles ionized charge in every pixel detector they hit, and this charge was measured by readout 
modules attached to groups of pixels\footnote{$\thicksim$16000 readout modules were connected to 66 million individual pixels}.  Each readout module created an 
electric field along the surface of several pixels that swept ionized charges into a charge collector.  The 
position resolution of pixel detectors improved when ionized charge from one charged particle was measured 
over a greater area, so all pixels were positioned relative to the magnetic field to create a 
Lorentz force that encouraged charges ionized in one pixel to migrate to nearby pixels.  The resulting pixel 
tracker configuration allowed interaction vertex coordinates to be measured with $\lesssim 20\mu$m resolution along the 
$z$ axis, and with $\lesssim 10\mu$m resolution in the plane perpendicular to the $z$ axis.

Located outside the pixel tracker, the silicon strip tracker was built from larger rectangular silicon detectors 
organized into four structures based on $|\eta|$, $|z|$ position and radius $r$ from the $z$ axis.  In the barrel region 
($0 < |\eta| \lesssim 1.2$), four cylindrical shells of 10 cm $\times$ $80-120$ $\mu$m silicon strips were placed within 20 cm $< r <$ 55 cm.  At
55 $< r < $110 cm where radiation levels were much lower, six layers of thicker, larger surface area (25 cm $\times$ $120-180$ $\mu$m) 
strips were placed in cylindrical shells for improved momentum resolution.  To improve track position resolution the two strip layers with smallest 
$|r|$ each contained two layers of silicon strips separated by a small distance and tilted relative 
to each other by 100 mrad.  For the same reason the first two of six strip detector layers in the region $r >$ 55 cm 
were also built with the double layer design.  Similar to the pixel tracker, the strip tracker in the endcap region 
($1.2 \lesssim |\eta| \leq 2.5$) used strip detectors arranged in disks.  Three disk layers were installed between 
$1.2 \lesssim |\eta| \lesssim 2.3$, 20 $< r <$ 55 cm and 65 $< |z| <$ 110 cm, and nine disk layers with thicker strips were installed in 
the region $1.0 \lesssim |\eta| \lesssim 2.5$, 20 $< r <$ 110 cm and 120 $< |z| < $ 280 cm, as shown in Figure \ref{fig:stripTracker} \cite{cmsTDR}.  
To improve track position resolution two of the three inner disk layers, and three of the nine outer disk layers 
used two closely spaced silicon strips tilted by 100 mrad.  The faces of all strip detectors pointed at the interaction 
point.

\begin{figure}[ht]
	\centering
	\includegraphics[width=0.8\textwidth]{figures/siliconStripAndPixelDetectorTwoDimView.png}
	\caption{The barrel and endcap sections of the silicon strip tracker for $\eta \geq 0.$ and one quadrant of $\phi$.  The pixel tracker is shown to scale in the bottom left corner.}
	\label{fig:stripTracker}
\end{figure}

Charged particles ionized charge in every strip detector they hit, and this charge was measured by readout 
modules attached to groups of strips\footnote{$\thicksim$15400 readout modules were connected to 9.6 million individual strips}.  Each readout module created an 
electric field along the surface of several strips that swept ionized charges into a charge collector.  The resulting 
strip tracker measured the $p_{T}$ of 100 GeV $p_{T}$ muons with resolution better than 2\% in the barrel, and 
better than 7\% in the endcap.

To measure the position and momenta of charged particles with the quoted resolutions, the alignment and calibration 
of the tracker were measured before and during pp collisions in 2015.  Before collisions, cosmic ray muons were 
used to measure the tracker alignment and momentum response, and derive calibration factors that accounted for 
changes in either.  During collisions $Z \rightarrow \mu\mu$ events and cosmic ray muon events were used to monitor 
and recalibrate the tracker alignment and momentum response.


\section{The Electromagnetic Calorimeter}
\label{sec:ecalDescription}

Surrounding the silicon tracker was the electromagnetic calorimeter (ECAL), which detected photons produced by 
pp interactions, and distinguished electrons and positrons ($e^{\pm}$) from other charged particles detected by the 
tracker.  The ECAL was an absorption calorimeter built from homogeneous, scintillating lead-tungstate (PbWO$_{4}$) crystals.  
In response to incident photons and $e^{\pm}$ with total energy $\gtrsim$ 1 GeV, the ECAL crystals emitted 
amounts of visible light proportional to the incident particle energy.  By measuring the amount of visible 
light produced by the lead-tungstate crystals, the ECAL measured the energies of photons and $e^{\pm}$ with $0 < |\eta| < 3.0$.

The ECAL contained $\thicksim$76000 crystals divided into two $|\eta|$ regions based on the rate of radiation 
damage.  In the low $|\eta|$ barrel region ($0 < |\eta| < 1.479$) where the rate of radiation damage was low, 61200 
crystals of length 23 cm (26 radiation lengths) were arranged in a cylindrical shell, and each crystal pointed 3 degrees away from the 
interaction point.  The front face of each crystal measured $\thicksim$22 $\times$ 22 mm, and resided 129 cm 
from the beam axis, $\thicksim$19 cm away from the outer most silicon tracker barrel layer.  In the high $|\eta|$ 
endcap region ($1.479 < |\eta| < 3.0$) where the rate of radiation damage was high, 14648 crystals (half in each 
endcap) of length 22 cm (25 radiation lengths) were installed in a disk, and each crystal pointed 3 degrees away from the interaction 
point.  The front face of endcap crystals measured $\thicksim$29 $\times$ 29 mm, and sat behind two layers of 
thick lead absorber and thin silicon detectors that constituted the preshower detector.  The preshower 
mitigated radiation damage to endcap region lead-tungstate crystals, and improved the spatial resolution with 
which endcap region electromagnetic showers were studied.

Photons and $e^{\pm}$ that impinged on the ECAL produced showers of lower energy $e^{-}$s and $e^{+}$s.  Incident $e^{\pm}$s interacted 
with lead-tungstate nuclei through the bremsstrahlung process and produced gamma rays.  Subsequently, gamma rays 
interacted with lead-tungstate nuclei and were converted into $e^{-}e^{+}$ pairs.  For every $e^{\pm}$ 
that impinged on the ECAL, the cycle of bremsstrahlung followed by pair conversion repeated many times, 
and a shower of lower energy $e^{-}$s and $e^{+}$s was made.  Every energetic photon that impinged on the 
ECAL first converted into an $e^{-}e^{+}$ pair, then this pair underwent repeated cycles of bremsstrahlung 
followed by pair conversion, and a shower of lower energy $e^{-}$s and $e^{+}$s was made.  The crystal lengths 
guaranteed an average incident photon or $e^{\pm}$ lost more than $99.999$\% of its incident energy through 
radiative processes before leaving the ECAL, so the cycle of bremsstrahlung followed by pair conversion 
stopped when the incident particle energy was too low to produce gamma rays through bremsstrahlung.  The 
transverse size of each crystal was large enough that a 3 $\times$ 3 crystal grid contained the full shower 
created by an incident photon or $e^{\pm}$ that did not interact with the silicon tracker.

Showers of $e^{-}$s and $e^{+}$s interacted with lead-tungstate atoms and quickly produced distinguishable 
signals that were measured to determine the energies of incident photons and $e^{\pm}$s.  Coulomb interactions 
transferred several eV of energy from shower $e^{\pm}$s to atomic electrons bound to lead-tungstate nuclei.  
In addition, relativistic shower $e^{\pm}$s produced Cherenkov photons with several eV of energy, and 
these Cherenkov photons were absorbed by atomic electrons bound to lead-tungstate nuclei.  The atomic 
electrons were excited to meta-stable states, and in $\lesssim$ 20 ns more than 80\% of atomic electrons 
returned to their ground states by emitting scintillation photons in the visible light regime.  The total energy of 
scintillation photons was proportional to the incident photon or $e^{\pm}$ energy, so the energies 
of scintillation photons were measured to determine the energy of the incident photon or $e^{\pm}$.  Most 
scintillation photons were emitted towards photodetectors at the back faces of the ECAL crystals.  To 
maximize the fraction of photons which reached the photodetectors, 5 faces of each crystal were polished 
to reduce the impact of surface impurities or imperfections, each crystal was wrapped in a reflective, 
non-absorbing material, and 17 different crystal shapes (all rectangular) were used to minimize gaps 
between crystals.  Scintillation photons were measured by avalanche photodiodes in the barrel region, 
and vacuum photo-triodes in the endcap region.  Both types of photodetectors converted scintillation 
photons into electric charge, and the amount of charge was measured to determine the energy of the 
incident photon or $e^{\pm}$.  With these technologies, the ECAL measured the energy (not $E_{T}$) of 
a 100 GeV photon or $e^{\pm}$ with resolution better than 1\%, and measured the energy E of any 
incident photon or $e^{\pm}$ with resolution given by:

\begin{equation}
	\frac{\sigma_{E}}{E} = \frac{2.8\%}{\sqrt{E/(1 GeV)}} \oplus \frac{0.128 GeV}{E} \oplus 0.3\%
\end{equation}

To measure photon and $e^{\pm}$ energies with the quoted resolution, the ECAL was calibrated before 
and during 2015 pp collisions.  Before collisions, the ECAL was calibrated based on 2012 pp collision 
data.  Once collisions began, data collected from collision events was used by several methods 
to derive ECAL calibration constants for every lead-tungstate crystal.  The following quantities 
were measured for every crystal in millions of events:

\begin{itemize}
	\item $\pi^{0}/\eta$ masses through their decays to $\gamma\gamma$
	\item the average crystal energy in randomly selected events 
		(considering many randomly selected events the energy measured should be symmetric in $\phi$, so within one $\phi$ ring each crystal should measure the same average energy)
	\item the ratio $\frac{energy E_{T} measured by the ECAL}{momentum p_{T} measured by the tracker}$ for isolated electrons coming from W or Z boson decays
\end{itemize}

and relative calibration constants were derived such that every crystal had the same calibrated response.  
Then, the absolute scale of each crystal's response was calibrated using $e^{\pm}$ from 
$Z \rightarrow e^{\pm}e^{\mp}$ decays.  In events where a Z boson decayed to an $e^{\pm}e^{\mp}$ pair, the 
crystal responses were corrected so that the average mass of all $e^{\pm}e^{\mp}$ pairs was equal to 
the true Z boson mass.

\section{The Hadronic Calorimeter}
\label{sec:hcalDescription}
The a

\section{The Muon Detectors}
\label{sec:muonDetectorsDescription}

\section{The Trigger System}
\label{sec:triggerDescription}





%use this lumi info in the trigger section.  it is not relevant for the detection of leptons and jets
%why are high lumi and energy needed
The LHC was built to discover new physics mediated by heavy particles which are weakly coupled to Standard Model (SM)
leptons, hadrons and bosons.  To produce heavy, weakly interacting particles, like a $2.0 TeV$ mass right handed W ($W_{R}$) boson
with production cross section 5 fb, at a rate sufficient for detection, average instantaneous luminosities of
$10^{33} \frac{1}{cm^{2}s}$ are needed.  At such luminosities, other more abundant or lower energy processes will occur
at a significant rate, as shown in Figure \ref{fig:smProductionXsxns}.

Of all processes created by collisions in the LHC, low energy, inelastic QCD interactions occur at the highest rate, of the order
$10^{8}$ events per second at $10^{33} \frac{1}{cm^{2}s}$.  Contributions from these events to new physics searches are mitigated by vetoing
on large groups of particles collinear with the beam line, or requiring the presence of leptons or photons.  However, these
requirements are not sufficient to suppress leptonic backgrounds created by the production and decay of heavy quarks, and QCD multijet
events in which a jet is incorrectly reconstructed as a lepton due to energetic neutral pions in the jet.  Such high energy QCD multijet
and heavy quark decay events occur at a rate of $10^{5}$ events per second at $10^{33} \frac{1}{cm^{2}s}$.  These event rates dwarf
the rate at which entire events can be read out of CMS and saved to disk, approximately $10^{3}$ events per second.  The tools, generally
called triggers, used to select and save collision events at approximately $10^{3}$ Hz will be discussed later.

\begin{figure}[h]
	\centering
	\includegraphics[width=0.6\textwidth]{figures/lhc_and_tevatron_cross_sections_2006.png}
	\caption{Production cross sections at the LHC and Tevatron as a function of center of mass energy. \cite{}}
	\label{fig:smProductionXsxns}
\end{figure}


%Operations
The LHC started operations with collisions at a lower center of mass energy and lower instantaneous luminosity than design.  In 2015 the center of mass
energy was increased to 13 TeV, and the luminosity was increased closer to the design level.  The 2015 data taking period was split into two portions - from the start
of 2015 collisions until August the time separation between consecutive proton bunches in each beam was 50 ns.  After August, the bunch
separation decreased to 25 ns to increase the instantaneous luminosity, and proton proton collisions continued until November.  From August until November, instantaneous 
luminosities reached $5x10^{33} \frac{1}{cm^{2}s}$, but problems with the CMS
magnet cooling system limited the amount of data collected to 2.6 $fb^{-1}$.  Decreasing the bunch spacing increased the sensitivity of every new physics search, but
came with the cost of more proton proton interactions in every bunch collision, or pileup.  At 13 TeV the total inelastic cross section is approximately 70 mb \cite{Haevermaet}, 
so for instantaneous luminosities between $3-5x10^{33} \frac{1}{cm^{2}s}$ in 25 ns running the expected pileup per bunch crossing is $8-13$, which is consistent with
the pileup observed in 25 ns collision data.  Pileup increases the complexity of events$\footnote{15 inelastic pp collisions yield a total flux of particles similar to one top antitop quark pair event}$, and makes event reconstruction and offline analysis more challenging.


\subsection{Compact Muon Solenoid Detector Overview}
\begin{itemize}
	\item trigger description, followed by particle flow event reconstruction
\end{itemize}
The Compact Muon Solenoid (CMS) experiment is designed to search for signs of new physics at the high energy and luminosity regime probed by the LHC.  The LHC was built
as a discovery machine, so particular attention has been paid to searches for physics beyond the SM (BSM), such as supersymmetry and extensions of the SM.

The The CMS detector approximates a cylinder in shape, with a length of 22 meters, outer radius of 7 meters, and weight of 14500 tons.  It is divided into a central barrel region ($|\eta| < 1.4$)
in which separate subdetectors are arranged as concentric cylindrical shells at increasing radii from the beam axis, and a forward endcap region ($|\eta| > 1.4$) 
in which subdetectors are arranged as layers stacked along the beam axis.  Illustrated in Figure \ref{fig:cmsDetectorComponentView}, the CMS subdetectors have complete 2$\pi$ radian coverage in
$\phi$, and hermetic calorimeter coverage from $-5 \leq \eta \leq 5$ ($-179^{\circ} \leq \theta \leq 0.77^{\circ}$) with respect to the beam axis.  Complete detector
coverage over this large angular range allows precise measurements to be made of missing energy, and the mass of yet-undiscovered light and heavy particles.

\begin{figure}[h]
	\centering
	\includegraphics[width=0.6\textwidth]{figures/missingImage.png}
	\caption{The CMS detector with the coordinate system shown   .  The x axis points towards...}
	\label{fig:cmsDetectorComponentView}
\end{figure}


Starting at the interaction point and going radially outward, the first subdetector encountered by particles produced in collisions is the silicon tracker.
The tracker consists of silicon pixel detectors surrounded by silicon strip detectors, all of which are immersed in a 3.8 Tesla magnetic field produced by a superconducting solenoid.
Both detectors are only sensitive to charged particles, and facilitate precise measurements of charged particle momenta
and the positions of interaction vertices by measuring the curvature of charged particle tracks which have $|\eta| \leq 2.5$.

Particles which exit the tracker with sufficient momentum to travel approximately 1.0 meter radially away from the interaction point will reach the
electromagnetic calorimeter (ECAL).  ECAL consists of approximately 76000 lead tungstate crystals which, like the tracker, are sit in a 3.8 Tesla
magnetic field.  The radiation hardness and fast scintillation time of lead tungstate were key factors in chosing to build ECAL out of it.
ECAL was designed to measure the energy, time, position and shower shape of electrons and photons with $|\eta| \leq 3.0$ by measuring scintillation
light produced by lead tungstate in electron and photon showers.

Particles which traverse ECAL with sufficient momentum to continue moving away from the interaction point hit the hadronic calorimeter (HCAL).  Built
from alternating layers of metal and scintillating plastic tiles, HCAL was designed to measure the energy of charged and neutral
hadrons, and to provide structural support to all of CMS.  Energetic hadrons produce showers of lower energy particles in metal layers, and the
showers which reach nearby scintillating tiles produce scintillation light.
HCAL extends out to $|\eta| = 3.0$, is enveloped by the 3.8 Tesla magnetic field, and
is complimented by a forward hadron calorimeter (HF) which extends the calorimetry coverage out to $|\eta| = 5.0$.  Due to the high radiation
environment in its vicinity, HF is built from radiation hard steel blocks laced with radiation hard quartz fibers.  Charged particles which travel
through HF produce Cherenkov light in the quartz fibers; from the Cherenkov light an energy measurement is extracted.

Surrounding HCAL is the CMS magnet.  It produces a 3.8 Tesla magnetic field from superconducting niobium titanate wire wound into a solenoid, and
carries more than 10000 amps of current during normal operation.

Particles which pass through HCAL and the magnet encounter the last CMS subdetector, designed to measure the energy of muons.  The muon
detectors consist of three technologies - drift tubes (DT) and resistive plate chambers (RPC) which cover out to $|\eta| \leq 1.3$, while cathode
strip chambers (CSC) and RPCs which cover $1.3 \leq |\eta| \leq 2.4$, with some overlap near $|\eta| = 1.3$.  All three technologies measure the amount
of charge ionized in a gas when a muon traverses an enclosed volume filled with gas, and convert this charge into an energy.

All CMS subdetectors are utilized in particle flow (PF) reconstruction and event triggering.  In. 

\subsection{Track and vertex reconstruction}

\subsection{Muon Reconstruction and Identification}
\begin{itemize}
	\item importance of muons, challenges with reco and energy measurements
	\item muon detector technologies
	\item muon reconstruction algorithms
	\item challenge with pT measurement of high pT muons
	\item explain TuneP algorithm and show performance plots
	\item explain ID variables, isHighPt ID and show performance plots
	\item explain usefulness of muon isolation (reject muons from QCD) and how it is calculated
	\item muon L1 and HLT: design and performance plots
\end{itemize}

\subsection{Electron Reconstruction and Identification}
\begin{itemize}
	\item importance of electrons, challenges with reco and energy measurements
	\item ECAL
	\item electron reconstruction algorithms
	\item challenge with pT measurement of high pT muons
	\item explain TuneP algorithm and show performance plots
	\item explain ID variables, HEEP ID and show performance plots
	\item explain usefulness of electron isolation (reject eles from QCD) and how it is calculated
	\item electron L1 and HLT: design and performance plots
\end{itemize}



\subsection{Particle Flow and Jet Reconstruction}



