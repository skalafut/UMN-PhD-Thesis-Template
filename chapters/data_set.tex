%%%%%%%%%%%%%%%%%%%%%%%%%%%%%%%%%%%%%%%%%%%%%%%%%%%%%%%%%%%%%%%%%%%%%%%%%%%%%%%%
% data_set.tex:
%%%%%%%%%%%%%%%%%%%%%%%%%%%%%%%%%%%%%%%%%%%%%%%%%%%%%%%%%%%%%%%%%%%%%%%%%%%%%%%%
\chapter{Background Estimation}
\label{sec:backgroundEstimation}
%%%%%%%%%%%%%%%%%%%%%%%%%%%%%%%%%%%%%%%%%%%%%%%%%%%%%%%%%%%%%%%%%%%%%%%%%%%%%%%%
Data used in the $ee$- and $\mu\mu$-channel searches contained events produced by ST backgrounds.  
Applying the event selection described in Chapter \ref{sec:event_selection_chapter} to data reduced 
the number of background events found in the $\Mlljj$ distribution, but not to a negligible level.  
The magnitude and shape of the $\Mlljj$ distribution found in background events was estimated using 
procedures described here.


\section{Top Quark Background}
\label{sec:topQrkBkgnds}
One of the two largest backgrounds in the \WR and \nul search was ST processes that produced at least 
one top quark, like top quark $\plus$ $W$ boson production (top+W, Figure \ref{fig:singleTopDiags}), 
and top anti-top quark pair production ($\ttbar$, Figure \ref{fig:ttbarDiag}).  Top 
quarks decay to a $W$ boson and bottom (b) quark, which subsequently decay to leptons and hadrons.  
Based on lepton universality in $W$ boson and b quark decays, in events where one or more top (or anti-top) 
quarks are produced and decay to two leptons, the $e\mu$ final state is produced at twice 
the rate of the $ee$ or $\mu\mu$ final state.  The \WR cannot decay to the $e\mu$ final state at leading 
order in the weak coupling constant, so data events with one reconstructed electron and muon were used to 
estimate the shape and magnitude of the $\Memujj$ distribution from top quark backgrounds.  Normalization 
factors of $\sim 0.5$ were then applied to scale the $\Memujj$ distribution into the $\Meejj$ or $\Mmumujj$ 
distribution.

\begin{figure}[h]
	\centering
	\includegraphics[width=0.7\textwidth]{figures/topAntiTopFeynDiagram.png}
	\caption{$\ttbar$ Feynman diagram \cite{ttbarDiagram}.}
	\label{fig:ttbarDiag}
\end{figure}

\begin{figure}[h]
	\centering
	\includegraphics[width=0.7\textwidth]{figures/singleTopQuarkFeynDiagrams.png}
	\caption{Single top quark Feynman diagrams \cite{singleTopQrkDiagrams}.}
	\label{fig:singleTopDiags}
\end{figure}

Data events in the $e\mu$-channel were selected during collisions using a Level-1 trigger that required 
a track in at least one muon DT or CSC detector with $\pt > 16$ $\GeV$.  Events that passed the Level-1 selection 
were reconstructed if they passed the following electron and muon HLT selections:

\begin{itemize}
	\item A track reconstructed in the silicon tracker with $\pt > 30$ $\GeV$ and $|\eta| < 2.4$ was geometrically matched to 
		the muon detector hits that passed the L1 trigger.  Considering this set of muon detector hits and the matching 
		reconstructed track:
	\begin{itemize}
		\item A curve representing the muon trajectory through CMS was fitted to the reconstructed track and at least one 
			muon detector hit with $\chi^{2}/nDOF < 20$.
		\item In the plane perpendicular to the beam axis, the distance between the reconstructed track origin and its 
			reconstructed vertex was $< 1$ mm.
	\end{itemize}
	\item One ECAL SC was detected with $\Et > 30$ $\GeV$.
	\item For the SC:
	\begin{itemize}
		\item The ratio of hadronic energy in the HCAL tower behind the SC to the SC energy was $< 0.15$ in the barrel, and $< 0.1$ in the endcap.
		\item Ninety percent of the SC energy was measured in an $(\eta, \phi)$ region that is two crystals wide in $\eta$.
		\item For SCs in the barrel, a reconstructed track with hits in at least two pixel tracker layers extrapolated to the 
			SC centroid along the beam axis within 2.3 \cm, and extrapolated to the SC centroid in $(\eta_{SC}, \phi_{SC})$ within the 
			$(\eta, \phi)$ area of one ECAL crystal.
	\end{itemize}
\end{itemize}

The reconstructed $e\mu$ data events were filtered by the electron, muon, and jet selections described 
previously, with the requirement that one selected lepton was a muon, the other was an electron, 
and at least one lepton had $\pt > 60$ $\GeV$.  Simulated ST background events were selected with the 
same online and offline requirements, and compared to $e\mu$-channel data events.  As shown in Figure 
\ref{fig:dataAndSimsInEMuChannel}, top quark backgrounds produced the majority of events that passed 
the $e\mu$-channel selections.

\begin{figure}[h]
	\centering
	\includegraphics[width=0.7\textwidth]{figures/Mlljj_eMuChannel_log.pdf}
	\caption{The $\Mlljj$ distribution from data and simulated ST events that passed the $e\mu$ selection, excluding 
	the $\Mlljj > 600 \GeV$ cut.  The bin widths were variable, and their contents were normalized to the bin widths.}
	\label{fig:dataAndSimsInEMuChannel}
\end{figure}

%first explain that the Mlljj distribution shape does not change between the emu, ee, and mumu channels
The $\Memujj$ distribution found in data was produced by top quark backgrounds, so its shape was expected 
to be the same in the $\Meejj$ and $\Mmumujj$ distributions.  This expectation was studied using simulated 
top quark background events.  Simulated 
events were selected with the $ee$-, $e\mu$-, and $\mu\mu$-channel selections, and these events were used 
to produce $\Meejj$, $\Memujj$, and $\Mmumujj$ distributions.  The ratio of the binned distributions $\Meejj / \Memujj$ 
and $\Mmumujj / \Memujj$, shown in Figure \ref{fig:ttbarSFratios}, was consistent with a constant value, 
independent of $\Mlljj$, within the statistical uncertainty of each bin.  The constant ratios indicated that the 
shape of the $\Mlljj$ distribution in top quark background events was the same in all three lepton channels, 
which meant the $\Memujj$ distribution found in data could be used to estimate the top quark background in 
the $ee$- and $\mu\mu$-channel.

%then explain how scale factors were derived to rescale the magnitude of the Memujj distribution to the 
%Meejj and Mmumujj distributions based on the difference between electron and muon acceptances, and 
%efficiencies of the trigger, reco, and offline selections
The $\Memujj$ distribution found in data represented top quark backgrounds, and differed from the top quark 
$\Meejj$ and $\Mmumujj$ distributions by a normalization factor (NF) equal to the difference in several quantities.  
The difference between the $e\mu$ and the same flavor $\ell\ell$ production rate is 0.5, so the NF 
was expected to be $\approx$0.5.  Differences existed between the electron and muon acceptance $\times$ 
selection efficiency\footnote{For example, muons were reconstructed in the region $1.44 \leq |\eta| \leq 1.57$, 
while electrons were not.}, and they caused the $ee:e\mu$ and $\mu\mu:e\mu$ NFs to differ from 0.5.  
The normalization factors were calculated using simulated top quark events selected with the $ee$-, $e\mu$-, 
and $\mu\mu$-channel selections and corrections described previously.  Selected events were used to 
produce the $\Mlljj$ distribution in each channel, and the integral of each distribution $\Nlljj$ was 
calculated.  The NF that scales the simulated $\Memujj$ distribution into the $\Meejj$ 
distribution was calculated as $\Neejj / \Nemujj \thickspace = \thickspace 0.432$, and for the $\Mmumujj$ 
distribution was calculated as $\Nmumujj / \Nemujj \thickspace = \thickspace 0.659$.  The lower acceptance 
$\times$ selection efficiency for electrons relative to muons decreased (increased) the $ee:e\mu$ 
($\mu\mu:e\mu$) NF relative to 0.5.  The NFs were 
calculated independent of $\Mlljj$, but were dominated by simulated events with $\Mlljj < 1500$ $\GeV$ 
because the number of simulated top quark events decreased rapidly with increasing $\Mlljj$.  As a 
result, an error entered the top quark background estimate that was proportional to the per-bin 
fluctuations of $\Mmumujj / \Memujj$ and $\Meejj / \Memujj$ around the values 0.659 and 0.432 shown 
in Figure \ref{fig:ttbarSFratios}.  The impact of this error was covered by assigning a 10\% uncertainty 
to both NFs.

After calculating the NFs, the top quark background in the $\mu\mu jj$ ($eejj$) final state was estimated by 
multiplying the $\Memujj$ distribution from data by $0.659$ ($0.432$).  Based on simulations of all ST backgrounds, 
contributions of non-top quark backgrounds to the $e\mu$ channel were more than one order of magnitude smaller 
than the 10\% top quark background uncertainty.  For this reason, simulated non-top quark background events 
were not subtracted from $e\mu$ channel data.

\begin{figure}[btp]
	\centering
	\subfigure{
		\includegraphics[width=0.45\textwidth]{figures/flavor_ratio_EE_variablebinwidth.pdf}
	}
	\subfigure{
		\includegraphics[width=0.45\textwidth]{figures/flavor_ratio_MuMu_variablebinwidth.pdf}
	}
	\label{fig:ttbarSFratios}
	\caption{The bin-by-bin ratio of the $\Mlljj$ and $\Memujj$ distributions from simulated top quark backgrounds, where 
		$\ell$ is an electron on the left, and a muon on the right.}
\end{figure}


\section{\DY Background}
\label{sec:dyBkgnd}
The second of the two largest backgrounds in the \WR and \nul search was the production and decay of a Z boson to 
a lepton pair in association with jets (\DY), $pp \rightarrow Z+jets \rightarrow \ell\ell+jets$.  The \DY 
background was estimated using simulated $\DY$+jets events selected with the $ee$- and $\mu\mu$-channel requirements 
described previously.  $\DY$+jets simulations were developed to model $\DY$+jets interactions observed in data, and 
mis-modelling in simulations lead to two discrepancies between $\DY$+jets simulations and data.

\subsection{\DY Normalization in $\Mlljj$}
\label{sec:dyNormInMlljj}
$\DY$+jets events were simulated using the \MADGRAPH generator at leading order in the electroweak and strong coupling 
constants, while $\DY$+jets events observed in data were produced at all orders in these coupling constants.  The 
simulated dataset contained more events than were expected in the 2.6 fb$^{-1}$ of data, so the simulated events were 
weighted to the integrated luminosity of data using an uncertain cross section $\times$ branching fraction calculated during 
simulations.  The cross section $\times$ branching fraction uncertainty was sensitive to the higher order electroweak 
and strong coupling corrections, and it produced a discrepancy in the $\Mlljj$ distribution normalization between data 
and simulated events.

The discrepancy in the $\Mlljj$ distribution normalization between simulated $\DY$+jets events and data was estimated 
using a $Z \rightarrow \ell\ell$ control region where data and simulated events were expected to agree.  Events in 
the $ee$-channel control region were selected during collisions using a Level-1 trigger that required one ECAL SC with 
$\Et > 30$ $\GeV$ and $|\eta| < 2.1$.  Events that passed the Level-1 selection were reconstructed offline if they 
passed the following double electron HLT selections: (simulated events were required to pass the same trigger selections)

\begin{itemize}
	\item One SC was detected with $\Et > 30$ $\GeV$, and a second non-overlapping SC was detected with $\Et > 4$ $\GeV$.
	\item For the SC with $\Et > 30$ $\GeV$:
	\begin{itemize}
		\item The ratio of hadronic energy in the HCAL tower behind the SC to the SC energy was $< 0.055$ in the barrel, and 
			$< 0.07$ in the endcap.
		\item Ninety percent of the SC energy was measured in an $(\eta, \phi)$ region that was two crystals wide in $\eta$.
		\item A reconstructed track with hits in at least two pixel tracker layers was matched to the SC, and extrapolated to the SC 
			centroid along the beam axis within 1 cm, and extrapolated to the SC centroid in $(\eta, \phi)$ within the $(\eta, \phi)$ 
			area of $\frac{1}{2}$ an ECAL crystal.  In addition, the SC energy and track momentum cannot differ by more than 50\%.
		\item In a cone of radius $\Delta R =$ 0.3 centered on the SC:
		\begin{itemize}
			\item The fraction of the total ECAL energy in the cone not associated with the SC was $< 0.225$ in the barrel, and 
				$< 0.121$ in the endcap.
			\item The total HCAL energy in the cone divided by the SC energy is $< 0.155$ in the barrel, and $< 0.16$ in the endcap.
		\end{itemize}
	\end{itemize}
\end{itemize}

Events in the $\mu\mu$-channel control region were selected during collisions using a Level-1 trigger that required a track 
in at least one muon DT or CSC detector with $\pt > 20$ $\GeV$.  Events that passed the Level-1 selection were reconstructed offline 
if they passed the following single muon HLT selections: (simulated events were required to pass the same trigger selections)

\begin{itemize}
	\item A track reconstructed in the silicon tracker with $\pt > 22$ $\GeV$ and $|\eta| < 2.4$ was geometrically matched to 
		the muon detector hits that passed the L1 trigger.  Considering this set of muon detector hits and the matching reconstructed 
		track:
	\begin{itemize}
		\item A curve representing the muon trajectory through CMS was fitted to the reconstructed track and at least 
			one muon detector hit with $\chi^{2}/nDOF < 20$.
		\item In the plane perpendicular to the beam axis, the distance between the reconstructed track origin and its 
			reconstructed vertex was $< 1$ mm.
		\item In a cone of radius $\Delta R =$ 0.3 centered on the muon trajectory:
		\begin{itemize}
			\item The total ECAL energy in the cone divided by the momentum measured by the muon detectors is $< 0.11$ in 
				the barrel, and $< 0.08$ in the endcap.
			\item The total HCAL energy in the cone divided by the momentum measured by the muon detectors is $< 0.21$ in 
				the barrel, and $< 0.22$ in the endcap.
			\item The total $\pt$ of all silicon tracker tracks in the cone (excluding the muon track) divided by the 
				muon track $\pt$ is $< 0.09$.
		\end{itemize}
	\end{itemize}
\end{itemize}

Events .

\subsection{\DY Shape in $\Mlljj$}
\label{sec:dyShapeInMlljj}
In $\DY$+jets events, in data or simulations, the Z production and decay was reconstructed as a vertex that was 
most often the primary vertex (PV), which was defined as the vertex with the highest $\sum \pt$ of all associated 
tracks in the event.  Selected reconstructed jets were required to have at least one charged hadron from the PV, and the partons 
radiated during the Z production and decay produced the majority of charged hadrons that traced back to the PV\footnote{The 
contamination of $Z \rightarrow \tau\tau$ in events with $ee$+jets or $\mu\mu$+jets final states was minimized by 
requiring two leptons with $\pt > 53$ $\GeV$.}.  $\DY$+jets events were simulated with up to 4 partons radiated from 
the $Z \rightarrow \ell\ell$ interaction, while $\DY$+jets events observed in data were produced with any number - 
0, 2, 6, etc. - of partons radiated from the $Z \rightarrow \ell\ell$ interaction.  The limit of 4 radiated partons 
in simulated events reduced the number of reconstructed jets that could pass the selections, and affected the energy 
of selected jets, as multiple partons could contribute to the same jet.  These jet modelling effects produced a discrepancy 
in the $\Mlljj$ distribution shape between data and simulated events.


The first discrepancy was a difference in normalization between data and \MC, caused by the cross section uncertainty in 
the simulation.  The effect of the discrepancy was estimated by comparing $Z \rightarrow \ell\ell$ events in data and 
\DY simulations, selected with the following requirements:

\begin{itemize}
	\item One of the background only triggers described in Chapter \ref{sec:triggers} was fired.
	\item At least two leptons were found with $\pt > 35 \GeV$ that passed identification selections described in 
		Chapter \ref{sec:event_selection_chapter}.  If more than two leptons were found, only the two highest $\pt$ 
		leptons were used.
	\item At least two jets were found with $\pt > 40 \GeV$ that passed identification selections.  If more than two jets 
		were found, only the two highest $\pt$ jets were used.
	\item The leptons and jets had $|\eta| < 2.4$.
	\item The two leptons had di-lepton mass $70 < M_{\ell\ell} < 110 \GeV$.
	\item Both leptons were separated from both jets by $\Delta R > 0.4$.
\end{itemize}

Using selected events, the $\Mll$ distribution was made and used to compare data to simulations.  Based on simulations 
of non-\DY backgrounds, 3\% of the selected events in data came from non-\DY processes.  The effect of subtracting 
simulated non-\DY events from data was much smaller than the total uncertainty on the \DY background 
estimate, so non-\DY backgrounds were not subtracted from data.  The normalization of the $\Mll$ distribution using 
data events exceeded that of simulated events by 15.7\% in the $ee$ channel, and 14.2\% in the $\mu\mu$ channel.  To 
correct the normalization discrepancy, the \DY background estimate derived from simulated events in the signal region 
and other control region was scaled up by $\sim$15\%.

Two additional $\DY$+jets datasets, simulated with different generators for the first simulation step, were used to 
estimate the uncertainty on the $\sim$15\% correction to the \DY background predicted by simulations.  The $Z \rightarrow \ell\ell$ 
selection described earlier without jet requirements was applied to data and all three simulated $\DY$+jets datasets.  
The jet requirements were removed to compare data to simulations in a phase space where all three simulated datasets 
yielded a large number of events after selections.  Using selected events, the difference in normalization between data and 
simulations in the $\Mll$ distribution was calculated, and the largest difference was taken as a systematic 
uncertainty on $\sim$15\%.  This uncertainty was estimated to be 2.0\% (1.0\%) in the $ee$ ($\mu\mu$) channel.



Unlike the jets produced naturally as a result of quark decay in $\ttbar \rightarrow \ell\ell jj$ events, jets 
produced in \DY events were produced randomly by the strong interaction, sometimes before the Z boson was produced.  Simulations of random 
jet production in \DY events were done only at leading order in the strong interaction, and as a result 
the $\pt, \eta, \phi$ of jets predicted by simulations did not perfectly agree with jets observed in data.  This disagreement 
created the second discrepancy between data and simulated \DY, which manifested as a difference in $\Mlljj$ shape.  
The effect of the $\Mlljj$ shape disagreement was estimated using data and simulations of ST backgrounds selected 
with the signal region requirements described in Chapter \ref{sec:wrCandSelection}, but with $\Mll < 180 \GeV$.  Subsequently, 
$\Mlljj$ distributions were built from selected data and all simulated events, and the \DY component of selected simulated events was 
rescaled by $1.157$ ($1.142$) in the $ee$ ($\mu\mu$) channel to correct for the normalization discrepancy discussed 
previously.  Then, the $\Mlljj$ distributions from data and simulations shown in 
Figure \ref{fig:mlljjLowDileptonMassSideband} were compared, and the largest percentage difference between data and 
all simulations in any bin was taken as a conservative systematic uncertainty on the estimated \DY background.  The 
mis-modelling of jets in simulations resulted in a symmetric 40\% uncertainty assigned to the \DY background in both channels, 
constant versus $\Mlljj$.

\begin{figure}[btp]
\centering
\subfigure{
  \includegraphics[width=0.45\textwidth]{figures/Mlljj_eeChnl_lowMllCR.png}
}
\subfigure{
  \includegraphics[width=0.45\textwidth]{figures/Mlljj_mumuChnl_lowMllCR.png}
}
\caption{The $\Mlljj$ distribution from data and simulated ST events that passed the $\Mll < 180 \GeV$ selection, the 
		$ee$ ($\mu\mu$) channel on the left (right).  The bin widths were variable, and their contents were normalized to 
	the bin widths.}
\label{fig:mlljjLowDileptonMassSideband}
\end{figure}

In an attempt to reduce the 40\% uncertainty, a separate set of simulated $\DY$+jets events were used, in which jet simulations 
were done at a higher order in the strong interaction.  By simulating jet production at a higher order in the strong 
interaction, it was thought that the disagreement in $\Mlljj$ shape could be reduced.  The results, shown in Figure 
\ref{fig:mlljjLowDileptonMassSidebandAMCNLO}, did not support switching simulated \DY datasets.  The primary detriment to the 
higher order \DY simulation was not the jet simulation, but the lack of events.  After selections, the leading order 
simulated $\DY$+jets dataset had $\sim$3 times more events across all $\Mlljj$, and $\sim$10 times more events with $\Mlljj > 1 \TeV$ 
than the higher order simulated $\DY$+jets dataset.

\begin{figure}[btp]
\centering
\subfigure{
  \includegraphics[width=0.45\textwidth]{figures/Mlljj_eeChnl_lowMllCR_AMCNLO.png}
}
\subfigure{
  \includegraphics[width=0.45\textwidth]{figures/Mlljj_mumuChnl_lowMllCR_AMCNLO.png}
}
\caption{The $\Mlljj$ distribution from data and simulated ST events that passed the $\Mll < 180 \GeV$ selection, the 
		$ee$ ($\mu\mu$) channel on the left (right).  The bin widths were variable, and their contents were normalized to 
	the bin widths.  In the \DY events, jet production was simulated at a higher order relative to the standard \DY dataset.}
\label{fig:mlljjLowDileptonMassSidebandAMCNLO}
\end{figure}

A third control region was used to validate the correction applied and uncertainty assigned to the estimated \DY background.  
In this control region, events from data and simulations of all ST backgrounds were selected with 
the signal region requirements described in Chapter \ref{sec:wrCandSelection}, but with $\Mlljj < 600 \GeV$.  Selected 
data and simulated events were used to make an $\Mll$ distribution, and the simulated \DY component was rescaled 
by $\sim1.15$.  Based on comparisons between data and simulations shown in Figure \ref{fig:mllInLowMlljjSideband}, the 
correction applied to simulated \DY events brought data and simulations of ST backgrounds into better agreement.  In addition, 
the uncertainty assigned to simulated \DY allowed the estimated \DY background to float in a range that improved the overall 
agreement between data and simulations.

\begin{figure}[btp]
\centering
\subfigure{
  \includegraphics[width=0.45\textwidth]{figures/Mll_eeChnl_lowMlljjCR.png}
}
\subfigure{
  \includegraphics[width=0.45\textwidth]{figures/Mll_mumuChnl_lowMlljjCR.png}
}
\caption{$\Mll$ for data and simulations of all ST backgrounds in the $\Mlljj < 600\GeV$ control region.  The 
$ee$ ($\mu\mu$) channel was on the left (right).}
\label{fig:mllInLowMlljjSideband}
\end{figure}

The \DY background was estimated by selecting simulated \DY events with the signal region selections described in Chapter 
\ref{sec:wrCandSelection}.  The weight of selected events were scaled by 1.157 (1.142) in the $ee$ ($\mu\mu$) channel, and 
the scaled events represented the estimated \DY background.  Based on comparisons between data and simulated \DY events in 
control regions, a 40\% uncertainty for all $\Mlljj > 600\GeV$ was assigned to the estimated \DY background.


\section{Diboson and W+jets Backgrounds}
\label{sec:dibosonAndWJetsBkgnds}
The production of weak boson $W/Z$ pairs (WW, WZ, ZZ), and single W bosons in association with jets yielded events where 
two charged leptons and two jets were reconstructed and passed the signal region selections.  The number of events in 
data produced by these processes was estimated using simulated W+jets and diboson events.  The expected number of events from diboson and W+jets 
processes, shown in Figure \ref{fig:allExpectedBkgnds}, was small relative to \DY and top quark backgrounds, and concentrated 
in the region $\Mlljj < 2000\GeV$.  The diboson and W+jets processes were neglected when calculating final results for 
this search.

\begin{figure}[h]
	\centering
	\subfigure{
		\includegraphics[width=0.45\textwidth]{figures/useOfLLJJMassAsFigureOfMerit.pdf}
	}
	\subfigure{
		\includegraphics[width=0.45\textwidth]{figures/Mlljj_mumuChnl_signalRegionNoData.pdf}
	}
	\label{fig:allExpectedBkgnds}
	\caption{$\Mlljj$ distributions from simulated \DY, diboson, W+jets backgrounds, and the top quark and QCD backgrounds estimated from 
		data. The normalization of the \WR $\Mlljj$ distribution was scaled down by 70\% to facilitate comparisons between expected 
		signal and backgrounds.  The $ee$ ($\mu\mu$) channel was shown on the left (right).}
\end{figure}


\section{QCD Background}
\label{sec:qcdBkgnd}
The production of jets through QCD processes yielded multi-jet events where two charged leptons and two jets were reconstructed 
and passed the signal region selections.  In such events, two real jets were incorrectly reconstructed and identified as charged 
leptons.  The QCD multi-jet background expected in data was estimated by first selecting events in data with the signal region requirements 
described in Chapter \ref{sec:wrCandSelection}, but with looser identification selections applied to lepton candidates.  Events where one 
or both selected leptons passed the standard lepton identification selections were discarded, as these selected leptons were most likely 
real charged leptons.  Then, remaining events were weighted by the probability for both selected leptons, which most likely 
originated from jets, to pass the standard lepton identification requirements.  This probability was determined in an independent 
sample of data and simulated events as a function of $\eta$ and $\pt$ or $\Et$ of leptons that were incorrectly reconstructed from 
real jets.  After the probability weights were applied to selected events, the contribution of weighted events to the $\Mlljj$ 
distribution was compared to the $\Mlljj$ distribution expected from other ST backgrounds, shown in Figure \ref{fig:allExpectedBkgnds}.  
The expected QCD background was negligible, and was ignored when calculating final results for this search.


%%%%%%%%%%%%%%%%%%%%%%%%%%%%%%%%%%%%%%%%%%%%%%%%%%%%%%%%%%%%%%%%%%%%%%%%%%%%%%%%
