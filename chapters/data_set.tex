%%%%%%%%%%%%%%%%%%%%%%%%%%%%%%%%%%%%%%%%%%%%%%%%%%%%%%%%%%%%%%%%%%%%%%%%%%%%%%%%
% data_set.tex:
%%%%%%%%%%%%%%%%%%%%%%%%%%%%%%%%%%%%%%%%%%%%%%%%%%%%%%%%%%%%%%%%%%%%%%%%%%%%%%%%
\chapter{Problem 1}
\label{Problem 1}
%%%%%%%%%%%%%%%%%%%%%%%%%%%%%%%%%%%%%%%%%%%%%%%%%%%%%%%%%%%%%%%%%%%%%%%%%%%%%%%%
a. the spring has $100$ Joules of energy

The spring is small, so the change in height of the ball as the spring uncoils can be neglected.

all of the spring's energy is converted into initial kinetic energy of the ball

spring energy = (1/2) $\times$ ball mass $\times$ $v_{0}^{2}$
			  = (1/2) $\times$ (0.5 kg) $\times$ $(20 m/s)^{2}$


b. the max ball height is $20$ meters

the ball moves straight up and down, and the velocity of the ball at its maximum height is zero.
Using this velocity information, and knowing that the acceleration which is causing the ball
to slow down is gravity, the max ball height can be related to the gravitational acceleration,
initial and final velocities as follows

$2 \times$ g $\times$ max height = $v_{f}^{2}$ - $v_{0}^{2}$ = -$v_{0}^{2}$
max height = $ -\frac{v_{0}^{2}}{2g} $ = $-\frac{(20 m/s)^{2}}{2(-9.81 m/s/s)}$


c. the ball is in the air for $4$ seconds

the time elapsed from when the ball leaves the spring, to when it reaches its maximum height
and velocity of $0$ is half of the total time in which the ball is in the air.
If $t_{m}$ is the time elapsed from when the ball leaves the spring to when it reaches its
maximum height, then

v($t_{m}$) = $0 m/s$ = $v_{0}$ + accel $\times t_{m} \rightarrow$ $t_{m}$ = 

%%%%%%%%%%%%%%%%%%%%%%%%%%%%%%%%%%%%%%%%%%%%%%%%%%%%%%%%%%%%%%%%%%%%%%%%%%%%%%%%
