%%%%%%%%%%%%%%%%%%%%%%%%%%%%%%%%%%%%%%%%%%%%%%%%%%%%%%%%%%%%%%%%%%%%%%%%%%%%%%%%
% data_set.tex:
%%%%%%%%%%%%%%%%%%%%%%%%%%%%%%%%%%%%%%%%%%%%%%%%%%%%%%%%%%%%%%%%%%%%%%%%%%%%%%%%
\chapter{Background Estimation}
\label{sec:backgroundEstimation}
%%%%%%%%%%%%%%%%%%%%%%%%%%%%%%%%%%%%%%%%%%%%%%%%%%%%%%%%%%%%%%%%%%%%%%%%%%%%%%%%
As discussed in the previous chapter, the event selection criteria were applied to the data to select events where two leptons and jets were 
reconstructed, and had kinematics consistent with the \WR decay progeny.  However, these criteria also selected data events 
produced by ST background interactions, like \DY+jets.  The contributions of ST backgrounds to the $\Mlljj$ distribution 
found in the data were predicted using Monte Carlo (\MC) simulations and control regions with no \WR signal contamination.  
The magnitudes of the individual backgrounds, how the backgrounds were simulated, and how the control regions were defined 
and used is described in this chapter.

The magnitude of the background produced by each ST process is proportional to the product of the cross section and branching ratio 
to $\ell\ell jj$ final states at $\sqrt{s} =$ 13 $\TeV$.  The production of jets in association with the \DY process, and the production 
of top quarks yielded backgrounds with the largest magnitudes.  The production of diboson (WW, WZ, ZZ) pairs, single W bosons in 
association with jets, and multiple jets through QCD also yielded backgrounds, but with significantly lower magnitudes.

The \DY process is initiated by a quark and anti-quark that annihilate into a $Z/\gamma^{*}$, which decays promptly into two opposite 
charge, same flavor leptons, as shown in Figure \ref{fig:dyDiags}.  The production of dilepton pairs through \DY has a cross section-
branching ratio product that peaks at 6000 pb for dilepton mass $\Mll \approx 90$ $\GeV$, and decreases for higher values of $\Mll$.  
At $\sqrt{s} = 13$ $\TeV$ the quarks that initiate the \DY interaction radiate two partons with a probability of $\sim$0.01, 
equal to the QCD coupling squared $\alpha_{QCD}^{2}$.  Therefore, in $\sim$1\% of \DY events two partons are radiated and hadronize 
into jets; the \DY process produces $\ell\ell jj$ final states with a cross section-branching ratio product of $\approx$60 pb.

Unlike \DY, the production of $t\bar{t}$ quark pairs, shown in Figure \ref{fig:ttbarDiag}, produces $\ell\ell jj$ final states without 
initial state parton radiation.  The product of the cross section and branching ratio, 86 pb, is similar to the product of the cross 
section and branching ratio of the \DY+2 jets process.  Since more than 99\% of top quarks decay to a $W$ boson and bottom quark, the 
production of single top quarks with a W boson (Figure \ref{fig:singleTopDiags}) yields two leptons and one jet when both $W$ bosons 
decay leptonically.  The gluon that initiates the top+W interaction radiates a gluon with $\sim$100\% probability, so the production 
of top+W yields two lepton and two jet final states with a cross section-branching ratio product of $\sim$7 pb.  The production of 
single top quarks through other processes, shown in Figure \ref{fig:singleTopDiags}, yield only one $W$, and therefore do not produce 
$\ell\ell jj$ final states at leading order in the electroweak coupling.

The only other processes that produce two leptons and jets without initial state parton radiation are the production of WZ and ZZ pairs.  
The WZ production yields two leptons and jets when the $W$ decays hadronically, and the $Z$ decays to charged leptons.  The ZZ process 
produces two leptons and jets when one $Z$ decays hadronically, and the other decays to charged leptons.  The combined product of the 
cross section and branching rato of the WZ and ZZ processes to $\ell\ell jj$ final states is $\sim$3 pb, negligible compared to the 
production of top quarks.

Other processes contributed to the total background, but at negligible levels.  The production of WW boson pairs yields two charged 
leptons when both $W$ bosons decay leptonically.  The two quarks that initiate the WW process radiate two partons with $\sim$1\% 
probability, so the WW process produces $\ell\ell jj$ final states with a cross section-branching ratio product of $\sim$0.1 pb, 
negligible compared to other backgrounds.  Although the $W$+jets and QCD multi-jet processes do not produce $\ell\ell$ final states 
at leading order in the electroweak coupling, a small fraction of their jets, less than 0.1\%, are incorrectly reconstructed as charged 
leptons.  Since the $W$+jets and QCD multi-jet processes produce multiple jets with cross section-branching ratio products in excess of 
several hundred pb \cite{wJetsMeas,jetProductionMeas}, they contributed to the $\Mlljj$ distribution found in data, but at a negligible 
level.  

In conclusion, the \DY+jets and top quark processes produced the largest backgrounds, and other processes produced significantly 
smaller backgrounds.  The shape and magnitude of the $\Mlljj$ distribution produced by these processes was estimated using Monte Carlo 
(\MC) simulations.
\clearpage

\begin{figure}
	\centering
	\begin{subfigure}[t]{2.4in}
		\centering
		\includegraphics[width=2.4in]{figures/dyNoJetFeynDiagram.png}
	\end{subfigure}
	\thickspace
	\begin{subfigure}[t]{2.4in}
		\centering
		\includegraphics[width=2.4in]{figures/dyThreeJetFeynDiagram.png}
	\end{subfigure}
	\caption{Feynman diagrams for the \DY interaction with 0 radiated partons, and 3 radiated partons \cite{dyDiagrams}.}
	\label{fig:dyDiags}
\end{figure}

\begin{figure}[h]
	\centering
	\includegraphics[width=0.45\textwidth]{figures/topAntiTopFeynDiagram.png}
	\caption{$t\bar{t}$ Feynman diagram \cite{ttbarDiagram}.}
	\label{fig:ttbarDiag}
\end{figure}

\begin{figure}[h]
	\centering
	\includegraphics[width=0.7\textwidth]{figures/singleTopQuarkFeynDiagrams.png}
	\caption{Single top quark Feynman diagrams \cite{singleTopQrkDiagrams}.}
	\label{fig:singleTopDiags}
\end{figure}

\clearpage

\section{Monte Carlo and Corrections}
\label{sec:MC}
Background processes were simulated in three steps.  The first step used one or two \MC generators to simulate the interaction between 
colliding protons, the decay of unstable particles, and the hadronization of partons leaving the interaction.  Particles produced by the 
interaction interacted with the sub-detectors, and the signals that were generated in the sub-detectors were simulated in the second step.  
In the third step, the reconstruction algorithms processed the simulated signals to reconstruct leptons and jets.

In the first step, one \MC generator simulated the interaction between colliding protons, and the decay of unstable particles.  
This generator was chosen so that the strengths of the generator matched the characteristics of the interaction.  The \MADGRAPH generator 
\cite{madgraph} simulates an interaction at leading order in the electroweak coupling with up to 4 additional partons radiated from the 
interaction.  The \POWHEG generator \cite{powheg} simulates an interaction at next-to-leading order in the electroweak and QCD couplings 
with up to one additional parton radiated from the interaction.  Lastly, the \PYTHIA generator \cite{pythia8,Sjostrand:2006za} simulates 
an interaction at leading order in the electroweak and QCD couplings with up to one additional parton radiated from the interaction.  The 
\MADGRAPH generator was used to simulate the processes with the highest products of the cross section and branching ratio - \DY+jets, 
$t\bar{t}$ pair production, and W+jets production.  The production of single top quarks with a W boson has a cross section-branching 
ratio product to $\ell\ell jj$ final states that increases by 7\% or more by going from leading order to next-to-leading order 
\cite{singleTopNLOvsLO}.  For this reason, \POWHEG was used to simulate the single top quark processes to more accurately estimate the 
background produced by single top quark processes.  Similar to the single top quark processes, the production of diboson pairs has 
a cross section-branching ratio product to $\ell\ell jj$ final states that increases, by up to 45\%, going from leading order to next-to-
leading order \cite{dibosonLOvsNLO}.  However, the product of the cross section and branching ratio to $\ell\ell jj$ final states for the 
diboson processes at next-to-leading order is still negligible compared to \DY+jets and top quark processes, so \PYTHIA was used to 
simulate the diboson processes.

The hadronization of partons was simulated separately with a single \MC generator.  Background processes that are simulated with any generator 
show the best agreement with experimental data when \PYTHIA is used to simulate parton hadronization \cite{pythiaForHadronization}, so 
all simulations used \PYTHIA and the NNPDF23 PDF set \cite{nnpdf} to simulate parton hadronization.  Excluding the partons, the generator 
that simulated the interaction between protons was also used to simulate the decay of unstable particles to quasi-stable and stable 
particles, like the $\Sigma^{\pm}$ and $\gamma$, that travel a mean distance c$\tau \gtrsim 2.4$ cm before decaying.  Particles that travel 
a mean distance of 2.4 cm have a small probability to interact with the first silicon pixel tracker layer located 4.4 cm from the 
IP \cite{cmsTdrPhysPerformance}, so they were not decayed before the detector response was simulated.


%SAVE THIS statement that explains why the aMCatNLO inclusive diboson simulated datasets were not used
%A second set of diboson$\to$leptons+jets events was simulated with a next-to-leading order \MC generator, but these events were 
%generated with positive and negative event weights.  Due to the presence of positive and negative event weights, in some $\Mlljj$ 
%regions the predicted number of diboson events was negative, and the prediction's statistical uncertainty was greater than 100\%.

The large instantaneous luminosity of the LHC beams produced multiple pp interactions (pileup) in each event in data.  In the second 
step, the pileup interactions were simulated, and GEANT4 \cite{geant4}, which is a general purpose detector simulation program, was used 
to simulate the propagation and decay of quasi-stable and stable particles, and their interactions with the detector.  The pileup was 
simulated by sampling a random integer $X$ from a Poisson distribution with a mean of 12, and mixing $X$ simulated minimum bias events 
into the event that were simulated through the first simulation step.  The distributions of reconstructed particle multiplicity and 
kinematics found in minimum bias events in data show the best agreement with those found in events simulated with \PYTHIA 
\cite{pythiaForHadronization}, so \PYTHIA was used to simulate minimum bias events.  After adding minimum bias events, GEANT4 propagated 
all particles through the 3.8 $\unit{T}$ magnetic field, and simulated their interactions with the detector and the resulting signals.  
In addition, GEANT4 also simulated the decays of quasi-stable particles to the following stable particles - 
$\gamma,p^{\pm},n^{0},\bar{n}^{0},\nu,e^{\pm}$.

In the third step, particle reconstruction algorithms reconstructed particles and vertices using the signals in the detector that were 
simulated by GEANT4.  The trigger selection criteria were applied and the final decision of each trigger algorithm was saved in each 
event, but no events were discarded.

After the third step, corrections were applied to simulated event weights.  As discussed 
in Chapter \ref{sec:reco_chapter}, the efficiencies of reconstruction algorithms and selection criteria differed between data and 
simulations.  These efficiency differences were resolved by changing the weight of each simulated event by up to $\pm$7\%, depending 
on the kinematics of the selected leptons.  Independent of the kinematics of the reconstructed particles, the weight of each simulated 
event was normalized to the integrated luminosity of the data, and adjusted further to match the pileup distribution found in the data.  
The simulated pileup distribution is a Poisson distribution with a mean of 12, but the pileup distribution found in the data is better 
represented by a Poisson distribution with mean 14 \cite{lumi}.  The discrepancy between the data and simulated pileup distributions was 
corrected by changing the weight of each simulated event, on average by $\sim$5\%.

Reconstructed leptons and jets were measured with different energies in data and simulated events, so energy corrections were applied to 
particles that were reconstructed in simulated events.  Energy corrections for simulated muons, electrons, and jets were derived using 
$Z \rightarrow \ell\ell$, Z+jet, dijet and $\gamma$+jet events from simulations and data.  The di-lepton and di-jet mass distributions 
found in these events were compared between data and simulations, and the energies of simulated leptons and jets were corrected so that 
the distributions matched.  The average correction was 1\% of a muon's $\pt$, 1\% of an electron's $\Et$, and 6\% of a jet's $\pt$.

The processes that were simulated and the corresponding number of simulated events is summarized in Table \ref{tab:centrallyProducedMC}.  
The number of simulated events is expressed in units of fb$^{-1}$, and should be compared to the 2.6 fb$^{-1}$ of data that was collected.

\begin{table}[bt]
	\caption{A summary of the background processes and the sizes of the simulated datasets.  The "Size" of a dataset is equal 
	to the number of simulated events of a specific process divided by the product of the cross section and branching ratio.}
\label{tab:centrallyProducedMC}

\centering
\resizebox{\textwidth}{!}{
	\begin{tabular}{ |c|c|c|c| } 
	\hline
	Dataset         & Step 1 Generator & cross section (pb) & Size (fb$^{-1}$)   \\
		\hline
		Inclusive DY+jets, $DY \rightarrow ll$ & \MADGRAPH   & 5991    & 1.51 \\ \hline
		DY+jets HT 100-200, $DY \rightarrow ll$ & \MADGRAPH   & 181.3    & 15.0 \\ \hline
		DY+jets HT 200-400, $DY \rightarrow ll$ & \MADGRAPH   & 50.42    & 19.3 \\ \hline
		DY+jets HT 400-600, $DY \rightarrow ll$ & \MADGRAPH   & 6.984    & 153. \\ \hline
		DY+jets HT $>$ 600, $DY \rightarrow ll$ & \MADGRAPH   & 2.704    & 369. \\ \hline
		t$\bar{t}$+jets $\rightarrow ll$+jets & \MADGRAPH  & 85.67    & 286. \\ \hline
		single t $\rightarrow$ leptons+jets  & \POWHEG & 80.95 & 20.8 \\ \hline
		single $\bar{t}$ $\rightarrow$ leptons+jets & \POWHEG & 136.0 & 24.3 \\ \hline
		$\bar{t}$+W $\rightarrow$ all   & \POWHEG & 35.85 & 27.6 \\ \hline
		t+W $\rightarrow$ all   & \POWHEG & 35.85 & 27.8 \\ \hline
		WW $\rightarrow$ all  & \PYTHIA & 113.8   & 8.73   \\ \hline
		ZZ $\rightarrow$ all  & \PYTHIA & 10.15   & 98.2   \\ \hline
		WZ $\rightarrow$ all  & \PYTHIA & 23.4   & 41.8   \\ \hline
		W+jets $\rightarrow l\nu$+jets & \MADGRAPH & 50270   & 1.44 \\ \hline
		%$\WR \rightarrow l\nul \rightarrow lljj$  & \PYTHIA & 1$\times 10^{-5}$ - 4.3 & 5$\times 10^{6}$ - 11.6   \\ \hline
		\end{tabular}
}
\end{table}


\section{Top Quark Background}
\label{sec:topQrkBkgnds}
As discussed in Chapter \ref{sec:lrsPhenomenology}, due to lepton flavor conservation the \WR cannot decay to final states with an 
electron and a muon.  Therefore, $e\mu jj$ events found in the data were produced only by background processes.  The production of 
a $t\bar{t}$ quark pair and a top quark with a W boson yields events with two partons and two W bosons.  Since a W boson decays to 
an electron or a muon with equal branching ratios, the top quark processes produce the $e\mu jj$ final state twice as often as the 
$eejj$ or $\mu\mu jj$ final states.  As no other ST processes produce the $e\mu jj$ final state at leading order in the electroweak 
coupling, the majority of $e\mu jj$ events found in data were produced by top quark processes.  The $e\mu jj$ final state was used 
as a control region to estimate the top quark background.

During collisions, $e\mu jj$ events were selected using a trigger that required one muon and one electron.  Leptons and jets 
that were reconstructed offline were selected with additional criteria.  Events that met the selection criteria, described in detail 
in Appendix \ref{app_trgOfflId}, had one electron, one muon, and at least two jets with the characteristics described previously in 
Chapter \ref{sec:onlineAndOfflineIdSel}.  However, the electron had an $\Et > 30$ $\GeV$, and the muon had a $\pt > 30$ $\GeV$.

In events selected by the trigger, leptons and jets were reconstructed, and events were selected using the event selection described in 
Chapter \ref{sec:reco_chapter} applied to the electron and muon selected by the trigger.  

These selection criteria were applied to simulated background events, and compared to the selected data events in Figure 
\ref{fig:dataAndSimsInEMuChannel}.  Comparing the magnitudes of different simulated backgrounds, the top quark background produces more 
than 98\% of selected $e\mu jj$ events, as expected.

\begin{figure}[h]
	\centering
	\includegraphics[width=0.7\textwidth]{figures/Mlljj_eMuChannel_log.pdf}
	\caption{The $\Mlljj$ distribution from data and simulated ST events that passed the $e\mu$ selection criteria, excluding 
	the $\Mlljj > 600 \GeV$ cut.  The bin widths were variable, and their contents were normalized to the bin widths.}
	\label{fig:dataAndSimsInEMuChannel}
\end{figure}

The $\Memujj$ distribution found in selected data events was scaled by one factor to estimate the top quark background in the $ee$- 
and $\mu\mu$-channels.  The factor is the ratio of $\frac{\ell\ell}{e\mu}$ production, which is 0.5 based on lepton universality in 
the decay of the W boson, multiplied by the ratio of the electron and muon selection efficiencies $\frac{e}{\mu}$, which is below 1 
due to the reduced acceptance of the ECAL relative to the muon detectors, and is not known $a$-$priori$.  The factor and its 
variation with $\Mlljj$ was estimated using simulated top quark events.  To make this estimation, simulated events were selected 
using the $e\mu$-, $ee$-, and $\mu\mu$-channel selection criteria.  Then, the $\Mlljj$ distribution found in each set of events was 
split into variable width bins such that each bin had the same number of events.  The first bin covered $600 < \Mlljj \leq 625$ $\GeV$, 
and the last bin covered $\Mlljj > 1160$ $\GeV$.  Then, the integral of each bin was calculated for all three distributions.  Finally, 
the integrals of the $\Meejj$ and $\Mmumujj$ bins were divided by the integrals of the $\Memujj$ bins.  The result, shown in Figure 
\ref{fig:ttbarSFratios}, represents the factor used to estimate the top quark background as a function of 
$\Mlljj$.  The factor within its statistical uncertainty is independent of $\Mlljj$, and is equal to 0.659 for $\Mmumujj / \Memujj$, 
and 0.432 for $\Meejj / \Memujj$.  It is assumed that the $\Mlljj$ distribution shape found in top quark events is independent of the final 
state lepton flavor.  The top quark contributions to the $\Meejj$ and $\Mmumujj$ distributions found in data were estimated by scaling 
the $\Memujj$ distribution found in data by 0.432 and 0.659, respectively.

\begin{figure}
	\centering
	\begin{subfigure}[t]{2.4in}
		\centering
		\includegraphics[width=2.4in]{figures/flavor_ratio_EE_variablebinwidth.pdf}
	\end{subfigure}
	\thickspace
	\begin{subfigure}[t]{2.4in}
		\centering
		\includegraphics[width=2.4in]{figures/flavor_ratio_MuMu_variablebinwidth.pdf}
	\end{subfigure}
	\caption{The bin-by-bin ratio of the $\Mlljj$ and $\Memujj$ distributions from simulated top quark backgrounds, where 
		$\ell$ is an electron on the left, and a muon on the right.}
	\label{fig:ttbarSFratios}
\end{figure}

Although the values 0.659 and 0.432 were calculated using all simulated events with $\Mlljj > 600$ $\GeV$, 
the majority of selected simulated events had $\Mlljj < 1500$ $\GeV$.  A 10\% uncertainty was assigned to the top quark background 
estimate to cover any deviation from 0.659 or 0.432 at high $\Mlljj$.  Based on the data and simulated events shown in Figure 
\ref{fig:dataAndSimsInEMuChannel}, the non-top quark backgrounds contributed $\sim$1\% to the $\Memujj$ distribution found in data.  
Their contribution was neglected because it was less than the 10\% top quark background uncertainty.


\section{\DY Background}
\label{sec:dyBkgnd}
The \DY+jets process produced $\ell\ell jj$ events at similar rates to the top quark processes.  Since the \WR decay does not produce 
events with an electron and a muon, the top quark background was estimated directly from data in the $e\mu jj$ control region.  No 
analogue of the $e\mu jj$ control region exists for the \DY+jets background, so it needed to be estimated using simulated events.  
The model of the \DY+jets process used in simulations is an approximation, and the differences between the \DY predictions and the data 
were estimated by comparing simulated background events to data events in three control regions.

\subsection{\DY normalization in $\Mlljj$}
\label{sec:dyNormInMlljj}
The approximations made in the \DY+jets model result in a difference between the data and the simulations in the normalization of the 
$\Mlljj$ distribution.  The size of this difference was estimated using the $Z \rightarrow \ell\ell$ control region.

The data and simulated \DY+jets events were compared in the $Z \rightarrow \ell\ell$ control region, where the majority of events found 
in the data were produced by the \DY+jets process.  During collisions, events that had two electrons or one muon were selected using the 
triggers described in Appendix \ref{sec:trgDyCR}.  Offline, leptons and jets were reconstructed, and additional selection criteria were 
applied.  Selected events had two leptons and jets with the following characteristics:

\begin{itemize}
	\item Each lepton had a $\pt > 35$ $\GeV$, and an $|\eta| < 2.4$.
	\item The two leptons had a di-lepton mass $70 < \Mll < 110$ $\GeV$.
	\item Each jet had a $\pt > 40$ $\GeV$, and an $|\eta| < 2.4$.
	\item Each jet was separated from both leptons by $\Delta R > 0.4$.
\end{itemize}

Electrons reconstructed in simulated events passed the trigger criteria with a different efficiency than electrons reconstructed in 
data events.  This efficiency difference was corrected by multiplying the weight of every simulated event by a value that depended on the 
$\Et$ and $\eta$ of the highest $\Et$ electron selected by the trigger.  The average value of the correction was 0.96.

The $\Mll$ distribution in data for $Z \rightarrow \ell\ell$ events was compared with the distribution of the same events in \DY+jets 
simulations.  The normalization of the $\Mll$ distribution in data and simulations was compared by 
integrating the distribution in data and simulated events, and comparing the integral values.  The normalization of the $\Mll$ distribution 
measured in data exceeded the normalization measured in simulated events by 15.7\% in the $ee$-channel, and 14.2\% in the $\mu\mu$-channel.  
As a result, the weight of simulated \DY+jets events was increased by 15.7\% in the $ee$-channel, and 14.2\% in the $\mu\mu$-channel to 
more accurately predict the \DY background.

The uncertainty on the $\sim$15\% \DY normalization correction was estimated using simulated \DY+jets events produced using two other \MC 
generators.  One generator simulates the \DY process at next-to-leading order in the electroweak and QCD couplings and radiates up to 
four partons.  The other generator was \POWHEG, which simulates the \DY process and radiates up to one parton.  Events from data and \DY+
jets simulations using all three generators were selected using the $Z \rightarrow \ell\ell$ selection criteria without jet requirements.  
The jet requirements were removed to compare data to simulations in a phase space where \POWHEG yielded a large number of selected events, 
comparable to the simulations produced with the other generators.  The normalization of the $\Mll$ distribution in data and simulations 
was compared by integrating the distribution measured in data and simulated events, and comparing the integral values.  Since no jet 
requirements are applied the simulations are expected to match the data, and any deviation from this expectation was taken as an 
uncertainty.  The largest difference in the normalization measured in data and any of the simulated events was 2.0\% in the $ee$-channel, 
and 1.0\% in the $\mu\mu$-channel.  This difference was taken as an uncertainty on the number of \DY background events that were predicted 
in each channel.

Based on simulations of all ST processes, the contribution of ST processes excluding \DY+jets to the data events selected in the 
$Z \rightarrow \ell\ell$ control region was 2\% of the total background that was predicted in the control region.  This 2\% is comparable 
to the \DY background normalization uncertainty, so the contribution of other backgrounds to the data in the control region was neglected.

\subsection{\DY shape in $\Mlljj$}
\label{sec:dyShapeInMlljj}
The approximations made in the \DY+jets model also cause the shape of the $\Mlljj$ distribution to differ between data and simulated 
\DY+jets events.  The size of this shape difference was estimated using the low $\Mll$ control region.

%old material, no longer needed
%The \DY+jets interaction was simulated with up to four radiated partons leaving the \DY interaction, and the \DY+jets interaction in 
%real collisions radiated any number of partons, with no upper limit.  As a result, the radiated parton $\pt$,$\eta$, and multiplicity 
%distributions found in \DY+jets events were not expected to agree between data and simulations.  More than 50\% of radiated partons 
%that had $|\eta| < 2.5$ and $\pt > 10$ $\GeV$ were reconstructed as jets \cite{pflowEventReco}, so the reconstructed jet kinematic 
%distributions found in data and simulated events were not expected to agree.  Therefore, the shapes of the $\Mlljj$ distributions found in 
%real and simulated \DY+jets events were expected to differ.

The data and simulated events of all ST processes were compared in the low $\Mll$ control region, where the majority of events found 
in the data were produced by the \DY+jets process.  Data and simulated events were selected using the event selection described 
previously in Chapter \ref{sec:reco_chapter}, but the two selected leptons were required to have a $\Mll < 180$ $\GeV$.  Based on 
the normalization correction described in Section \ref{sec:dyNormInMlljj}, the weight of events selected in \DY+jets 
simulations was increased by 15.7\% in the $ee$-channel, and 14.2\% in the $\mu\mu$-channel.  Then, the $\Mlljj$ distribution 
measured in data and simulated events were compared.  The size of the $\Mlljj$ shape difference was calculated as the largest difference 
between the data and the total background in any bin shown in Figure \ref{fig:mlljjLowMllCR}.  The maximum difference, 40\% for 
$\Mlljj > 1.9$ $\TeV$ in both channels, was taken as the shape difference for events with $\Mlljj > 0.6$ $\TeV$.

\begin{figure}
	\centering
	\begin{subfigure}[t]{2.4in}
		\centering
		\includegraphics[width=2.4in]{figures/Mlljj_eeChnl_lowMllCR.png}
	\end{subfigure}
	\thickspace
	\begin{subfigure}[t]{2.4in}
		\centering
		\includegraphics[width=2.4in]{figures/Mlljj_mumuChnl_lowMllCR.png}
	\end{subfigure}
	\caption{The $\Mlljj$ distributions from data and simulated background events that passed the low $\Mll$ control region selection 
		criteria.  The $ee$-channel is on the left, and the $\mu\mu$-channel on the right.  The bin widths are variable, and the bin 
	contents are normalized to their widths.}
	\label{fig:mlljjLowMllCR}
\end{figure}

Initially the large shape difference was attributed to limitations of the \DY+jets simulation - the \DY process was simulated 
only at leading order in the electroweak and QCD couplings.  This was tested by simulating a separate set of 
\DY+jets events using a different \MC generator at next-to-leading order in the electroweak and QCD couplings, and with up to four 
partons radiated from the initial state quarks.  The procedure described in Section \ref{sec:dyNormInMlljj} was repeated to 
re-calculate a \DY+jets normalization correction for the simulated \DY events.  The $\Mlljj$ distribution measured in simulated events, 
compared to data in Figure \ref{fig:mlljjLowMllCRAMC}, shows a larger difference with the data when using the next-to-leading order 
\DY+jets simulation than with the leading order simulation.  The larger difference was therefore not caused by higher order QCD or 
electroweak interactions, but by a significant decrease in selected events.  Comparing the leading order 
and next-to-leading order \DY+jets simulations after applying the low $\Mll$ selection criteria, the next-to-leading order simulation 
produced a factor of $\sim$3 fewer events with $\Mlljj > 0.6$ $\TeV$, and a factor of $\sim$10 fewer events with $\Mlljj > 1$ $\TeV$.  
Similar deficits in selected events were found in events that had $\Mlljj > 600$ $\GeV$ and $\Mll > 200$ $\GeV$.  For these reasons the 
next-to-leading order \DY+jets simulation was not used to estimate the \DY+jets background, and the large shape difference could not 
be attributed to next-to-leading order effects.  Instead, it was investigated if the shape difference should be applied as a correction 
or assigned as an uncertainty.

The effect of the 40\% $\Mlljj$ shape difference on the \DY background prediction can be accounted for in two ways.  The simulated \DY+jets 
event weights can be adjusted to match the $\Mlljj$ distribution found in data, or an uncertainty can be assigned to the \DY background 
prediction.  Correcting the simulated event weights to match the data was not done for two reasons.  First, the variation of the 
$\Mlljj$-dependent correction versus $\Mll$ could not be checked.  This was a significant concern in particular for events that had 
$\Mlljj > 1900$ $\GeV$, because the weight of those events would have been corrected by 40\%.  Secondly, the uncertainty on the correction 
was dominated by the statistical uncertainty of the data, which exceeded 30\% for $\Mlljj > 1900$ $\GeV$.  Since the correction had a 
large statistical uncertainty and could not be validated in events with $\Mlljj > 600$ $\GeV$ and $\Mll > 200$ $\GeV$, the effect of 
the 40\% shape difference was accounted for by assigning an uncertainty to the \DY prediction.

The magnitude of the uncertainty was 
determined by counting the number of predicted \DY+jets events in bins of $\Mlljj$.  Each bin is linked to a specific \mWR hypothesis, 
and the shape of the \DY $\Mlljj$ distribution in each bin is irrelevant.  Thus a shape difference and a normalization difference have 
the same effect - the total number of events in the bin change by the magnitude of the difference.  Therefore, the effect of the shape 
difference was accounted for by assigning a 40\% uncertainty to the normalization of the $\Mlljj$ distribution measured in simulated 
\DY+jets events, independent of $\Mlljj$.

\begin{figure}
	\centering
	\begin{subfigure}[t]{2.4in}
		\centering
		\includegraphics[width=2.4in]{figures/Mlljj_eeChnl_lowMllCR_AMCNLO.pdf}
	\end{subfigure}
	\thickspace
	\begin{subfigure}[t]{2.4in}
		\centering
		\includegraphics[width=2.4in]{figures/Mlljj_mumuChnl_lowMllCR_AMCNLO.pdf}
	\end{subfigure}
	\caption{The $\Mlljj$ distribution found in data and simulated background events that passed the low $\Mll$ control region selection 
		criteria.  The $ee$-channel is on the left, and the $\mu\mu$-channel is on the right.  The bin contents are normalized to their 
	widths.}
	\label{fig:mlljjLowMllCRAMC}
\end{figure}

The low $\Mlljj$ control region was used to validate the $\sim$15\% correction applied, and the 40\% uncertainty assigned to the \DY 
prediction.  Events from data and all background simulations were selected using the same selection criteria as the low $\Mll$ 
control region, but with $\Mll > 200$ $\GeV$ and $\Mlljj < 600$ $\GeV$.  The weight of simulated \DY events was increased by $\sim$15\%, and 
then the $\Mll$ distributions found in selected data and simulated background events were compared.  The comparison, shown in Figure 
\ref{fig:mllInLowMlljjSideband}, shows that the $\sim$15\% \DY correction brought the background estimate into better agreement with the 
data.  In addition, the 40\% \DY uncertainty was not too conservative, because the disagreement between data and estimated backgrounds 
approached 40\% in several bins.

\begin{figure}
	\centering
	\begin{subfigure}[t]{2.4in}
		\centering
		\includegraphics[width=2.4in]{figures/Mll_eeChnl_lowMlljjCR.png}
	\end{subfigure}
	\thickspace
	\begin{subfigure}[t]{2.4in}
		\centering
		\includegraphics[width=2.4in]{figures/Mll_mumuChnl_lowMlljjCR.png}
	\end{subfigure}
	\caption{The $\Mll$ distribution found in data and simulated background events that passed the low $\Mlljj$ control region 
		selection criteria.  The $ee$-channel is on the left, and the $\mu\mu$-channel is on the right.}
	\label{fig:mllInLowMlljjSideband}
\end{figure}

\subsection{\DY summary}
The \DY contribution to the $\Mlljj$ distribution found in data was estimated by selecting simulated \DY events using the selection criteria 
described in Chapter \ref{sec:reco_chapter}.  The weight of each selected event was increased by 15.7\% the $ee$-channel, and by 14.2\% in 
the $\mu\mu$-channel.  When calculating the results, a 40\% uncertainty was assigned to the \DY prediction.


\section{Diboson and W+jets Backgrounds}
\label{sec:dibosonAndWJetsBkgnds}
The diboson (WW, WZ, ZZ) and W+jets processes produced $\ell\ell jj$ events at a much lower rate than the \DY+jets process.  No 
control region existed where the diboson or W+jets backgrounds could be extracted directly from data, so their contributions to the 
$\Mlljj$ distributions found in data were estimated using simulated events.  The $\Mlljj$ distribution measured in selected diboson and 
W+jets events (Figure \ref{fig:allExpectedBkgnds}) was concentrated in the $\Mlljj \leq 2.0$ $\TeV$ region, and its integral was less 
than 3\% of the total \DY and top quark prediction.  For these reasons the diboson and W+jets contributions to the $\Mlljj$ distribution 
in the data were neglected.

\begin{figure}
	\centering
	\begin{subfigure}[t]{2.4in}
		\centering
		\includegraphics[width=2.4in]{figures/useOfLLJJMassAsFigureOfMerit.pdf}
	\end{subfigure}
	\thickspace
	\begin{subfigure}[t]{2.4in}
		\centering
		\includegraphics[width=2.4in]{figures/Mlljj_mumuChnl_signalRegionNoData.pdf}
	\end{subfigure}
	\caption{The $\Meejj$ (left) and $\Mmumujj$ (right) distributions found in selected signal and background events.  The top 
		quark and QCD backgrounds are estimated using data. The \WR $\Mlljj$ distribution normalization is reduced by 70\%.}
	\label{fig:allExpectedBkgnds}
\end{figure}


\section{QCD Background}
\label{sec:qcdBkgnd}
Similarly, the QCD multi-jet processes produced $\ell\ell jj$ events at a much lower rate than the \DY+jets process.  Unlike the 
other backgrounds, the leptons selected offline in QCD events were incorrectly reconstructed from jets that contained energetic 
electrons or photons.  Events in data where two jets may have been identified as leptons were selected using the online and offline 
selection criteria described in Chapter \ref{sec:reco_chapter}, but using the following (reduced) lepton ID selection criteria:

\textbf{Muons}
\begin{itemize}
	\item The silicon tracker track was reconstructed from signals in at least 1 silicon pixel detector layer, and signals in at least 
		5 layers in the entire tracker.
\end{itemize}

\textbf{Electrons}
\begin{itemize}
	\item The electron track was reconstructed from signals in every silicon pixel and inner strip detector layers, or all but 1 layer.
	\item The electron track's origin was separated from its vertex by a small distance $\Delta_{xy}$ in the $x-y$ 
		plane: $\Delta_{xy} < 0.2$ mm in the tracker barrel, and $\Delta_{xy} < 0.5$ mm in the tracker endcap.
\end{itemize}

Events were rejected if one or both selected leptons passed the standard lepton ID selection criteria.  In the selected events, 
the $\pt$'s and $\eta$'s of both selected leptons were used as inputs to a $\pt,\eta$-dependent probability function.  This function, 
derived elsewhere \cite{ZprimeRunOneAndTwo}, calculates the probability that a jet reconstructed as a lepton will pass the standard lepton 
ID selection criteria\footnote{The form of the function differed for electrons and muons.}.  The probability calculated for both selected 
leptons was applied to each selected event as a weight.  The $\Mlljj$ distribution found in selected weighted events (Figure 
\ref{fig:allExpectedBkgnds}) was negligible compared to other backgrounds, so the QCD background was neglected.

\section{Background Estimation Summary}
ST processes produced events in data where two leptons and jets were reconstructed, and passed the selection criteria that was 
designed to select $\WR \rightarrow \ell\ell jj$ events.  The $\Mlljj$ distribution produced by each interaction, and the uncertainty on 
its normalization was estimated using data and simulated events in control regions.  The \DY+jets and top quark processes were estimated 
to give more than 97\% of the background, so the sum of the \DY and top quark backgrounds was identified as the total predicted 
background.  This was compared to data in bins of $\Mlljj$ that were linked to specific \mWR hypotheses.  These bins 
and the comparison between the data and predicted background is discussed in the next chapter.


%%%%%%%%%%%%%%%%%%%%%%%%%%%%%%%%%%%%%%%%%%%%%%%%%%%%%%%%%%%%%%%%%%%%%%%%%%%%%%%%
