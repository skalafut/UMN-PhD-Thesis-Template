%%%%%%%%%%%%%%%%%%%%%%%%%%%%%%%%%%%%%%%%%%%%%%%%%%%%%%%%%%%%%%%%%%%%%%%%%%%%%%%%
% data_set.tex:
%%%%%%%%%%%%%%%%%%%%%%%%%%%%%%%%%%%%%%%%%%%%%%%%%%%%%%%%%%%%%%%%%%%%%%%%%%%%%%%%
\chapter{Background Estimation}
\label{sec:backgroundEstimation}
%%%%%%%%%%%%%%%%%%%%%%%%%%%%%%%%%%%%%%%%%%%%%%%%%%%%%%%%%%%%%%%%%%%%%%%%%%%%%%%%
The online and offline event selection criteria were applied to the data to select events where two leptons and jets were 
reconstructed, and had kinematics consistent with $\WR \rightarrow \ell\ell jj$ decay progeny.  However, applying these 
selection criteria also selected data events produced by ST background interactions like $\DY$+jets.  
The contribution of ST backgrounds to the $\Mlljj$ distribution found in the data were predicted using Monte Carlo (\MC) 
simulations, and control regions with no \WR signal contamination.  The magnitudes of individual backgrounds, how 
they were simulated, and how the control regions were used to estimate the background contributions are described here.

%\section{Background Magnitudes}
%\label{sec:bkgndMags}
The .

%DESCRIBE magnitudes of backgrounds using txt written in reco+selection chapter

\section{Monte Carlo}
\label{sec:MC}
%Events from \WR processes, characterized by different \mnul and \mWR, and backgrounds are simulated in three steps.
%\PYTHIA also provides a flexible, generic \WR signal model that is based on the ST weak interaction model; for this reason \PYTHIA 
%was also used to simulate $\WR \rightarrow \ell\ell jj$ interactions.  

Background events are simulated in three steps.  The first 
step simulates the hard interaction between colliding protons, the decay of unstable particles, and the hadronization of 
partons leaving the interaction with one or two \MC generators.  For particles produced in the first step that interact with 
any sub-detector, the second step simulates the electronic signals generated in each sub-detector.  In the third step, the 
reconstruction algorithms process these signals to reconstruct leptons and jets.

In the first step one \MC generator simulates the hard interaction between colliding protons, and the decay of unstable particles.  
The \MC generator used to simulate the each background was chosen to match the strengths of the generator 
with the characteristics of the hard interaction.  The \MADGRAPH generator \cite{madgraph} simulates all Feynman diagrams of a hard 
interaction at leading order in the electroweak coupling with up to 4 additional partons radiated from the interaction.  Radiating 
one additional parton with $|\eta| < 2.5$ and $\pt > 30$ $\GeV$ virtually guarantees that an additional jet will be reconstructed 
\cite{pflowEventReco}, but reduces the cross section $\times$ branching ratio by $\alpha_{QCD} \sim$0.1.  Due to the cross section penalty, 
simulating up to 4 additional radiated partons was only warranted for high cross section $\times$ branching ratio interactions - \DY, 
$t\bar{t} \rightarrow \ell\ell jj$, and $W \rightarrow \ell\nu$.  Therefore, \MADGRAPH was used to simulate the $\DY$+jets, $t\bar{t}$ 
and $W$+jets interactions.  In contrast with \MADGRAPH, the \POWHEG generator \cite{powheg} simulated all Feynman diagrams of a hard 
interaction at next-to-leading order in the electroweak and QCD couplings with up to one additional parton radiated from the interaction.  
\POWHEG was used to simulate single top quark interactions because their cross sections $\times$ branching ratios to $\ell\ell jj$ 
final states are at least 7\% larger at next-to-leading order relative to leading order \cite{singleTopNLOvsLO}.  
At leading order the diboson (WW, WZ, ZZ) interactions collectively have a cross section $\times$ branching ratio to $\ell\ell jj$ 
final states that is $\sim$50$\times$ less than that of the $\DY$+jets or top quark interactions.  The next-to-leading order correction 
to the diboson cross section $\times$ branching ratio, an increase up to 45\% \cite{dibosonLOvsNLO}, is not large enough to make 
the diboson interactions a significant background, so the next-to-leading order simulation is not needed.  The production of diboson 
pairs and their decays to lepton+jet final states was simulated using the leading order \PYTHIA generator \cite{pythia8,Sjostrand:2006za}, 
with up to one additional parton radiated from the interaction.  Events simulated with any \MC generator show the best agreement with 
experimental data when \PYTHIA is used to simulate parton hadronization \cite{pythiaForHadronization}, so all simulations use \PYTHIA 
and the NNPDF23 PDF set \cite{nnpdf} to hadronize partons.  Excluding partons, the \MC generator that simulates the hard interaction 
also simulates decays of unstable particles with mean lifetime $\tau \lesssim 8\times10^{-11}$ seconds into longer lifetime quasi-stable 
particles, like the $\Sigma^{\pm}$ and $\pi^{\pm}$, and stable particles, like the $\gamma$ and n$^{0}$.  These quasi-stable particles 
travel a mean distance c$\tau \gtrsim 2.4$ cm before decaying, so a few percent of these particles will travel far enough before 
decaying to generate a signal in the first layer of the silicon pixel tracker, located 4.4 cm from the interaction point 
\cite{cmsTdrPhysPerformance}.

%SAVE THIS statement that explains why the aMCatNLO inclusive diboson simulated datasets were not used
%A second set of diboson$\to$leptons+jets events was simulated with a next-to-leading order \MC generator, but these events were 
%generated with positive and negative event weights.  Due to the presence of positive and negative event weights, in some $\Mlljj$ 
%regions the predicted number of diboson events was negative, and the prediction's statistical uncertainty was greater than 100\%.

In the second step the effect of pileup is simulated, and subsequently GEANT \cite{geant4} simulates the propagation and decay of 
quasi-stable and stable particles, and their interactions with the CMS detector.  The large instantaneous luminosity of the LHC 
beams produced multiple pp interactions, pileup, in each selected data event.  When \MC simulations started in the spring of 2015, 
the pileup distribution in upcoming data events was predicted to be a Poisson distribution with mean 12.  In each event produced 
by the first step, the effect of pileup is simulated by pulling a 
random integer $X$ from a Poisson distribution with mean 12, and mixing $X$ simulated minimum bias events into the event.  The 
minimum bias events were simulated only through the first simulation step.  \PYTHIA simulates minimum bias events whose reconstructed 
particle kinematics show the best agreement with particles reconstructed in minimum bias events in data \cite{pythiaForHadronization}, 
so \PYTHIA was used to simulate minimum bias events.  After adding minimum bias events to simulate pileup, GEANT propagates all particles 
(stable and quasi-stable) in the 3.8 $\unit{T}$ magnetic field, simulates their interactions with the detector, and simulates the decays 
of quasi-stable particles to stable particles - $\gamma,p^{\pm},n^{0},\bar{n}^{0},\nu,e^{\pm}$.  Finally, for particles that interact 
with the detector, GEANT simulates the signal generated in the detector.

In the third step the online trigger and offline reconstruction algorithms process the detector signals simulated by GEANT.  The 
final decision of each trigger algorithm, to keep or discard the event, is saved in each simulated event, but no events are discarded 
for any reason.  Then, the same offline reconstruction algorithms used in real collisions are used to reconstruct interaction vertices, 
leptons, and jets.

After finishing the third simulation step, particle energy corrections and event weights were applied to simulated events.  As 
explained previously, the energies of simulated jets, muons, and electrons were corrected to match the energy of particles 
reconstructed in data.  The differences in lepton reconstruction, trigger selection, and offline ID selection efficiencies between 
data and simulated events were corrected by changing the weight of each simulated event, up to $\pm$7\%, depending on the $\pt$ and 
$\eta$ of the selected leptons.  Independent of the reconstructed particle kinematics, the weight of each simulated event 
was normalized to the integrated luminosity of the data, and adjusted further to match the pileup distribution found in the data.  The 
simulated pileup distribution is a Poisson distribution with mean 12, but the pileup distribution found in the data is better 
represented by a Poisson distribution with mean 14 \cite{lumi}.  The discrepancy between the data and simulated pileup distributions 
was corrected by changing the weight of each simulated event, on average by $\sim$5\%, depending on the simulated pileup.

The simulated interactions are summarized in Table \ref{tab:centrallyProducedMC}.  The number of simulated events are expressed in units 
of integrated luminosity, and should be compared to the 2.6 fb$^{-1}$ of data \cite{lumi}.

\begin{table}[bt]
	\caption{The ST background and \WR signal (only $\mnul = \frac{1}{2}\mWR$) simulated datasets.  The "Size" of a dataset is equal 
	to the number of simulated events divided by the 13 $\TeV$ cross section $\times$ branching fraction of the process.}
\label{tab:centrallyProducedMC}

\centering
\resizebox{\textwidth}{!}{
	\begin{tabular}{ |c|c|c|c| } 
	\hline
	Dataset         & Step 1 Generator & cross section (pb) & Size (fb$^{-1}$)   \\
		\hline
		Inclusive DY+jets, $DY \rightarrow ll$ & \MADGRAPH   & 5991    & 1.51 \\ \hline
		DY+jets HT 100-200, $DY \rightarrow ll$ & \MADGRAPH   & 181.3    & 15.0 \\ \hline
		DY+jets HT 200-400, $DY \rightarrow ll$ & \MADGRAPH   & 50.42    & 19.3 \\ \hline
		DY+jets HT 400-600, $DY \rightarrow ll$ & \MADGRAPH   & 6.984    & 153. \\ \hline
		DY+jets HT $>$ 600, $DY \rightarrow ll$ & \MADGRAPH   & 2.704    & 369. \\ \hline
		t$\bar{t}$+jets $\rightarrow ll$+jets & \MADGRAPH  & 85.67    & 286. \\ \hline
		single t $\rightarrow$ leptons+jets  & \POWHEG & 80.95 & 20.8 \\ \hline
		single $\bar{t}$ $\rightarrow$ leptons+jets & \POWHEG & 136.0 & 24.3 \\ \hline
		$\bar{t}$+W $\rightarrow$ all   & \POWHEG & 35.85 & 27.6 \\ \hline
		t+W $\rightarrow$ all   & \POWHEG & 35.85 & 27.8 \\ \hline
		WW $\rightarrow$ all  & \PYTHIA & 113.8   & 8.73   \\ \hline
		ZZ $\rightarrow$ all  & \PYTHIA & 10.15   & 98.2   \\ \hline
		WZ $\rightarrow$ all  & \PYTHIA & 23.4   & 41.8   \\ \hline
		W+jets $\rightarrow l\nu$+jets & \MADGRAPH & 50270   & 1.44 \\ \hline
		$\WR \rightarrow l\nul \rightarrow lljj$  & \PYTHIA & 1$\times 10^{-5}$ - 4.3 & 5$\times 10^{6}$ - 11.6   \\ \hline
		\end{tabular}
}
\end{table}

%MOVE information about WR signal events into appendix on WR signal MC
Two sets of \WR signal events were simulated using \PYTHIA - one set was simulated through all three simulation steps, the other 
was simulated through the first step.  The set simulated through all three steps was produced with \mWR stepping from 0.8 to 6 
$\TeV$ in increments of 0.2 $\TeV$, and $\mnul = \frac{1}{2}\mWR$; these fully reconstructed events are used to calculate cross 
section $\times$ branching ratio limits on the $\WR \rightarrow \ell\ell jj$ process as a function of \mWR.  The other set was 
produced with \mWR stepping from 0.8 to 4.0 $\TeV$ in increments of 0.1 $\TeV$, and, at each \mWR, several \mnul values between 
$0.1 \leq \mnul < \mWR$ $\TeV$.  If no \WR signal is found, these events are used to set \mWR and \mnul exclusion limits for 
$\mnul \neq \frac{1}{2}\mWR$.


\section{Top Quark Background}
\label{sec:topQrkBkgnds}
%here add information about top quark background production rate and kinematics that was previously written 
%in the reco and event selection chapter
%RESUME HERE

Top quark processes, including $t\bar{t}$ (Figure \ref{fig:ttbarDiag}) and top quark $\plus$ W (top+W, Figure 
\ref{fig:singleTopDiags}), produce two $W$ bosons.  When both bosons decay to leptons the $\ell\ell jj$ final state is produced.  
Due to lepton universality the two $W$ bosons decay to the $e\mu$ final state twice as often as the $ee$ or $\mu\mu$ final state.  
No other pertinent ST backgrounds or the \WR (Chapter \ref{sec:lrsPhenomenology}) produce the $e\mu$ final state at leading order in 
the electroweak coupling, so $e\mu jj$ events selected in data represented the top quark background.  The $e\mu jj$ events were 
selected during collisions by the same Level-1 single muon trigger described in Chapter \ref{sec:muOnlineSel}.  Then, events were 
required to pass the following $e\mu$ HLT selection criteria:

\begin{itemize}
	\item A track reconstructed in the silicon tracker with $\pt > 30$ $\GeV$ and $|\eta| < 2.4$ was geometrically matched to 
		the muon detector track segment that passed the L1 trigger.
	\item In the plane perpendicular to the beam axis, the distance between the silicon tracker track origin and its 
		reconstructed vertex was $< 1$ mm.
	\item One 5 $\times$ 5 ECAL crystal cluster was required to have $\Et > 30$ $\GeV$.
	\item For the selected ECAL cluster (energy E):
	\begin{itemize}
		\item The hadronic energy behind the cluster was $<$ 15\% of E in the barrel, and $<$ 10\% of E in the endcap. 
		\item Ninety percent of E was measured in an area that was two crystals wide in $\eta$.
		\item If the cluster was in the barrel, a reconstructed track with signals in at least two pixel tracker layers 
			extrapolated close to the cluster position.  The track extrapolated from the pixel tracker to within $2.3$ cm 
			of the cluster $z$ position, and to within 1 ECAL crystal area of the cluster $(\eta,\phi)$ position.
	\end{itemize}
\end{itemize}

In events selected by the $e\mu$ trigger, muons, electrons, and jets were reconstructed and selected using the reconstruction algorithms 
and selection criteria described in Chapter \ref{sec:reco_chapter}.  The offline selection criteria applied to the two leptons was 
applied to the reconstructed electrons and muons.  Events were selected if a reconstructed electron and muon had $\pt > 50$ $\GeV$, 
and one had $\pt > 60$ $\GeV$.  The online and offline selection criteria were applied to all simulated background events, and compared 
to the selected data events in Figure \ref{fig:dataAndSimsInEMuChannel}.  Comparing the magnitudes of different simulated backgrounds, 
the top quark background produces more than 95\% of selected $e\mu jj$ events.

The top quark background in the $e\mu$-channel ($e\mu jj$ final state) was multiplied by the ratio of $\ell\ell / e\mu$ production $\times$ 
$\frac{e}{\mu}$ selection efficiency to predict the top quark background in the same flavor $\ell\ell$-channels.  As explained 
previously, the $\ell\ell / e\mu$ production ratio is 0.5, independent of lepton energy, because of lepton universality in $W$ boson 
decays.  The $\frac{e}{\mu}$ selection efficiency ratio was below 1 for two reasons: electrons reconstructed in the ECAL barrel-endcap 
transition region, $1.44 < |\eta| < 1.57$, were ignored, and the offline electron identification selection criteria were less efficient 
than the muon selection criteria.  The $\frac{e}{\mu}$ selection efficiency ratio was expected to be constant as a function of $\Mlljj$, 
and this was checked using simulated top quark events.  First, simulated top quark events were selected in three groups: those passing 
the $e\mu$-channel, $ee$-channel, or $\mu\mu$-channel selection criteria.  Then, selected events were used .


The $\ell\ell / e\mu$ production ratio $\times$ the $\frac{e}{\mu}$ selection efficiency was 
calculated as the ratio .

using simulated top quark background events selected in three groups: events that passed the $e\mu$-channel selection, the $ee$-
channel selection, and the $\mu\mu$-channel selection.  The ratio of .



%Simulated events are not the main element of the top quark bkgnd estimate
%the estimate is made using data, so discuss this first, then explain role of simulations when appropriate
Simulations of $t\bar{t}$ (Figure \ref{fig:ttbarDiag}), top quark $\plus$ W (top+W, Figure \ref{fig:singleTopDiags}), and other 
top quark production processes were used to estimate the top quark background.  .

One of the two largest backgrounds in the \WR and \nul search was ST processes that produced at least 
one top quark, like top quark $\plus$ $W$ boson production (top+W, Figure \ref{fig:singleTopDiags}), 
and top anti-top quark pair production ($\ttbar$, Figure \ref{fig:ttbarDiag}).  Top 
quarks decay to a $W$ boson and bottom (b) quark, which subsequently decay to leptons and hadrons.  
Based on lepton universality in $W$ boson and b quark decays, events with top (or anti-top) quarks produce 
the $e\mu jj$ final state ($e\mu$-channel) at twice the rate of the $eejj$ or $\mu\mu jj$ final states.  At leading order 
in the weak coupling constant, the \WR cannot decay through a \nul to the $e\mu jj$ final state, so $e\mu$-channel 
data events are essentially free of \WR signal.  This allowed $e\mu$-channel data events to be used to 
estimate the top quark contribution to the $\Meejj$ and $\Mmumujj$ distributions found in data.

\begin{figure}[h]
	\centering
	\includegraphics[width=0.7\textwidth]{figures/topAntiTopFeynDiagram.png}
	\caption{$\ttbar$ Feynman diagram \cite{ttbarDiagram}.}
	\label{fig:ttbarDiag}
\end{figure}

\begin{figure}[h]
	\centering
	\includegraphics[width=0.7\textwidth]{figures/singleTopQuarkFeynDiagrams.png}
	\caption{Single top quark Feynman diagrams \cite{singleTopQrkDiagrams}.}
	\label{fig:singleTopDiags}
\end{figure}

The reconstructed $e\mu$ data events were selected using the electron, muon, and jet requirements described 
in Chapter \ref{sec:event_selection_chapter}, and with the additional requirements that one selected lepton was a 
muon, the other was an electron, and at least one lepton had $\pt > 60$ $\GeV$.  Simulated ST background events were selected with the 
same online and offline requirements, and compared to the $e\mu$-channel data.  Top quark backgrounds 
produced the majority of events that passed the $e\mu$-channel selections, as shown in Figure \ref{fig:dataAndSimsInEMuChannel}.  

\begin{figure}[h]
	\centering
	\includegraphics[width=0.7\textwidth]{figures/Mlljj_eMuChannel_log.pdf}
	\caption{The $\Mlljj$ distribution from data and simulated ST events that passed the $e\mu$ selection, excluding 
	the $\Mlljj > 600 \GeV$ cut.  The bin widths were variable, and their contents were normalized to the bin widths.}
	\label{fig:dataAndSimsInEMuChannel}
\end{figure}

The $\Memujj$ distribution found in data were used to represent the top quark background, and its shape was expected 
to be the same in the $\Meejj$ and $\Mmumujj$ distributions.  This expectation was tested using simulated 
top quark events.  Simulated events were selected using the $ee$-, $e\mu$-, and $\mu\mu$-channel requirements, and were used 
to produce $\Meejj$, $\Memujj$, and $\Mmumujj$ distributions.  The ratios of the binned distributions $\Meejj / \Memujj$ 
and $\Mmumujj / \Memujj$, shown in Figure \ref{fig:ttbarSFratios}, were consistent with constant values within 
the statistical uncertainty of each bin.  Each ratio was approximately independent of $\Mlljj$, which indicated that the $\Mlljj$ 
distribution shape in top quark events was the same in all three lepton channels.  Therefore, the shape of 
the $\Memujj$ distribution found in data represented the shape of the top quark background in the $\Meejj$ 
and $\Mmumujj$ distributions.

As the top quark background shape was the same in all lepton channels, the only difference between the top quark 
background in the $e\mu$-channel and either same flavor $\ell\ell$-channel was a difference in normalization.
This normalization difference was estimated using the invariant mass 
ratios, $\Mlljj / \Memujj$, obtained from simulated top quark events passing selections.  For each channel an average 
ratio, or normalization factor (NF), was calculated by integrating the $\Mlljj$ and $\Memujj$ distributions and 
dividing the integrals, $\int \Mlljj / \int \Memujj$.  The NF was 0.659 for the $\mu\mu$-channel, 
and 0.432 for the $ee$-channel.  These NFs differed from the expected value of 0.5 because the acceptance $\times$ 
selection efficiency of electrons was lower than that of muons.  Multiplying the $\Memujj$ distribution found in 
data by 0.659 (0.432) predicted the top quark contribution to the $\Mmumujj$ ($\Meejj$) distribution found in data.

The NFs were calculated using simulated events with $\Mlljj > 600$, but were dominated by events with $\Mlljj < 1500$ $\GeV$.  
As a result, an error entered the top quark background estimate that was proportional to the per-bin fluctuations 
of the $\Mlljj / \Memujj$ ratios around the NFs shown in Figure \ref{fig:ttbarSFratios}.  The impact of this error 
was covered by assigning a 10\% uncertainty to both NFs.  The contribution of non-top quark backgrounds to the $\Memujj$ 
distribution found in data, represented by simulated events in Figure \ref{fig:dataAndSimsInEMuChannel}, was more 
than one order of magnitude smaller than the NF uncertainty, so these backgrounds were not subtracted from 
the $e\mu$-channel data.

\begin{figure}[btp]
	\centering
	\subfigure{
		\includegraphics[width=0.45\textwidth]{figures/flavor_ratio_EE_variablebinwidth.pdf}
	}
	\subfigure{
		\includegraphics[width=0.45\textwidth]{figures/flavor_ratio_MuMu_variablebinwidth.pdf}
	}
	\label{fig:ttbarSFratios}
	\caption{The bin-by-bin ratio of the $\Mlljj$ and $\Memujj$ distributions from simulated top quark backgrounds, where 
		$\ell$ is an electron on the left, and a muon on the right.}
\end{figure}


\section{\DY Background}
\label{sec:dyBkgnd}
The second of the two largest backgrounds in the \WR and \nul search was the production and decay of a Z boson to 
a lepton pair in association with jets (\DY), $pp \rightarrow Z+jets \rightarrow \ell\ell+jets$.  The \DY 
background was estimated using simulated $\DY$+jets events selected with the $ee$- and $\mu\mu$-channel requirements 
described previously.  Using simulated events to predict the \DY background transformed known limitations of 
the $\DY$+jets simulation into changes to the normalization and shape of the \DY background.  The magnitude of 
these changes were estimated in control regions where data and simulated background events were expected to agree.


\subsection{\DY Normalization in $\Mlljj$}
\label{sec:dyNormInMlljj}
$\DY$+jets events were simulated using the \MADGRAPH generator at leading order in the electroweak and strong coupling 
constants, while $\DY$+jets events observed in data were produced at all orders in these coupling constants.  The 
simulated dataset contained more events than were expected in the 2.6 fb$^{-1}$ of data, so the simulated events were 
weighted to the integrated luminosity of data using an uncertain cross section $\times$ branching fraction calculated during 
simulations.  The cross section $\times$ branching fraction uncertainty was sensitive to the higher order electroweak 
and strong coupling corrections, and the size of these corrections created a discrepancy in the $\Mlljj$ distribution 
normalization between data and simulated events.

The magnitude of the normalization discrepancy in the $\Mlljj$ distribution was estimated 
using the $Z \rightarrow \ell\ell$ control region where data and simulated background events were expected to agree.  Data and simulated 
$\DY$+jets events in the $ee$-channel control region were selected by a Level-1 trigger that required one ECAL SC with 
$\Et > 30$ $\GeV$ and $|\eta| < 2.1$.  Events that passed the Level-1 selection were used offline if they passed the 
following double electron HLT selections:

\begin{itemize}
	\item One SC was detected with $\Et > 30$ $\GeV$, and a second non-overlapping SC was detected with $\Et > 4$ $\GeV$.
	\item For the SC with $\Et > 30$ $\GeV$:
	\begin{itemize}
		\item The ratio of hadronic energy in the HCAL tower behind the SC to the SC energy was $< 0.055$ in the barrel, and 
			$< 0.07$ in the endcap.
		\item Ninety percent of the SC energy was measured in an $(\eta, \phi)$ region that was two crystals wide in $\eta$.
		\item A reconstructed track with hits in at least two pixel tracker layers was matched to the SC, and extrapolated to the SC 
			centroid along the beam axis within 1 cm, and extrapolated to the SC centroid in $(\eta, \phi)$ within the $(\eta, \phi)$ 
			area of $\frac{1}{2}$ an ECAL crystal.  In addition, the SC energy and track momentum cannot differ by more than 50\%.
		\item In a cone of radius $\Delta R =$ 0.3 centered on the SC:
		\begin{itemize}
			\item The fraction of the total ECAL energy in the cone not associated with the SC was $< 0.225$ in the barrel, and 
				$< 0.121$ in the endcap.
			\item The total HCAL energy in the cone divided by the SC energy is $< 0.155$ in the barrel, and $< 0.16$ in the endcap.
		\end{itemize}
	\end{itemize}
\end{itemize}

Data and simulated events in the $\mu\mu$-channel control region were selected by a Level-1 trigger that required a track 
in at least one muon DT or CSC detector with $\pt > 20$ $\GeV$.  Events that passed the Level-1 selection were used offline 
if they passed the following single muon HLT selections:

\begin{itemize}
	\item A track reconstructed in the silicon tracker with $\pt > 22$ $\GeV$ and $|\eta| < 2.4$ was geometrically matched to 
		the muon detector hits that passed the L1 trigger.  Considering this set of muon detector hits and the matching reconstructed 
		track:
	\begin{itemize}
		\item A curve representing the muon trajectory through CMS was fitted to the reconstructed track and at least 
			one muon detector hit with $\chi^{2}/nDOF < 20$.
		\item In the plane perpendicular to the beam axis, the distance between the reconstructed track origin and its 
			reconstructed vertex was $< 1$ mm.
		\item In a cone of radius $\Delta R =$ 0.3 centered on the muon trajectory:
		\begin{itemize}
			\item The total ECAL energy in the cone divided by the momentum measured by the muon detectors is $< 0.11$ in 
				the barrel, and $< 0.08$ in the endcap.
			\item The total HCAL energy in the cone divided by the momentum measured by the muon detectors is $< 0.21$ in 
				the barrel, and $< 0.22$ in the endcap.
			\item The total $\pt$ of all silicon tracker tracks in the cone (excluding the muon track) divided by the 
				muon track $\pt$ is $< 0.09$.
		\end{itemize}
	\end{itemize}
\end{itemize}

Similar to the offline lepton and jet selections used in the \WR search defined in Chapter 
\ref{sec:event_selection_chapter}, reconstructed events in both channels were required to have two leptons and two jets passing kinematic 
and ID selections.  However, looser lepton $\pt$ and $\Mll$ selections were used to select events in the $Z \rightarrow \ell\ell$ 
mass window.  The differences and similarities between the $Z \rightarrow \ell\ell$ control region and the \WR search selections 
are listed in Table \ref{tab:cutCompSignalRegAndZllReg}.

\begin{table}[h]
	\caption{The corrections and selections applied to events used in the \WR search, and events used to estimate the \DY 
	background normalization discrepancy.  \textbf{Differences} in \textbf{bold}. (energies and masses in $\GeV$)}
	\label{tab:cutCompSignalRegAndZllReg}
	\centering
	\begin{tabular}{c|c|c}
		Correction & \WR search region & $Z \rightarrow \ell\ell$ region \\  \hline
		lepton energy & applied & identical applied \\
		jet energy & applied & identical applied \\
		trigger selection efficiency & applied & applied (different values) \\
		lepton reco and selection efficiency & applied & identical applied \\	\hline
		Selection & \WR search region & $Z \rightarrow \ell\ell$ region \\  \hline
		jet $\pt$ and $\eta$ & $\pt > 40$, $|\eta| < 2.4$ & identical applied \\
		jet ID & applied & identical applied \\
		lepton-jet separation & $\Delta R > 0.4$ & identical applied \\
		lepton ID & applied & identical applied \\
		\textbf{lepton} $\pt$ \textbf{and} $\eta$ & $\pt > 60, 53$, both $|\eta| < 2.4$ & both $\pt > 35$, $|\eta| < 2.4$ \\
		\textbf{di-lepton mass} $\Mll$ & $> 200$ & $70 < \Mll < 110$ \\
		\textbf{di-lepton di-jet mass} $\Mlljj$ & $> 600$ & none applied \\	\hline
	\end{tabular}
\end{table}

Events in data and $\DY$+jets simulations passing the $Z \rightarrow \ell\ell$ selections were used to produce $\Mll$ 
distributions.  In both lepton channels, the $\Mll$ distribution shape agreed between data and simulated events.  The 
$\Mll$ distribution normalization in data events exceeded that in simulated events by 15.7\% in the $ee$-channel, 
and by 14.2\% in the $\mu\mu$-channel.  This normalization discrepancy was applied as an increase to the simulated $\DY$ 
background in each channel.  The contamination of other ST backgrounds in 
selected data events was estimated using simulated background events selected with the requirements described above, and 
was $\sim$3\% of the estimated $\DY$ background.  This was comparable to the uncertainty on the $\sim$15\% correction 
discussed next, so other ST backgrounds were ignored when the \DY background normalization correction was calculated.

Two additional simulated $\DY$+jets datasets were produced with different \MC generators, and used to estimate the uncertainty 
on the $\sim$15\% \DY background correction.  Events from data and all three simulated $\DY$+jets datasets were selected 
using the $Z \rightarrow \ell\ell$ selections without jet requirements.  The jet requirements were removed to compare data 
to simulations in a phase space where all three simulated datasets yielded a large number of events after selections.  Selected 
events were used to produce $\Mll$ distributions, and the difference in the $\Mll$ normalization was calculated between data and each 
set of simulated events.  The largest difference between data and any set of simulated events was taken as the uncertainty on 
$\sim$15\%.  The uncertainty was 2.0\% in the $ee$-channel, and 1.0\% in the $\mu\mu$-channel.

\subsection{\DY Shape in $\Mlljj$}
\label{sec:dyShapeInMlljj}
In $\DY$+jets events from data or simulations, the Z production and decay was reconstructed as a vertex that was 
most often the primary vertex (PV), which was defined as the vertex with the highest $\sum \pt$ of all associated 
tracks in the event.  Selected reconstructed jets were required to have at least one charged hadron from the PV, and the partons 
radiated during the Z production and decay produced the majority of charged hadrons that traced back to the PV\footnote{The 
contamination of $Z \rightarrow \tau\tau$ in events with $ee$+jets or $\mu\mu$+jets final states was minimized by 
requiring two leptons with $\pt > 53$ $\GeV$.}.  $\DY$+jets events were simulated with up to 4 partons radiated from 
the $Z \rightarrow \ell\ell$ interaction, while $\DY$+jets events observed in data were produced with any number - 
0, 2, 6, etc. - of partons radiated from the $Z \rightarrow \ell\ell$ interaction.  The limit of 4 radiated partons 
in simulated events reduced the number of reconstructed jets that could pass the selections, and affected the energy 
of selected jets, as multiple partons could contribute to the same jet.  These jet modelling effects produced a discrepancy 
in the $\Mlljj$ distribution shape between data and simulated events.

The magnitude of the $\Mlljj$ distribution shape discrepancy was estimated 
using a low $\Mll$ control region where data and simulated ST background events were expected to agree.  Events from data and all simulated ST 
backgrounds in the $ee$- and $\mu\mu$-channel low $\Mll$ control regions were selected using the electron and muon triggers described in 
Chapter \ref{sec:triggers}.  Events passing the trigger selections were corrected and selected in the same way as events 
used in the \WR search region (Table \ref{tab:cutCompSignalRegAndZllReg}), but were required to have $\Mll < 180$ $\GeV$.  
Selected events in data and simulations were used to produce $\Mlljj$ distributions, and the $\sim$15\% increase was 
applied to simulated $\DY$+jets events.  The $\Mlljj$ distributions from data and simulated events, shown in Figure 
\ref{fig:mlljjLowDileptonMassSideband}, were compared, and the $\Mlljj$ shape discrepancy was estimated as the largest 
percentage difference between data and the total simulated background in any bin.  The shape discrepancy magnitude was 
conservatively estimated to be 40\% in both lepton channels, constant versus $\Mlljj$.  When calculating results the 
40\% uncertainty was applied to the expected number of \DY events.

\begin{figure}[btp]
\centering
\subfigure{
  \includegraphics[width=0.45\textwidth]{figures/Mlljj_eeChnl_lowMllCR.png}
}
\subfigure{
  \includegraphics[width=0.45\textwidth]{figures/Mlljj_mumuChnl_lowMllCR.png}
}
\caption{The $\Mlljj$ distribution from data and simulated ST background events that passed the low $\Mll$ region selection, with 
	the $ee$- ($\mu\mu$-) channel on the left (right).  The bin widths are variable, and the bin contents are normalized to their widths.}
\label{fig:mlljjLowDileptonMassSideband}
\end{figure}

The $\DY$+jets simulation was produced at leading order in the strong coupling constant, and it was studied if the 40\% 
\DY background uncertainty could be reduced by simulating higher order strong interactions.  A separate set of $\DY$+jets 
events were simulated at next-to-leading order in the strong coupling constant, and were selected using the low $\Mll$ 
control region requirements.  Using the higher order simulation did not reduce the 40\% uncertainty, as shown in Figure 
\ref{fig:mlljjLowDileptonMassSidebandAMCNLO}, because the higher order simulation lacked events with $\Mlljj > 600$ 
$\GeV$.  Relative to the leading order $\DY$+jets simulation, the higher order simulation had a factor of $\sim$3 fewer 
events with $\Mlljj > 600$ $\GeV$, and a factor of $\sim$10 fewer events with $\Mlljj > 1$ $\TeV$.

\begin{figure}[btp]
\centering
\subfigure{
  \includegraphics[width=0.45\textwidth]{figures/Mlljj_eeChnl_lowMllCR_AMCNLO.pdf}
}
\subfigure{
  \includegraphics[width=0.45\textwidth]{figures/Mlljj_mumuChnl_lowMllCR_AMCNLO.pdf}
}
\caption{The $\Mlljj$ distribution from data and simulated ST events that passed the $\Mll < 180 \GeV$ selection, the 
		$ee$ ($\mu\mu$) channel on the left (right).  The bin contents are normalized to their widths.  Jet production 
	in \DY events was simulated at a higher order relative to the default \DY dataset.}
\label{fig:mlljjLowDileptonMassSidebandAMCNLO}
\end{figure}

The low $\Mlljj$ control region was used to validate the $\sim$15\% correction applied, and 40\% uncertainty assigned to 
the estimated \DY background.  In this control region, events from data and simulations of all ST backgrounds were selected 
using the \WR search requirements (Table \ref{tab:cutCompSignalRegAndZllReg}), but with $\Mlljj < 600$ $\GeV$.  Selected 
data and simulated events were used to make $\Mll$ distributions, and the simulated \DY background was increased by $\sim$15\%.  
Comparing the $\Mll$ distributions found in data and simulations, shown in Figure \ref{fig:mllInLowMlljjSideband}, indicated 
that the $\sim$15\% \DY correction brought data and simulations into better agreement.  In addition, the comparison showed 
that the 40\% \DY normalization uncertainty was not too conservative, as the disagreement between data and simulated 
backgrounds approached 40\% in several bins.

\begin{figure}[btp]
\centering
\subfigure{
  \includegraphics[width=0.45\textwidth]{figures/Mll_eeChnl_lowMlljjCR.png}
}
\subfigure{
  \includegraphics[width=0.45\textwidth]{figures/Mll_mumuChnl_lowMlljjCR.png}
}
\caption{$\Mll$ for data and simulations of all ST backgrounds in the $\Mlljj < 600\GeV$ control region.  The 
$ee$ ($\mu\mu$) channel was on the left (right).}
\label{fig:mllInLowMlljjSideband}
\end{figure}

\subsection{\DY Background Estimate}
The \DY contribution to the $\Mlljj$ distribution found in data was estimated by selecting simulated \DY events with the 
selections described in Chapter \ref{sec:event_selection_chapter}.  The weight of each selected event was multiplied by 1.157 
in the $ee$-channel, and by 1.142 in the $\mu\mu$-channel.  When calculating results a 40\% uncertainty was assigned to the 
estimated number of \DY events.


\section{Diboson and W+jets Backgrounds}
\label{sec:dibosonAndWJetsBkgnds}
The production of diboson (WW, WZ, ZZ), and single W bosons with jets (W+jets) yielded events where two charged leptons and 
jets were reconstructed.  The diboson and W+jets contributions to the $\Mlljj$ distribution found in data was estimated by 
selecting simulated diboson and W+jets events with the requirements described in Chapter \ref{sec:event_selection_chapter}.  
The result, shown in Figure \ref{fig:allExpectedBkgnds}, was that diboson and W+jets backgrounds were small compared to 
\DY and top quark backgrounds, and concentrated in the region $\Mlljj < 2000$ $\GeV$ where a \WR boson was excluded by 
previous searches.  Based on their small contributions, the estimated diboson and W+jets backgrounds were neglected when c
alculating results.

\begin{figure}[h]
	\centering
	\subfigure{
		\includegraphics[width=0.45\textwidth]{figures/useOfLLJJMassAsFigureOfMerit.pdf}
	}
	\subfigure{
		\includegraphics[width=0.45\textwidth]{figures/Mlljj_mumuChnl_signalRegionNoData.pdf}
	}
	\label{fig:allExpectedBkgnds}
	\caption{The $\Mlljj$ distributions from simulated \DY, diboson, W+jets backgrounds, and the top quark and QCD backgrounds estimated from 
		data. The normalization of the \WR $\Mlljj$ distribution was scaled down by 70\% to facilitate comparisons between expected 
		signal and backgrounds.  The $ee$ ($\mu\mu$) channel was shown on the left (right).}
\end{figure}


\section{QCD Background}
\label{sec:qcdBkgnd}
QCD multi-jet production yielded events where two charged leptons and jets were reconstructed.  In those events, two real 
jets were incorrectly reconstructed and identified as charged leptons.  The QCD multi-jet contribution to the $\Mlljj$ 
distribution found in data was estimated by selecting events in data with loose lepton ID requirements, then weighting selected 
events by the probability for selected leptons to pass the default (tight) lepton ID reuiqrements.  Data events were selected 
with the requirements described in Chapter \ref{sec:event_selection_chapter}, but with the following (looser) lepton ID 
requirements:

\textbf{Muons}
\begin{itemize}
	\item The silicon tracker track was reconstructed from at least 1 hit in the silicon pixel detector, and at least 5 hits in the 
		entire tracker.
	\item The fitted track representing the estimated muon trajectory through all of CMS originated at a 
		point that was within 2 (5) mm of the muon's reconstructed vertex in the $x-y$ plane ($z$ axis). 
\end{itemize}

\textbf{Electrons}
\begin{itemize}
	\item At least 90\% of the SC energy was measured in a region 2 crystals wide in $\eta$.
	\item The ratio of hadronic energy in the HCAL tower behind the SC to the SC energy was $< 0.15$ 
		in the barrel, and $< 0.10$ in the endcap.
	\item The electron track missed 1 or fewer layers in the silicon pixel or inner silicon strip detectors.
	\item The electron track origin and its reconstructed vertex were separated by a small distance in the $x-y$ plane, 
		$\Delta_{xy} < 0.2$ mm in the tracker barrel, and $\Delta_{xy} < 0.5$ mm in the tracker endcap.
\end{itemize}

Events were not selected if one or more reconstructed leptons passed the default (tighter) lepton ID selections, as these 
events likely had one or more real leptons.  In selected events, the $\Et$ or $\pt$, and $\eta$ of the two selected leptons 
were used to calculate a probability for both leptons to pass the default lepton ID selections.  The probability was calculated 
using empirical formulas for electrons and muons derived from data, and each selected event was weighted by its probability.  
The contribution of weighted events, representing the QCD background, to the $\Mlljj$ distribution found in data is shown in 
Figure \ref{fig:allExpectedBkgnds}, and was negligible compared to other ST backgrounds.  Based on its small contribution, the 
QCD background was ignored when calculating results.


%%%%%%%%%%%%%%%%%%%%%%%%%%%%%%%%%%%%%%%%%%%%%%%%%%%%%%%%%%%%%%%%%%%%%%%%%%%%%%%%
