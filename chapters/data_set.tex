%%%%%%%%%%%%%%%%%%%%%%%%%%%%%%%%%%%%%%%%%%%%%%%%%%%%%%%%%%%%%%%%%%%%%%%%%%%%%%%%
% data_set.tex:
%%%%%%%%%%%%%%%%%%%%%%%%%%%%%%%%%%%%%%%%%%%%%%%%%%%%%%%%%%%%%%%%%%%%%%%%%%%%%%%%
\chapter{Problem 1}
\label{Problem 1}
%%%%%%%%%%%%%%%%%%%%%%%%%%%%%%%%%%%%%%%%%%%%%%%%%%%%%%%%%%%%%%%%%%%%%%%%%%%%%%%%
a. the velocity is $13 \frac{m}{s}$ downward, the momentum is $13000 \frac{kg m}{s}$ downward\newline
   the kinetic energy is $78000$ Joules, and the acceleration is $9.8 \frac{m}{s^{2}}$ downward\newline
                           
initial elevator velocity $v_{0}$ = $0 \frac{m}{s}$\newline
downward velocity $v_{8f}$ after $x_{8e} = 8.0$ meter fall is
$v_{8f}^{2} = 2gx_{8e} \rightarrow v_{8f} = 12.5 \frac{m}{s}$\newline
                                      
acceleration comes from gravity, so acceleration magnitude is $9.8 \frac{m}{s^{2}}$\newline
                                     
momentum magnitude and kinetic energy are calculated using $v_{8f}$\newline
kinetic energy = (1/2)$m_{elev}v_{8f}^{2} = 78400$ Joules\newline
momentum magnitude = $m_{elev}v_{8f} = 12528 \frac{kg m}{s}$\newline
                                                      
b. spring constant = $180000 \frac{N}{m}$\newline
                                                      
the spring constant can be obtained in two equivalent ways using energy conservation.\newline
The spring absorbs all the kinetic energy of the elevator, obtained from part a, plus\newline
the gravitational potential energy which the elevator has when it sits $1.0$ meter\newline
above the point where it comes to rest.\newline
                                            
Equivalently, the spring absorbs all of the potential energy which the elevator has\newline
when it sits $9.0$ meters above the point where it comes to rest (where the spring\newline
has compressed $\Delta x_{s} = 1.0$ meter).\newline
                                                                  
Equating the potential energy of the compressed spring with spring constant k to the \newline
elevator total energy yields\newline

(1/2)k$\Delta x_{s}^{2}$ = $m_{elev}g(9.0 meters)$\newline
or                                                   
(1/2)k$\Delta x_{s}^{2}$ = (1/2)$m_{elev}v_{8f}^{2}$ + $m_{elev}g(1.0 meter)$\newline
                                                    
yields k = $176400 \frac{N}{m}$\newline
                                                       
c. when the elevator stops moving the net force = $170000$ Newtons upward   net accel = $170 \frac{m}{s^{2}}$ upward\newline
                                                    
magnitude of the net force $F_{net}$ = spring force - elevator weight\newline
$F_{net}$ = k$\Delta x_{s}$ - $m_{elev}g = 166590$ Newtons\newline
net acceleration magnitude = $\frac{F_{net}}{m_{elev}} = 166.6 \frac{m}{s^{2}}$\newline
                                                                           
d. when the elevator is $4.0$ meters above the spring its speed = $8.9 \frac{m}{s}$\newline
no energy is lost or dissipated in the spring and elevator system, so the speed of the elevator when it is\newline
moving up and is $4.0$ meters above the spring is equal to the speed of the elevator when it is moving down\newline
and has fallen a height of $4.0$ meters.\newline

$v_{4f}^{2}$ = $2g(4.0 meters)$\newline
speed $v_{4f} = \sqrt{2g(4.0 meters)} = 8.86 \frac{m}{s}$

%%%%%%%%%%%%%%%%%%%%%%%%%%%%%%%%%%%%%%%%%%%%%%%%%%%%%%%%%%%%%%%%%%%%%%%%%%%%%%%%
