%%%%%%%%%%%%%%%%%%%%%%%%%%%%%%%%%%%%%%%%%%%%%%%%%%%%%%%%%%%%%%%%%%%%%%%%%%%%%%%%
% Uncertainties and Results:
%%%%%%%%%%%%%%%%%%%%%%%%%%%%%%%%%%%%%%%%%%%%%%%%%%%%%%%%%%%%%%%%%%%%%%%%%%%%%%%%
\chapter{Statistical Analysis, Uncertainties and Results}
\label{statAnalysis_uncerts_results}
%%%%%%%%%%%%%%%%%%%%%%%%%%%%%%%%%%%%%%%%%%%%%%%%%%%%%%%%%%%%%%%%%%%%%%%%%%%%%%%%
The results and their uncertainties obtained in this search are presented here.  Evidence in 
support of a \WR and \nul signal was sought in finite sized windows of $\Mlljj$, whose sizes 
were chosen to minimize the upper bound \WR cross section limit in the absence of a signal.  
Uncertainties that affected the background and signal estimates were measured in each $\Mlljj$ 
window.  The main uncertainties that affected the background estimate arose from limited event 
statistics in $e\mu$ data, and discrepancies between data and simulations in \DY-rich control 
regions.  The main uncertainties that affected the signal estimate came from luminosity and 
pileup measurements in data.  Additional sources of uncertainty existed and were measured, 
but had a smaller cumulative impact on the results.  Following uncertainty estimation, the 
results were obtained by comparing data, expected SM backgrounds and hypothetical 
\WR and \nul signals using the $\Mlljj$ distribution, and limits on the \WR cross section, 
$\mWR$ and $\mnul$ masses derived from data and expected backgrounds.

%%%%%%%%%%%%%%%%%%%%%%%%%%%%%%%%%%%%%%%%%%%%%%%%%%%%%%%%%%%%%%%%%%%%%%%%%%%%%%%%
% Statistical Analysis and Uncertainties 
%%%%%%%%%%%%%%%%%%%%%%%%%%%%%%%%%%%%%%%%%%%%%%%%%%%%%%%%%%%%%%%%%%%%%%%%%%%%%%%%
\section{Statistical Analysis and Uncertainties}
\label{sec:massWindows_uncertainties}
%%%%%%%%%%%%%%%%%%%%%%%%%%%%%%%%%%%%%%%%%%%%%%%%%%%%%%%%%%%%%%%%%%%%%%%%%%%%%%%%
The \WR and \nul signals considered in this search were characterized by $\Mlljj$ distributions 
that peaked at \mWR, and had tails that extended several hundred $\GeV$ below and above the 
peak.  The only other assumption made regarding the shape of signals in $\Mlljj$ was that 
the tails grew larger as \mWR increased, as shown in Figure \ref{fig:fig:signalShapesAfterSelection}.  
The general shape of \WR signals in the $\Mlljj$ distribution motivated the use of finite 
size $\Mlljj$ windows when analyzing the data.

\begin{figure}[btp]
	\centering
	\subfigure{
		\includegraphics[width=0.45\textwidth]{figures/Mlljj_signalRegionCuts_severalWrSignals_EE.pdf}
	}
	\subfigure{
		\includegraphics[width=0.45\textwidth]{figures/Mlljj_signalRegionCuts_severalWrSignals_MuMu.pdf}
	}
	\label{fig:signalShapesAfterSelection}
	\caption{$\Mlljj$ distribution after all selections for several \WR signal hypotheses in the $ee$ (left) and $\mu\mu$ (right) 
		channels.  The tails of the distribution grew as \mWR increased.}
\end{figure}

The SM backgrounds that passed the signal region selection were characterized by $\Mlljj$ 
distributions that decreased rapidly with increasing $\Mlljj$, whereas any \WR signal 
after the same selection appeared as an individual peak in the $\Mlljj$ distribution.  
This distinction motivated the division of the entire $\Mlljj > 600\GeV$ range into finite 
width $\Mlljj$ windows linked to specific \mWR hypotheses.  The optimal size of each window was 
determined for each \mWR hypothesis using the following procedure:

\begin{itemize}
	\item $\sim$150 $\Mlljj$ windows of different sizes were defined based on the expected width of the $\Mlljj$ distribution.
	\item In each window:
	\begin{itemize}
		\item The number of expected signal events $A$ and all SM background events $B$ were calculated.
		\item A Poisson distribution was made with mean equal to $B$, and a random number $C$ representing 
			the number of measured events in the window was pulled from the Poisson distribution.
		\item Using the procedure described in Chapter \ref{sec:searchResults}, the probability that $C$ 
			was measured due to a fluctuation in $A$ and $B$ was calculated, and transformed into a limit 
			on .
	\end{itemize}
\end{itemize}


%For each \mWR hypothesis, a $\Mlljj$ window ($ min < \Mlljj < max$) was defined such 
%that the expected upper limit\footnote{The expected limit was calculated by setting the 
%number of observed events equal to the number of events expected from SM backgrounds.} on 
%the \WR cross section was minimized.  In each window the number of expected SM background 
%events was assumed to follow a Poisson distribution with mean equal to the number of 
%events predicted in Chapter \ref{sec:backgroundEstimation}.  The .
%
%the limit was calculated using a procedure 
%described later in Chapter \ref{sec:searchResults}, which was only sensitive to the number 
%of events in the window, not the shape of their distribution.  The windows used for each \mWR 
%mass hypothesis, shown in Table \ref{tab:masscuts}, were chosen such that the expected upper 
%limit on the \WR cross section was minimized.

\begin{table}[h]
\caption{$\Mlljj$ window ranges that minimized the expected upper limit on the \WR cross section at different \mWR values.}
\label{tab:masscuts}
\centering
\begin{tabular}{|c|r@{ - }l|r@{ - }l|} \hline
\mWR (\GeV) & \multicolumn{4}{c|}{\Mlljj window (\GeV)}  \\\hline
& \multicolumn{2}{c|}{Electrons}  & \multicolumn{2}{c|}{Muons}  \\  \hline
 800  & 700       &  1100       &  700       &  1200      \\  \hline
1000  & 900       &  1300       &  900       &  1400      \\  \hline
1200  & 1100       &  1550       &  1100       &  1650      \\  \hline
1400  & 1250       &  1750       &  1300       &  1850      \\  \hline
1600  & 1450      &  2000       &  1500      &  2100      \\  \hline
1800  & 1600      &  2250       &  1600      &  2300      \\  \hline
2000  & 1850      &  2550       &  1850      &  2600      \\  \hline
2200  & 2000      &  2800       &  2000      &  2850      \\  \hline
2400  & 2150      &  3100       &  2150      &  3100      \\  \hline
2600  & 2250      &  3400       &  2300      &  3400      \\  \hline
2800  & 2350      &  3700       &  2400      &  3700      \\  \hline
3000  & 2500      &  4000       &  2500      &  3950      \\  \hline
3200  & 2550      &  4300       &  2700      &  4250      \\  \hline
3600  & 2700      &  4900       &  2900      &  4850      \\  \hline
3800  & 2750      &  5200       &  2950      &  5150      \\  \hline
4000  & 2800      &  5500       &  3000      &  5450      \\  \hline
4200  & 2800      &  5750       &  3100      &  5750      \\  \hline
4400  & 2850      &  6050       &  3150      &  6100      \\  \hline
4600  & 2850      &  6300       &  3150      &  6400      \\  \hline
4800  & 2850      &  6600       &  3200      &  6700      \\  \hline
5000  & 2900      &  6850       &  3200      &  7000      \\  \hline
5200  & 2900      &  7050       &  3200      &  7300      \\  \hline
5600  & 2900      &  7500       &  3200      &  7850      \\  \hline
5800  & 2950      &  7700       &  3200      &  8150      \\  \hline
6000  & 2950      &  7900       &  3200      &  8400      \\  \hline
\end{tabular}
\end{table}




%%%%%%%%%%%%%%%%%%%%%%%%%%%%%%%%%%%%%%%%%%%%%%%%%%%%%%%%%%%%%%%%%%%%%%%%%%%%%%%%
% Results
%%%%%%%%%%%%%%%%%%%%%%%%%%%%%%%%%%%%%%%%%%%%%%%%%%%%%%%%%%%%%%%%%%%%%%%%%%%%%%%%
\section{Results}
\label{sec:searchResults}
%describe how limits are calculated with and without syst uncertainties before showing results
%when explaining limit calculation procedure without syst uncertainties, refer back to 
%Chapter \ref{sec:massWindows_uncertainties}

%%%%%%%%%%%%%%%%%%%%%%%%%%%%%%%%%%%%%%%%%%%%%%%%%%%%%%%%%%%%%%%%%%%%%%%%%%%%%
