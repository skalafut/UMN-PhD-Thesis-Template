%%%%%%%%%%%%%%%%%%%%%%%%%%%%%%%%%%%%%%%%%%%%%%%%%%%%%%%%%%%%%%%%%%%%%%%%%%%%%%%%
% Uncertainties and Results:
%%%%%%%%%%%%%%%%%%%%%%%%%%%%%%%%%%%%%%%%%%%%%%%%%%%%%%%%%%%%%%%%%%%%%%%%%%%%%%%%
\chapter{Statistical Analysis, Uncertainties and Results}
\label{statAnalysis_uncerts_results}
%%%%%%%%%%%%%%%%%%%%%%%%%%%%%%%%%%%%%%%%%%%%%%%%%%%%%%%%%%%%%%%%%%%%%%%%%%%%%%%%
The results and their uncertainties obtained in this search are presented here.  Evidence in 
support of a \WR and \nul signal was sought in finite sized windows of $\Mlljj$, whose sizes 
were chosen to minimize the upper bound \WR cross section limit in the absence of a signal.  
Uncertainties that affected the background and signal estimates were measured in each $\Mlljj$ 
window.  The main uncertainties that affected the background estimate arose from limited event 
statistics in $e\mu$ data, and discrepancies between data and simulations in \DY-rich control 
regions.  The main uncertainties that affected the signal estimate came from luminosity and 
pileup measurements in data.  Additional sources of uncertainty existed and were measured, 
but had a smaller cumulative impact on the results.  Following uncertainty estimation, the 
results were obtained by comparing data, expected SM backgrounds and hypothetical 
\WR and \nul signals using the $\Mlljj$ distribution, and limits on the \WR cross section, 
$\mWR$ and $\mnul$ masses derived from data and expected backgrounds.

%%%%%%%%%%%%%%%%%%%%%%%%%%%%%%%%%%%%%%%%%%%%%%%%%%%%%%%%%%%%%%%%%%%%%%%%%%%%%%%%
% Statistical Analysis and Uncertainties 
%%%%%%%%%%%%%%%%%%%%%%%%%%%%%%%%%%%%%%%%%%%%%%%%%%%%%%%%%%%%%%%%%%%%%%%%%%%%%%%%
\section{Statistical Analysis and Uncertainties}
\label{sec:massWindows_uncertainties}
%%%%%%%%%%%%%%%%%%%%%%%%%%%%%%%%%%%%%%%%%%%%%%%%%%%%%%%%%%%%%%%%%%%%%%%%%%%%%%%%
The $\Mlljj$ signature of \WR and \nul signals considered in this search were characterized 
by distributions that peaked at \mWR, and had tails that extended several hundred $\GeV$ 
below and above the peak.  The only other assumption made regarding the shape of signals in 
$\Mlljj$ was that the tails grew larger as \mWR increased, as shown in Figure \ref{fig:fig:signalShapesAfterSelection}.  
The.

\begin{figure}[btp]
	\centering
	\subfigure{
		\includegraphics[width=0.45\textwidth]{figures/Mlljj_signalRegionCuts_severalWrSignals_EE.pdf}
	}
	\subfigure{
		\includegraphics[width=0.45\textwidth]{figures/Mlljj_signalRegionCuts_severalWrSignals_MuMu.pdf}
	}
	\label{fig:signalShapesAfterSelection}
	\caption{$\Mlljj$ distribution after all selections for several \WR signal hypotheses in the $ee$ (left) and $\mu\mu$ (right) 
		channels.  The tails of the distribution grew as \mWR increased.}
\end{figure}


%A cut\&count approach is effective in analyzing the data without exploiting further characteristics 
%of the signal model used as benchmark and to reduce the effect of uncertainties on the shapes of 
%the backgrounds, especially in the high \Mlljj region.


For each \mWR mass hypothesis, a $\Mlljj$ window ($ min < \Mlljj < max$) was defined such 
that the expected upper limit\footnote{The expected limit was calculated by setting the 
number of observed events equal to the number of events expected from SM backgrounds.} on 
the \WR cross section was minimized.  In each window the limit was calculated by counting 
the number of events that fell in the window, without regard to the shape of their distribution.  
a.


The expected limit is calculated using a Bayesian approach assuming a flat prior.  Three hundred toys are thrown, and for each toy the number of observed
events is pulled from a Poisson distribution with mean equal to the number of expected background events.  For each toy a \WR cross section upper limit is calculated, and the
median cross section upper limit from all toys is taken as the final expected limit.
The \Mlljj windows corresponding to different \MWR hypotheses and obtained from the expected limit minimization procedure are shown in Table~\ref{tab:masscuts}.




%%%%%%%%%%%%%%%%%%%%%%%%%%%%%%%%%%%%%%%%%%%%%%%%%%%%%%%%%%%%%%%%%%%%%%%%%%%%%%%%
% Results
%%%%%%%%%%%%%%%%%%%%%%%%%%%%%%%%%%%%%%%%%%%%%%%%%%%%%%%%%%%%%%%%%%%%%%%%%%%%%%%%
\section{Results}
\label{sec:searchResults}

%%%%%%%%%%%%%%%%%%%%%%%%%%%%%%%%%%%%%%%%%%%%%%%%%%%%%%%%%%%%%%%%%%%%%%%%%%%%%
