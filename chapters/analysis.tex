%%%%%%%%%%%%%%%%%%%%%%%%%%%%%%%%%%%%%%%%%%%%%%%%%%%%%%%%%%%%%%%%%%%%%%%%%%%%%%%%
% oscillation_analysis.tex: Analysis of Neutrino Oscillations:
%%%%%%%%%%%%%%%%%%%%%%%%%%%%%%%%%%%%%%%%%%%%%%%%%%%%%%%%%%%%%%%%%%%%%%%%%%%%%%%%
\chapter{Group Problem}
\label{Group Problem}
%%%%%%%%%%%%%%%%%%%%%%%%%%%%%%%%%%%%%%%%%%%%%%%%%%%%%%%%%%%%%%%%%%%%%%%%%%%%%%%%
initial velocity $v_{0}$\newline
launch angle $\theta$\newline
ball mass m\newline
initial height above ground $y_{0}$\newline
gravity g = $9.81 \frac{m}{s^{2}}$

a. max height above ground = $y_{0} + \frac{v_{0}^{2}\sin^{2}(\theta)}{2g}$\newline
use energy conservation to find the max height $h_{m}$\newline
above the launch point\newline
$0.5m(v_{0}\sin(\theta))^{2} = mgh_{m}$\newline
$h_{m} = \frac{(v_{0}\sin(\theta))^{2}}{2g}$\newline
the max height above the ground is $h_{m} + y_{0}$\newline

b. time in air = $\frac{-1}{g}(-v_{0}\sin(\theta) - \sqrt{2gy_{0} + (v_{0}\sin(\theta))^{2}})$\newline
after the ball flies a time $t_{f}$ it hits the\newline
ground, and its Y position satisfies this equation\newline
0 = $y_{0} + (v_{0}\sin(\theta))t_{f} - \frac{g}{2}t_{f}^{2}$\newline
solving for the time in air $t_{f}$ yields\newline
$t_{f} = \frac{-1}{g}(-v_{0}\sin(\theta) +/- \sqrt{2gy_{0} + (v_{0}\sin(\theta))^{2}})$\newline

c. travels a distance = $v_{0}\cos(\theta)(\frac{-1}{g}(-v_{0}\sin(\theta) - \sqrt{2gy_{0} + (v_{0}\sin(\theta))^{2}}))$\newline

d. speed at ground = $\sqrt{v_{0}^{2} + 2gy_{0}}$\newline
speed = $\sqrt{v_{x}^{2} + v_{y}^{2}}$\newline
$v_{x} = constant = v_{0}\cos(\theta)$\newline
the magnitude of the final y velocity $v_{y}$ is equal\newline
to the magnitude of the initial y velocity plus the\newline
increase in kinetic energy due to falling an additional\newline
distance $y_{0}$, so\newline
$0.5mv_{y}^{2} = 0.5m(v_{0}\sin(\theta))^{2} + mgy_{0}$\newline
so $v_{y} = \sqrt{(v_{0}\sin(\theta))^{2} + 2gy_{0}}$\newline
thus final speed = $\sqrt{(v_{0}\cos(\theta))^{2} + (v_{0}\sin(\theta))^{2} + 2gy_{0}}$\newline

e. angle $\theta_{end} = \arctan(-\frac{\sqrt{(v_{0}\sin(\theta))^{2} + 2gy_{0}}}{v_{0}\cos(\theta)})$\newline

f. max height is $y_{0}$\newline

g. time in air = $\frac{-1}{g}(v_{0}\sin(\theta) - \sqrt{2gy_{0} + (v_{0}\sin(\theta))^{2}})$\newline

the ball hits the ground after flying a time $t_{f}$, so\newline
0 = $y_{0} - (v_{0}\sin(\theta))t_{f} - \frac{g}{2}t_{f}^{2}$\newline
solving for the time in air $t_{f}$ yields\newline
$t_{f} = \frac{-1}{g}(v_{0}\sin(\theta) +/- \sqrt{2gy_{0} + (v_{0}\sin(\theta))^{2}})$\newline

h. travels a distance = $v_{0}\cos(\theta)(\frac{-1}{g}(v_{0}\sin(\theta) - \sqrt{2gy_{0} + (v_{0}\sin(\theta))^{2}}))$\newline

i. same speed as earlier = $\sqrt{v_{0}^{2} + 2gy_{0}}$\newline

j. same angle as earlier $\theta_{end} = \arctan(-\frac{\sqrt{(v_{0}\sin(\theta))^{2} + 2gy_{0}}}{v_{0}\cos(\theta)})$\newline

k. the speed of the ball when it hits the ground is\newline
independent of the launch angle $\theta$. This final\newline
speed only differs from the initial launch speed if\newline
there is a difference in gravitational potential energy\newline
between the launch position and the landing position\newline
on the ground.\newline

%%%%%%%%%%%%%%%%%%%%%%%%%%%%%%%%%%%%%%%%%%%%%%%%%%%%%%%%%%%%%%%%%%%%%%%%%%%%%}}}
