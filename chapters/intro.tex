%%%%%%%%%%%%%%%%%%%%%%%%%%%%%%%%%%%%%%%%%%%%%%%%%%%%%%%%%%%%%%%%%%%%%%%%%%%%%%%
% intro.tex: Introduction to the thesis
%%%%%%%%%%%%%%%%%%%%%%%%%%%%%%%%%%%%%%%%%%%%%%%%%%%%%%%%%%%%%%%%%%%%%%%%%%%%%%%%
\chapter{Introduction}
\label{intro_chapter}
%%%%%%%%%%%%%%%%%%%%%%%%%%%%%%%%%%%%%%%%%%%%%%%%%%%%%%%%%%%%%%%%%%%%%%%%%%%%%%%%
The Standard Model (SM) of particle physics describes the fundamental components of matter and their interactions.  
In the SM, matter is categorized into three generations, each with two doublets of particles.  Three doublets 
contain massive quark pairs, which participate in strong and weak nuclear interactions, and electromagnetic interactions.  
The other doublets contain pairs of massive charged leptons and massless neutral leptons (neutrinos), which 
participate in electromagnetic and weak nuclear interactions.  Quark bound states form nuclei, which become atoms 
and molecules when one or more electrons, the lightest charged lepton, bind to nuclei.  Interactions between quarks 
are mediated by gluons, while interactions between leptons or between leptons and quarks are mediated by the photon $\gamma$, 
the Higgs boson $H^{0}$, and the charged $W^{\pm}$ and neutral $Z^{0}$ weak bosons.  Experimental evidence (CITATIONS) 
proves that neutrinos are massive particles, but are several orders of magnitude lighter than the lightest charged 
lepton or quark.  The SM must be extended to accommodate massive neutrinos, and a consequence of this extension is the 
possibility that heavier neutrinos $N_{l}$, and heavier charged weak bosons $W^{\pm}_{R}$ exist.

The SM neutrinos and their interactions mediated by charged $W^{\pm}$ weak bosons play an important role in particle 
physics.  Neutrinos were first proposed by Pauli in 1929 to preserve energy conservation in beta decays.  It took 
nearly thirty years for experimental proof of neutrinos, when in 1956 Reines and Cowan discovered electron neutrinos while 
conducting a nuclear reactor experiment in search of inverse beta decay (CITE).  Experimental proof of the second neutrino 
type, the muon neutrino, came in 1962 (CITE), while proof of the third neutrino type, the tau neutrino, only came in the 
last 20 years (CITE).  The .


%\begin{itemize}
%
%%\item Chapter 1 introduces the analytic goals pursued in this thesis.
%
%\item Chapter 2 briefly presents the history of, and science behind, the
%subjects presented in this thesis.
%
%\item In Chapter 3 the experiment is outlined.
%
%\item Chapter 4 describes the simulation process used in the analysis.
%
%\item Chapter 5 follows the chain of reconstruction software used to obtain
%meaningful results from data.
%
%\item Chapter 6 hashes out the strategy for analysis and presents the data and
%simulated sets that will be used in the analysis.
%
%\item Chapter 7 demonstrates the implementation of the event selection
%processes.
%
%\item In Chapter 8 those events selected in Chapter 7 are analyzed.
%
%\item Chapter 9 presents a final discussion of the analyses presented in the
%thesis.
%
%\end{itemize}
%%%%%%%%%%%%%%%%%%%%%%%%%%%%%%%%%%%%%%%%%%%%%%%%%%%%%%%%%%%%%%%%%%%%%%%%%%%%%%%%
