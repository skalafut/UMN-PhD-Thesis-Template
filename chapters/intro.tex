%%%%%%%%%%%%%%%%%%%%%%%%%%%%%%%%%%%%%%%%%%%%%%%%%%%%%%%%%%%%%%%%%%%%%%%%%%%%%%%
% intro.tex: Introduction to the thesis
%%%%%%%%%%%%%%%%%%%%%%%%%%%%%%%%%%%%%%%%%%%%%%%%%%%%%%%%%%%%%%%%%%%%%%%%%%%%%%%%
\chapter{Introduction}
\label{intro_chapter}
%%%%%%%%%%%%%%%%%%%%%%%%%%%%%%%%%%%%%%%%%%%%%%%%%%%%%%%%%%%%%%%%%%%%%%%%%%%%%%%%
The Standard Model (SM) of particle physics describes the fundamental components of matter and their interactions.  
In the SM, matter is categorized into three generations, each with two doublets of particles.  Three doublets 
contain massive quark pairs, which participate in strong and weak nuclear interactions, and electromagnetic interactions.  
The other doublets contain pairs of massive charged leptons and massless neutral leptons (neutrinos), which 
participate in electromagnetic\footnote{only the charged leptons participate in electromagnetic interactions} and weak nuclear interactions.  Quark bound states form nuclei, which become atoms 
and molecules when one or more electrons, the lightest charged lepton, bind to nuclei.  Interactions between quarks 
are mediated by gluons, while interactions between leptons or between leptons and quarks are mediated by the photon $\gamma$, 
and the charged $W^{\pm}$ and neutral $Z$ weak bosons.  Experimental evidence \cite{NOvAresults,mainzPhaseIIResults,t2kResults} 
substantiates that neutrinos have mass, and are several orders of magnitude lighter than the lightest charged 
lepton or quark.  To accommodate massive neutrinos, the SM must be extended.  A consequence of this extension is the 
possibility that heavier neutrinos $N_{l}$, and heavier charged weak bosons $W^{\pm}_{R}$ exist.

The SM neutrinos and their interactions mediated by $W^{\pm}$ play an important role in particle 
physics.  Neutrinos were first proposed by Pauli in 1929 to preserve energy conservation in beta decays.  It took 
nearly thirty years for experimental proof of neutrinos, when in 1956 Reines and Cowan discovered electron neutrinos while 
searching for inverse beta decay \cite{firstNuDiscovery}.  Experimental proof of the second (muon) and third (tau) neutrino flavors came in 
1962 \cite{muNuDiscovery} and 2000 \cite{tauNuDiscovery}.  Neutrino flavor oscillations, first predicted by Pontecorvo in 1957, suggest that 
neutrinos have mass.  Later, results from solar electron neutrino experiments \cite{kamiokandeTwo,solarNuSummary} motivated 
dedicated neutrino oscillation experiments.  There, neutrinos interact with protons and neutrons in a detector through 
$W^{\pm}$ exchange, and neutrinos are transformed into detectable charged leptons of the same flavor.  Results from 
oscillation experiments support SM extensions in which the known neutrinos have mass, and additional heavy neutrinos 
$N_{l}$ and heavy weak bosons $W^{\pm}_{R}$ exist.

Heavy neutrinos $N_{l}$ and weak bosons $W^{\pm}_{R}$ predicted by SM extensions can be identified through a larger 
than expected number of observed events with two charged SM leptons and two hadronic jets.  A $W^{\pm}_{R}$ can 
decay to an unstable heavy neutrino $N_{l}$ and a same flavor, SM charged lepton $l_{1}$.  Subsequently, the 
neutrino $N_{l}$ can decay to a second, same flavor charged lepton $l_{2}$ and a virtual $W^{*}_{R}$.  The $W^{*}_{R}$ can 
decay to two SM quarks that hadronize into jets $j_{1}$ and $j_{2}$, thus yielding the final state $l_{1}l_{2}j_{1}j_{2}$.  
This final state is also produced by SM processes, principally those that produce one or more top quarks, or 
a $Z$ boson in association with quarks or gluons.  However, in the mass distributions $M_{lljj}$ and $M_{ljj}$ 
of the final state particles, all SM backgrounds manifest as a distribution that decreases rapidly with increasing 
mass, whereas the $W^{\pm}_{R}$ and $N_{l}$ signal appears as a finite width peak centered on the $W^{\pm}_{R}$ 
or $N_{l}$ mass value.

This thesis presents a search for heavy neutrinos $N_{l}$ and weak bosons $W^{\pm}_{R}$ in the CMS data collected 
in 2015 using the CERN Large Hadron Collider.

%%%%%%%%%%%%%%%%%%%%%%%%%%%%%%%%%%%%%%%%%%%%%%%%%%%%%%%%%%%%%%%%%%%%%%%%%%%%%%%%
