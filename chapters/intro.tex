%%%%%%%%%%%%%%%%%%%%%%%%%%%%%%%%%%%%%%%%%%%%%%%%%%%%%%%%%%%%%%%%%%%%%%%%%%%%%%%
% intro.tex: Introduction to the thesis
%%%%%%%%%%%%%%%%%%%%%%%%%%%%%%%%%%%%%%%%%%%%%%%%%%%%%%%%%%%%%%%%%%%%%%%%%%%%%%%%
\chapter{Introduction}
\label{intro_chapter}
%%%%%%%%%%%%%%%%%%%%%%%%%%%%%%%%%%%%%%%%%%%%%%%%%%%%%%%%%%%%%%%%%%%%%%%%%%%%%%%%
%make this more technical, and reduce the long discussion of historical pretexts to the SM
%
%present what is needed, and include enough BSM content to smoothly transition to chapter 2

%history of SM motivating quantum field theories, ending with something like "Particle colliders 
%have tested the SM over the past several decades, and experimental evidence has repeatedly 
%substantiated its predictions."
%
%proceed to describe successes of SM that have been substantiated by experimental evidence
%
%then focus on neutrino mass and matter-antimatter asymmetry not explained by SM
%
%finish by describing structure of the thesis

Physics searches for theories that describe the motion of matter, and interactions between matter.  
In the early 20th century, the incompatibility of classical mechanics theories with 
observations of the hydrogen atom motivated the theory of quantum mechanics.  Concurrently, experimental 
observations that conflicted with classical mechanics and electrodynamics theories led to the development 
of special relativity.  In the 1950s, new experimental observations motivated combining special 
relativity and quantum mechanics into quantum field theories.  The Standard Model (SM) \cite{weinbergSM,salamSM} 
is a quantum field theory that describes interactions between subatomic particles, and was developed 
in light of experimental observations made in the 1960s and earlier.  Over the past 
several decades particle colliders have tested the SM, and experimental evidence has repeatedly 
substantiated its predictions.

Neutrinos and the weak interaction played an important role in the development of the SM.  
Neutrinos were first proposed in 1929 to preserve energy conservation in beta decays, and were first 
substantiated by experimental evidence \cite{firstNuDiscovery} of the electron neutrino in 1956.  
Subsequent predictions of muon and tau neutrinos were later substantiated 
by experimental evidence \cite{muNuDiscovery,tauNuDiscovery}.  As neutrinos were predicted to interact only through weak 
interactions, advances in the understanding of neutrino physics and the theory of weak interactions often coincided.  
A 1932 theoretical model of the weak interaction was proposed to explain beta decay, and included a massless 
neutral lepton later identified as the electron neutrino.  In the 1950s experimental measurements of 
hadron decay rates through the weak interaction, like $K^{+} \rightarrow 2\pi, 3\pi$, motivated new 
models of the weak interaction that did not conserve parity.  Parity violation in weak interactions was 
substantiated by experimental evidence \cite{weakParityViolation} in 1957.  In the following decade, the 
SM predicted the existence of three massless neutrinos\footnote{At the time only two were supported 
by experimental evidence}, and predicted the weak interaction could be modeled as a parity violating 
quantum field theory.

The success of the SM is exemplified by the quantum field model of electromagnetic and weak (electroweak) 
interactions.  The electroweak model predicts the existence of a massive, neutral gauge boson, the $Z$, 
that mediates weak interactions between fermions.  Existence of the $Z$ was first substantiated by 
observations of neutral current scattering between neutrinos \cite{nuScattering} in 1234.  Later at 
LEP and the Tevatron, precise measurements of electroweak coupling strengths and gauge boson ($\gamma$, $W$, $Z$) 
masses put indirect limits on the Higgs boson mass.  The Higgs boson was observed in 2012 at the LHC 
by the ATLAS and CMS experiments, and its mass\cite{combinedHiggsResult} of 125 $\GeV$ was consistent 
with previous electroweak limits.


The SM successfully explains many observed phenomena, but does not explain neutrino flavor oscillations 
or the matter-antimatter asymmetry.  Neutrino flavor oscillations like $\nu_{e} \rightarrow \nu_{\mu}$, first predicted in 1957, 
were supported by evidence from several experiments \cite{kamiokandeTwo,solarNuSummary,NOvAresults,mainzPhaseIIResults,t2kResults}, and 
substantiated that neutrinos had mass.  Neutrinos did not have mass in the SM, so an extension was needed.  
A consequence of this extension was the possibility that heavier neutrinos \nul and heavier charged weak 
bosons $W^{\pm}_{R}$ (\WR) existed.  TALK ABOUT matter-antimatter asymmetry.

Evidence of heavy neutrinos \nul and weak bosons \WR predicted by SM extensions were sought through a larger 
than expected number of observed events with two charged SM leptons and two jets.  A \WR could decay through 
a \nul to two SM charged leptons and jets, $\WR \rightarrow l_{1}\nul \rightarrow l_{1}l_{2}j_{1}j_{2}$.  
Several SM processes also yielded this final state, principally those that produced one or more top quarks, or 
a $Z$ boson in association with quarks or gluons.  However, in the mass distributions $M_{lljj}$ and $M_{ljj}$ 
of the final state particles, all SM backgrounds manifested as a distribution that decreased rapidly with increasing 
mass, whereas a real \WR and \nul signal appeared as a finite width peak centered on the \WR or \nul mass value.

In this thesis, a search for heavy neutrinos \nul and weak bosons \WR in the CMS data collected 
in 2015 using the CERN Large Hadron Collider is presented.

%%%%%%%%%%%%%%%%%%%%%%%%%%%%%%%%%%%%%%%%%%%%%%%%%%%%%%%%%%%%%%%%%%%%%%%%%%%%%%%%
