%%%%%%%%%%%%%%%%%%%%%%%%%%%%%%%%%%%%%%%%%%%%%%%%%%%%%%%%%%%%%%%%%%%%%%%%%%%%%%%
% intro.tex: Introduction to the thesis
%%%%%%%%%%%%%%%%%%%%%%%%%%%%%%%%%%%%%%%%%%%%%%%%%%%%%%%%%%%%%%%%%%%%%%%%%%%%%%%%
\chapter{Introduction}
\label{intro_chapter}
%%%%%%%%%%%%%%%%%%%%%%%%%%%%%%%%%%%%%%%%%%%%%%%%%%%%%%%%%%%%%%%%%%%%%%%%%%%%%%%%

The Standard Theory of particle physics (ST) \cite{weinbergSM,salamSM} 
is a quantum field theory that describes the fundamental components of matter and their interactions.  It 
was developed in the light of experimental observations made in the 1960s and earlier.  Over the past 
several decades the details of the theory have survived many experimental tests.

Neutrinos and the weak interaction played an important role in the development of the ST.  
Neutrinos were first proposed in 1929 to preserve energy conservation in beta decays, and were first 
confirmed by experimental evidence \cite{firstNuDiscovery} with the observation of a neutrino in 1956, 
later identified as the electron neutrino.  Initially it was believed only one neutrino existed.  The 
subsequent discovery of a second neutrino \cite{muNuDiscovery}, later identified as the muon neutrino, 
motivated theories that predicted each charged lepton had a corresponding neutrino.  These predictions 
were confirmed by experimental evidence \cite{tauNuDiscovery} of the tau neutrino in 2001.  
Neutrinos were predicted to interact only through weak 
interactions, so advances in the understanding of neutrino physics and the theory of weak interactions often coincided.  
A 1932 theoretical model of the weak interaction was proposed to explain beta decay, and included a massless 
neutral lepton later identified as the electron neutrino.  In the 1950s experimental measurements of 
hadron decay rates through the weak interaction, like $K^{+} \rightarrow 2\pi, 3\pi$, motivated new, 
parity violating models of the weak interaction; parity violating weak interactions were 
observed experimentally \cite{weakParityViolation} in 1957.  At the time neutrinos were thought 
to be massless for lack of experimental evidence, and the ST was developed with massless neutrinos, and 
modeled the weak interaction as a parity violating quantum field theory.

The success of the ST is exemplified by the quantum field model of electromagnetic and weak (electroweak) 
interactions.  The electroweak model predicts the existence of a massive, neutral gauge boson, the $Z$, 
that mediates weak interactions between fermions.  Existence of the $Z$ was first substantiated by 
observations of neutral current scattering between neutrinos \cite{nuScattering} in 1973.  Later at 
LEP and the Tevatron, precise measurements of electroweak coupling strengths and gauge boson ($\gamma$, $W$, $Z$) 
masses put indirect limits on the Higgs boson mass.  The Higgs boson was observed in 2012 at the Large Hadron Collider 
(LHC) by the ATLAS and CMS experiments, and its mass\cite{combinedHiggsResult} of 125 $\GeV$ was consistent 
with previous electroweak limits.

The ST predicts many experimental observations of the universe, but there are indications that the ST is 
not a complete theory of the universe.  Within the electroweak sector discussed previously, the ST does 
not predict the experimentally observed baryon-antibaryon asymmetry or neutrino oscillaions.  Neutrino flavor 
oscillations were first predicted in 1957, and are supported by experimental evidence 
\cite{kamiokandeTwo,solarNuSummary,NOvAresults,mainzPhaseIIResults,t2kResults,dayaBayResults}.  This evidence substantiates 
models with massive fermionic neutrinos, and conflicts with the ST.  The ST predicts the universe has more 
baryons than antibaryons due to CP violation in the electroweak model, but the experimentally observed 
asymmetry is larger than predicted \cite{surveyOfExtensions}.  The Left-Right Symmetric (LRS) extension of 
the ST predicts massive neutrinos and a baryon-antibaryon asymmetry consistent with experimental observations, 
and retain ST predictions supported by experimental evidence.

The LRS model extends the ST electroweak sector by adding a $SU(2)_{R}$ gauge group and three heavy, right-handed 
neutrinos \nul.  Due to the new gauge group, the LRS model predicts a new charged weak boson \WR that couples to 
\nul and all right-handed ST fermions.  The \WR coupling to ST quarks enables a search for the LRS model using 
data collected from proton-proton collisions at the CERN LHC.  In this thesis, a search for heavy neutrinos \nul 
and a \WR boson in events with two charged leptons and two jets collected by the CMS experiment in 2015 
is presented.

%%%%%%%%%%%%%%%%%%%%%%%%%%%%%%%%%%%%%%%%%%%%%%%%%%%%%%%%%%%%%%%%%%%%%%%%%%%%%%%%
