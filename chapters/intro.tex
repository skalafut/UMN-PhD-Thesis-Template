%%%%%%%%%%%%%%%%%%%%%%%%%%%%%%%%%%%%%%%%%%%%%%%%%%%%%%%%%%%%%%%%%%%%%%%%%%%%%%%
% intro.tex: Introduction to the thesis
%%%%%%%%%%%%%%%%%%%%%%%%%%%%%%%%%%%%%%%%%%%%%%%%%%%%%%%%%%%%%%%%%%%%%%%%%%%%%%%%
\chapter{Introduction}
\label{intro_chapter}
%%%%%%%%%%%%%%%%%%%%%%%%%%%%%%%%%%%%%%%%%%%%%%%%%%%%%%%%%%%%%%%%%%%%%%%%%%%%%%%%
The Standard Model (SM) of particle physics is a quantum field theory that describes the fundamental 
components of matter and their interactions.  The SM was proposed in 
1967 \cite{weinbergSM,salamSM}, and postulates that spin $\frac{1}{2}$ fermions represent matter, and 
integer spin bosons mediate interactions between matter.  Matter is categorized into three generations 
of fermions, each with one doublet of massive charged quarks, and one doublet with a massive charged 
lepton and a massless neutral lepton (neutrino).  The SM also includes anti-particles of all fermions.  
The first generation of particles consists of the up and down quarks, and the electron $e^{-}$ and electron neutrino $\nu_{e}$.  
The second generation contains the muon and muon neutrino, and the charm and strange quark, and the 
third generation contains the tau and tau neutrino, and the top and bottom quark.

The SM describes the electromagnetic, weak nuclear, and strong nuclear interactions using one framework 
where quantized parameters of SM fermions, like electric charge, drive interactions between fermions.  
Each interaction can transform one set of fermions into a different set, and conserves the quantized 
parameter that drives the interaction.  The interactions are represented mathematically by gauge groups, and the 
gauge bosons that mediate interactions are represented by combinations of gauge group 
generators.  The strong interaction, which conserves color charge and is represented by the $SU(3)_{C}$ 
group, occurs between quarks and is mediated by gluons, which are represented by the $SU(3)_{C}$ group 
generators.  The proton is built from two positively charged up quarks and one negatively charged down 
quark, and is stable because the attraction between quarks from the strong interaction overwhelms the 
electromagnetic repulsion between the two up quarks.  The groups $SU(2)_{L} \times U(1)$ represent the 
electromagnetic interaction between charged fermions, and the weak nuclear interaction between all fermions.  
The electromagnetic interaction conserves electric charge, and is mediated by the photon.  The weak 
interaction conserves weak isospin and hypercharge, and is mediated by the charged $W^{\pm}$ and neutral 
$Z$.  For reasons discussed in Chapter \ref{wrBosonAndHeavyNu}, the photon, $W^{\pm}$ and $Z$ are 
represented by combinations of the $SU(2)_{L} \times U(1)$ group generators.

Color charge, color neutral stability and asymptotic freedom drive the dynamics of the SM strong interaction.  
Every quark is produced with one of six possible color charges: red, blue, green, or anti-colors.  Gluons 
that mediate the strong interaction carry two different color charges, like red and blue, or anti-blue and 
green.  Color charge is conserved in strong interactions, and only color neutral quark bound states are 
quasi-stable\footnote{mean lifetime $\gtrsim 10^{-25}$ seconds}.  Thus, quasi-stable bound states can be formed 
from two or three quarks.  
When a proton-proton collision produces a free quark with too much energy to form a color neutral bound state, 
the strong interaction works to maintain stability.  A property of the strong interaction called 
asymptotic freedom allows the free quark to travel a finite distance away from the other final state quarks 
before the strong interaction acts to maintain color neutrality.  Once this distance becomes too large, the 
strong interaction restores color neutrality by creating an energetic gluon in the vicinity of the free quark.  This 
gluon decays to a quark and anti-quark, of which at least one has the correct color charge to 
form a color neutral bound state with the initial free quark.  If the new quark and anti-quark do not 
form a color neutral, 3 quark bound state with the initial free quark, the process of quark anti-quark 
pair production will continue until all quarks in the final state form color neutral bound states.  If 
instead a proton-proton collision produces a free gluon, the gluon decays to a quark and anti-quark with 
different colors, and the process described for one free quark ensues for both quark and anti-quark.  For 
the remainder of this thesis, the shower of color neutral hadrons originating from a free quark or gluon is 
called a jet.

Neutrinos and the weak interaction played an important role in the development of the SM.  
Neutrinos were first proposed in 1929 to preserve energy conservation in beta decays, where an unstable 
neutron decayed to a proton and other particles through a $W^{-}$ boson.  Almost 30 years later the first 
experimental evidence of neutrinos came, when in 1956 a search for inverse beta decay discovered the electron 
neutrino \cite{firstNuDiscovery}.  Subsequent predictions of muon and tau neutrinos were later substantiated 
by experimental evidence \cite{muNuDiscovery,tauNuDiscovery}.  As neutrinos were detected through weak 
interactions, advances in the understanding of neutrino physics were linked to advances in the 
understanding of the weak interaction.  In 1932 a theoretical model of the weak interaction was proposed to 
explain beta decay, and included a neutral lepton that was later identified as the electron neutrino.  Weak 
interaction models were revised in the 1950s when existing models did not predict the decay rates of hadrons 
through the weak interaction observed in experiments, like $K^{+} \rightarrow 2\pi, 3\pi$.  This discrepancy 
motivated a new theory of the weak interaction that did not conserve parity.  Parity violation in the weak 
interaction was supported by experimental evidence \cite{weakParityViolation} in 1957.  Developed $\sim$10 years 
later, the SM included parity violation in the weak interaction, and three massless neutrinos.  Due to parity 
violation, anti-neutrinos were left-handed, and neutrinos were right-handed.

The SM successfully explains many phenomena observed in physics and astronomy, but does not explain neutrino 
flavor oscillations.  Neutrino flavor oscillations, first predicted in 1957, and through which neutrinos of one 
lepton flavor oscillate into other lepton flavors $\nu_{e} \rightarrow \nu_{\mu}$, are supported by evidence 
from several experiments \cite{kamiokandeTwo,solarNuSummary,NOvAresults,mainzPhaseIIResults,t2kResults}, and 
substantiate that neutrinos have mass.  Neutrinos cannot have mass in the SM, so the SM must be extended to 
accommodate massive neutrinos.  A consequence of this extension is the possibility that heavier neutrinos 
\nul and heavier charged weak bosons $W^{\pm}_{R}$ (\WR) exist.

Heavy neutrinos \nul and weak bosons \WR predicted by SM extensions can be identified through a larger 
than expected number of observed events with two charged SM leptons and two jets.  A \WR can 
decay to an unstable heavy neutrino \nul and a same flavor, SM charged lepton $l_{1}$.  Subsequently, the 
neutrino \nul can decay to a second, same flavor charged lepton $l_{2}$ and a virtual $W^{*}_{R}$.  The $W^{*}_{R}$ can 
decay to two SM quarks that hadronize into jets $j_{1}$ and $j_{2}$, thus yielding the final state $l_{1}l_{2}j_{1}j_{2}$.  
This final state is also produced by SM processes, principally those that produce one or more top quarks, or 
a $Z$ boson in association with quarks or gluons.  However, in the mass distributions $M_{lljj}$ and $M_{ljj}$ 
of the final state particles, all SM backgrounds manifest as a distribution that decreases rapidly with increasing 
mass, whereas a real \WR and \nul signal appears as a finite width peak centered on the \WR or \nul mass value.

This thesis presents a search for heavy neutrinos \nul and weak bosons \WR in the CMS data collected 
in 2015 using the CERN Large Hadron Collider.

%%%%%%%%%%%%%%%%%%%%%%%%%%%%%%%%%%%%%%%%%%%%%%%%%%%%%%%%%%%%%%%%%%%%%%%%%%%%%%%%
