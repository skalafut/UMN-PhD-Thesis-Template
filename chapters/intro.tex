%%%%%%%%%%%%%%%%%%%%%%%%%%%%%%%%%%%%%%%%%%%%%%%%%%%%%%%%%%%%%%%%%%%%%%%%%%%%%%%
% intro.tex: Introduction to the thesis
%%%%%%%%%%%%%%%%%%%%%%%%%%%%%%%%%%%%%%%%%%%%%%%%%%%%%%%%%%%%%%%%%%%%%%%%%%%%%%%%
\chapter{Introduction}
\label{intro_chapter}
%%%%%%%%%%%%%%%%%%%%%%%%%%%%%%%%%%%%%%%%%%%%%%%%%%%%%%%%%%%%%%%%%%%%%%%%%%%%%%%%
The Standard Model (SM) of particle physics is a quantum field theory that describes the fundamental 
components of matter and their interactions.  The SM was proposed in 
1967 \cite{weinbergSM,salamSM}, and postulated that spin $\frac{1}{2}$ fermions represent matter, and 
integer spin bosons mediate interactions between matter.  Matter is categorized into three generations 
of fermions, each with one doublet of massive charged quarks, and one doublet with a massive charged 
lepton and a massless neutral lepton (neutrino).  The SM also included anti-particles of all fermions.  
The first generation of particles consisted of the up and down quarks, and the electron $e^{-}$ and electron neutrino $\nu_{e}$.  
The second generation contained the muon and muon neutrino, and the charm and strange quark, and the 
third generation contained the tau and tau neutrino, and the top and bottom quark.

The SM described the electromagnetic, weak nuclear, and strong nuclear interactions using one framework 
where quantized parameters of SM fermions, like electric charge, drove interactions between fermions.  
Each interaction could transform one set of fermions into a different set, and conserved the quantized 
parameter that drove the interaction.  The interactions were represented mathematically by gauge groups, and the 
gauge bosons that mediate interactions were represented by combinations of gauge group 
generators.  The strong interaction, which conserved color charge and was represented by the $SU(3)_{C}$ 
group, occured between quarks and was mediated by gluons, which were represented by the $SU(3)_{C}$ group 
generators.  The proton, built from two positively charged up quarks and one negatively charged down 
quark, was stable because the attraction between quarks from the strong interaction overwhelmed the 
electromagnetic repulsion between the two up quarks.  The groups $SU(2)_{L} \times U(1)$ represented the 
electromagnetic interaction between charged fermions, and the weak nuclear interaction between all fermions.  
The electromagnetic interaction conserved electric charge, and was mediated by the photon.  The weak 
interaction conserved weak isospin and hypercharge, and was mediated by the charged $W^{\pm}$ and neutral 
$Z$.  For reasons discussed in Chapter \ref{wrBosonAndHeavyNu}, the photon, $W^{\pm}$ and $Z$ were 
represented by combinations of the $SU(2)_{L} \times U(1)$ group generators.

Color charge, color neutral stability and asymptotic freedom drove the dynamics of the SM strong interaction.  
Every quark was produced with one of six possible color charges: red, blue, green, or anti-colors.  Gluons 
that mediated the strong interaction carried two different color charges, like red and blue, or anti-blue and 
green.  Color charge was conserved in strong interactions, and only color neutral quark bound states were 
quasi-stable\footnote{mean lifetime $\gtrsim 10^{-25}$ seconds}.  Thus, quasi-stable bound states could be formed 
from two or three quarks.  
When a proton-proton collision produced a free quark with too much energy to form a color neutral bound state, 
the strong interaction worked to maintain stability.  A property of the strong interaction called 
asymptotic freedom allowed the free quark to travel a finite distance away from the other final state quarks 
with no interference from the strong interaction.  Once this distance became too large, the 
strong interaction restored local color neutrality by creating an energetic gluon in the vicinity of the free quark.  This 
gluon decayed to a quark and anti-quark, of which at least one had the correct color charge to 
form a color neutral bound state with the initial free quark.  If a color neutral, 3 quark bound state could not 
be formed, the process of quark anti-quark 
pair production continued until all quarks in the final state formed color neutral bound states.  If 
instead a proton-proton collision produced a free gluon, the gluon decayed to a quark and anti-quark with 
different colors, and the process described for one free quark ensued for the quark and anti-quark.  For 
the remainder of this thesis, the shower of color neutral hadrons originating from a free quark or gluon is 
called a jet.

Neutrinos and the weak interaction played an important role in the development of the SM.  
Neutrinos were first proposed in 1929 to preserve energy conservation in beta decays, where an unstable 
neutron decayed to a proton and other particles through a $W^{-}$ boson.  Almost 30 years later the first 
experimental evidence of neutrinos came, when in 1956 a search for inverse beta decay discovered the electron 
neutrino \cite{firstNuDiscovery}.  Subsequent predictions of muon and tau neutrinos were later substantiated 
by experimental evidence \cite{muNuDiscovery,tauNuDiscovery}.  As neutrinos could only be detected through weak 
interactions, advances in the understanding of neutrino physics and of the weak interaction were linked.  
In 1932 a theoretical model of the weak interaction was proposed to 
explain beta decay, and included a neutral lepton that was later identified as the electron neutrino.  Weak 
interaction models were revised in the 1950s when existing models predicted decay rates of hadrons through the 
weak interaction, like $K^{+} \rightarrow 2\pi, 3\pi$, that were inconsistent with experiments.  This discrepancy 
motivated a new theory of the weak interaction that did not conserve parity.  Parity violation in the weak 
interaction was supported by experimental evidence \cite{weakParityViolation} in 1957.  Developed $\sim$10 years 
later, the SM included three massless neutrinos, and a parity violating weak interaction.

The SM successfully explained many phenomena observed in physics and astronomy, but did not explain neutrino 
flavor oscillations.  Neutrino flavor oscillations like $\nu_{e} \rightarrow \nu_{\mu}$, first predicted in 1957, 
were supported by evidence from several experiments \cite{kamiokandeTwo,solarNuSummary,NOvAresults,mainzPhaseIIResults,t2kResults}, and 
substantiated that neutrinos had mass.  Neutrinos did not have mass in the SM, so an extension was needed.  
A consequence of this extension was the possibility that heavier neutrinos \nul and heavier charged weak 
bosons $W^{\pm}_{R}$ (\WR) existed.

Evidence of heavy neutrinos \nul and weak bosons \WR predicted by SM extensions were sought through a larger 
than expected number of observed events with two charged SM leptons and two jets.  A \WR could decay through 
a \nul to two SM charged leptons and jets, $\WR \rightarrow l_{1}\nul \rightarrow l_{1}l_{2}j_{1}j_{2}$.  
Several SM processes also yielded this final state, principally those that produced one or more top quarks, or 
a $Z$ boson in association with quarks or gluons.  However, in the mass distributions $M_{lljj}$ and $M_{ljj}$ 
of the final state particles, all SM backgrounds manifested as a distribution that decreased rapidly with increasing 
mass, whereas a real \WR and \nul signal appeared as a finite width peak centered on the \WR or \nul mass value.

This thesis presents a search for heavy neutrinos \nul and weak bosons \WR in the CMS data collected 
in 2015 using the CERN Large Hadron Collider.

%%%%%%%%%%%%%%%%%%%%%%%%%%%%%%%%%%%%%%%%%%%%%%%%%%%%%%%%%%%%%%%%%%%%%%%%%%%%%%%%
