%%%%%%%%%%%%%%%%%%%%%%%%%%%%%%%%%%%%%%%%%%%%%%%%%%%%%%%%%%%%%%%%%%%%%%%%%%%%%%%
% intro.tex: Introduction to the thesis
%%%%%%%%%%%%%%%%%%%%%%%%%%%%%%%%%%%%%%%%%%%%%%%%%%%%%%%%%%%%%%%%%%%%%%%%%%%%%%%%
\chapter{Introduction}
\label{intro_chapter}
%%%%%%%%%%%%%%%%%%%%%%%%%%%%%%%%%%%%%%%%%%%%%%%%%%%%%%%%%%%%%%%%%%%%%%%%%%%%%%%%
The Standard Model (SM) of particle physics describes the fundamental components of matter and their interactions.  
In the SM, matter is categorized into three generations, each with two doublets of particles.  Three doublets 
contain massive quark pairs, which participate in strong and weak nuclear interactions, and electromagnetic interactions.  
The other doublets contain pairs of massive leptons and massless neutrinos ($l^{\pm},\nu_{l}$), which participate in 
electromagnetic and weak nuclear interactions.

%\begin{itemize}
%
%%\item Chapter 1 introduces the analytic goals pursued in this thesis.
%
%\item Chapter 2 briefly presents the history of, and science behind, the
%subjects presented in this thesis.
%
%\item In Chapter 3 the experiment is outlined.
%
%\item Chapter 4 describes the simulation process used in the analysis.
%
%\item Chapter 5 follows the chain of reconstruction software used to obtain
%meaningful results from data.
%
%\item Chapter 6 hashes out the strategy for analysis and presents the data and
%simulated sets that will be used in the analysis.
%
%\item Chapter 7 demonstrates the implementation of the event selection
%processes.
%
%\item In Chapter 8 those events selected in Chapter 7 are analyzed.
%
%\item Chapter 9 presents a final discussion of the analyses presented in the
%thesis.
%
%\end{itemize}
%%%%%%%%%%%%%%%%%%%%%%%%%%%%%%%%%%%%%%%%%%%%%%%%%%%%%%%%%%%%%%%%%%%%%%%%%%%%%%%%
