%%%%%%%%%%%%%%%%%%%%%%%%%%%%%%%%%%%%%%%%%%%%%%%%%%%%%%%%%%%%%%%%%%%%%%%%%%%%%%%%
% acknowledge.tex: Acknowledgements
%%%%%%%%%%%%%%%%%%%%%%%%%%%%%%%%%%%%%%%%%%%%%%%%%%%%%%%%%%%%%%%%%%%%%%%%%%%%%%%%

There are many people that I want to acknowledge for their contributions to my research, life and personal development 
as a graduate student.

First and foremost, I thank my parents Tom and Jane Kalafut, and my brother and sister Devin and Megan Kalafut for 
the time they spent with me, their advice, and everything that they did for me during my graduate career.

During the orientation for new PhD students, Professor Joseph Kapusta and I became fast friends 
through our mutual interest in cars and racing.  I will be forever indebted to Joe because he helped me understand quantum 
field theory, and, most importantly, he introduced me to my research advisor - Professor Roger Rusack.  For me, Roger 
was the ideal advisor.  He has immense patience, exemplary knowledge of the English language, grammar and writing, and 
impressive management abilities.  Thanks to Roger I wrote a doctoral dissertation that I am proud of, and I developed a 
large network of contacts within the CMS collaboration at Fermilab and CERN that helped me achieve my goals.  In addition, 
Roger instilled in me greater self-confidence, emphasized the value of fastidious research, and, above all, helped me 
improve myself.  Through Roger I had the privelege of getting to know and working with Professors Jeremy Mans and Yuichi 
Kubota at Minnesota, Dr. Frank Chlebana and Professor Richard Cavanaugh at Fermilab, Dr. Rajdeep Chatterjee, and Dr. Shervin 
Nourbahksh and many other CMS ECAL experts at CERN.  

I would also like to thank all my collaborators in CMS and at the LHC who have made this research possible through their 
dedication. The excellence of the CMS and LHC physics programs speaks for itself, but none the less I am grateful to all those
involved.

The methods and results presented in this thesis were improved and refined through conversations with the people discussed 
previously, and several other people that played important roles in my graduate career.  My collaborators Shervin Nourbahksh 
and Dr. Nicole Manuela Ruckstuhl and Giulia Negro at CERN, Peter Hansen and Andrew Evans at Minnesota, and Professor Peter 
Wittich and Jorge Chaves at Cornell.  They all provided invaluable insight into particle physics research, and literally 
gave me food for thought by sharing their knowledge of international cuisines in response to my countless questions about 
food.  I also thank Dr. Niki Saoulidou and Dr. Dinko Ferencek for their useful questions and comments during the analysis 
review process.

I am grateful to Dr. Joe (Nathaniel) Pastika, Alexey Finkel, Dr. Tutanon Sinthuprasith, and Gabriele Meoni for their friendship 
and their advice in research and important aspects of life.  All of them helped me improve myself, and become more productive 
in research and in recreational pursuits.

Since childhood I have been interested in other cultures, and curious to learn more about them.  In 2012 an opportunity to 
learn more about Russian culture presented itself; I took advantage of the opportunity, and, in the time since then, my life 
has transformed into something better than I ever imagined.  Many native Russian speakers around the world have contributed 
to this transformation, and I want to thank several people in particular: Vladimir Bychkov and his wife Nina Nikitina, 
Alexandra Rezova, Misha Kreshchuk, Serdar Kurbanov, Dr. Maxim Konyushikhin and his wife Valentina Maslova, and Professor Misha 
Shifman.

%%%%%%%%%%%%%%%%%%%%%%%%%%%%%%%%%%%%%%%%%%%%%%%%%%%%%%%%%%%%%%%%%%%%%%%%%%%%%%%%
